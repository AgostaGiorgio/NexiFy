\begin{flushleft}
    {Per la realizzazione della piattaforma sarà necessario l'utilizzo di CDN per distribuire sul territorio contenuti multimediali mantenendo questi ultimi sempre accessibili in maniera ottimale da qualsiasi posizione geografica. Deve inoltre essere possibile, in maniera efficiente, estrarre dati e statistiche sugli accessi alla CDN. Si useranno basi di dati cloud-based per mantenere informazioni su dati utenti, e implementando funzionalità lato server per gestire le richieste degli utenti, la loro autenticazione e altro.}
\end{flushleft}

\begin{flushleft}
    {La fattibilità del progetto segue dalla grande disponibilità di aziende che offrono servizi cloud, tra cui CDN, a prezzi contenuti: per esempio AWS CloudFront. Questi servizi includono anche delle API per interfacciarsi in maniera efficiente con la CDN.}
\end{flushleft}

\begin{flushleft}
    {La piattaforma acquisirà utenti sfruttando:}
    
    \begin{itemize}
        \item {La disponibilità di contenuti sia video che musicali, non presente in piattaforme quali Netflix e Spotify}
        
        \item {La possibilità per piccoli creatori di pubblicare autonomamente contenuti, ma senza degradare la qualità dei contenuti (come invece avviene in piattaforme quali YouTube), grazie alla quota annuale da pagare per poter pubblicare}
    \end{itemize}
\end{flushleft}