 In questa sezione saranno esplicitati tutti i termini necessari alla comprensione del progetto.

\begin{itemize}
	%\item  \textbf{multimedia-manager}: servizio usato da partner per la gestione (caricamento/eliminazione) dei propri video e/o tracce audio. %viene citato solamente qui nel glossario
	\item \textbf{Piano di abbonamento}: pacchetto di funzionalità, acquistabile dagli utenti per un certo periodo (e rinnovabile alla fine del periodo). A volte è indicato impropriamente come "abbonamento"
	\item \textbf{API}: interfaccia offerta all'esterno di un sistema, per poter usare delle funzionalità interne al sistema
	\item \textbf{CDN}: sistema di server largamente distribuita, usata per la distribuzione di contenuti quali tracce video e audio.
	\item \textbf{ORM}: tecnica per interfacciarsi con basi di dati relazionali, astraendo dall'implementazione del dbms
	\item \textbf{Piattaforma}: componente del sistema accessibile dagli utenti.
	\item \textbf{Server}: macchina logica (composta da uno o piu macchine fisiche) su cui risiede la piattaforma in parte o nella sua totalità.
	\item \textbf{RUP}: Rational Unified Process, processo di sviluppo basato su fasi temporali, iterazioni e attività.
	\item \textbf{Risorsa Multimediale}: intendiamo un file di tipo multimediale, quindi un video, una canzone, ecc. (a volte è indicata solamente come risorsa)
	\item \textbf{Prodotto}: si intende un singolo prodotto video o audio creato da un partner (e.g: un singolo film, una singola canzone, un singolo episodio di una serie tv, ecc.)
	\item \textbf{Playlist}: lista ordinata di prodotti
	\item \textbf{Contenuto}: un singolo prodotto o una playlist
\end{itemize}