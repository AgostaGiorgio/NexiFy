 In questa sezione saranno esplicitati tutti i termini necessari alla comprensione del progetto.

\begin{itemize}
    	\item \textbf{Piattaforma multimediale on-demand}: sistema strutturato in grado di fornire su richiesta degli utenti una risorsa multimediale (audio, video), consiste in un  \textbf{multimedia-manager}, di una  \textbf{interfaccia web} e di un  \textbf{interfaccia per dispositivi mobili}.
	\item \textbf{Interfaccia web}: sistema online che permette di interagire con la  \textbf{piattaforma multimediale on-demand} riprodurre video e audio oltre che accedere al  \textbf{multimedia-manager}.
	\item \textbf{Interfaccia per dispositivi mobile}:  sistema scaricabile su dispositivi come smartphone e tablet che permette di interagire con la  \textbf{piattaforma multimediale on-demand} riprodurre video e audio oltre che accedere al  \textbf{multimedia-manager}.
	\item  \textbf{multimedia-manager}: servizio usato da partner per la gestione (caricamento/eliminazione) dei propri video e/o tracce audio.
	\item \textbf{Abbonamento}: pacchetto di funzionalità accessibili dai vari utenti.
	\item \textbf{CDN}: sistema di server largamente distribuita usata per la distribuzioni di contenuti quali tracce video e audio.
	\item \textbf{Server}: macchina fisica su cui risiede l'intera piattaforma.
\end{itemize}