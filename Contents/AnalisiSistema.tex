\subsection{Analisi Nomi-Verbi}
Viene qui rappresentato l’elenco delle funzionalità che il sistema deve avere, sulla base dal Documento di Visione.
Inoltre, sono presenti le definizioni necessarie.
La legenda spiega come interpretare le informazioni.

\subsubsection{Legenda}

\begin{itemize}
    \item Nomi propri: istanze;
    \item Nomi comuni o predicati nominali: \textcolor{red}{classi} o \textcolor{orange}{attributi};
    %\item \textcolor{green}{Verbi transitivi}: metodi;
    \item \textcolor{green}{Verbi}: metodi;
    \item \textcolor{cyan}{Verbi modali}: precondizioni, postcondizioni, o condizioni di invarianza;
    %\item \textcolor{blue}{Verbi intransitivi}: eccezioni o eventi dipendenti dal tempo;
    \item \textcolor{violet}{Aggettivi}: valore di attributo o classe;
    \item  \underline{Riferimenti} alle definizioni.
\end{itemize}

\subsubsection{Definizioni}
\begin{enumerate}
    \item \textcolor{red}{Utente del sistema}: fornito di informazioni riguardanti \textcolor{orange}{nome}, \textcolor{orange}{cognome},
    \textcolor{orange}{data di nascita}, \textcolor{orange}{email}, \textcolor{orange}{password}, \textcolor{orange}{metodi di pagamento}
    Ha la possibilità di sottoscrivere \textcolor{red}{abbonamenti};
    \item \textcolor{red}{Piano di abbonamento}: selezionabile dagli utenti, ha proprietà quali \textcolor{orange}{nome},
    \textcolor{orange}{prezzo}, \textcolor{orange}{durata}, \textcolor{orange}{disponibilità di sottoscrizione}.
    Esso raggruppa una lista di \textcolor{red}{servizi} offerti;
    \item \textcolor{red}{Servizio}: caratterizzato da \textcolor{orange}{ID} e \textcolor{orange}{descrizione};
    \item \textcolor{red}{Prodotto multimediale}: oltre agli elementi identificativi, è caratterizzato da \textcolor{orange}{genere},
    \textcolor{orange}{visibilità}, \textcolor{red}{Utente} \textcolor{orange}{proprietario}. Esso può essere di tipo
    \textcolor{violet}{video} o \textcolor{violet}{audio}: nel primo caso è fornito di \textcolor{orange}{traccia audio} e
    \textcolor{orange}{traccia video}, nel secondo solo \textcolor{orange}{traccia audio}, ma con accompagnamento di
    \textcolor{orange}{lyrics} o \textcolor{orange}{video musicale};
    \item \textcolor{orange}{Playlist}: contenente una lista di \textcolor{red}{prodotti multimediali}.\\
    Casi particolari di essa sono:
    \begin{enumerate}
        \item \textcolor{red}{Serie TV}: formata solamente da \textcolor{violet}{prodotti video}, tutti pubblicati dallo stesso \textcolor{red}{utente};
        \item \textcolor{red}{Album}: formato solamente da \textcolor{violet}{prodotti audio}, tutti pubblicati dallo stesso \textcolor{red}{utente};
    \end{enumerate}
    \item \textcolor{red}{Visualizzazione}: proveniente da un \textcolor{red}{utente} verso un \textcolor{red}{prodotto multimediale},
    considera anche l'\textcolor{orange}{istante} in cui avviene;
    \item \textcolor{red}{Segnalazione}: relativa ad un \textcolor{red}{prodotto multimediale}, caratterizzata dall' \textcolor{red}{utente} che 
    la effettua e una \textcolor{orange}{motivazione};
    \item \textcolor{red}{Coda di riproduzione}: relativa ad un singolo \textcolor{red}{utente}, composta da una lista di  
    \textcolor{red}{prodotti multimediali};
\end{enumerate}
