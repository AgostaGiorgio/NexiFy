
\subsection{Analisi Nomi-Verbi}
Viene qui rappresentato l’elenco delle funzionalità che il sistema deve avere, sulla base dal Documento di Visione.
Inoltre, sono presenti le definizioni necessarie.
La legenda spiega come interpretare le informazioni.

\subsubsection{Legenda}

\begin{itemize}
    \item Nomi propri: istanze;
    \item Nomi comuni o predicati nominali: \textcolor{red}{classi} o \textcolor{orange}{attributi};
    %\item \textcolor{teal}{Verbi transitivi}: metodi;
    \item \textcolor{teal}{Verbi}: metodi;
    \item \textcolor{cyan}{Verbi modali}: precondizioni, postcondizioni, o condizioni di invarianza;
    %\item \textcolor{blue}{Verbi intransitivi}: eccezioni o eventi dipendenti dal tempo;
    \item \textcolor{violet}{Aggettivi}: valore di attributo o classe;
    \item  \underline{Riferimenti} alle definizioni.
\end{itemize}

\subsubsection{Definizioni}
\begin{enumerate}
    \item \textcolor{red}{Utente del sistema}: fornito di informazioni riguardanti \textcolor{orange}{nome}, \textcolor{orange}{cognome},
    \textcolor{orange}{data di nascita}, \textcolor{orange}{email}, \textcolor{orange}{password}, \textcolor{orange}{metodi di pagamento}
    Ha la possibilità di sottoscrivere \textcolor{red}{abbonamenti}. Un tipo particolare di utente è il \textcolor{red}{partner}, che pubblica
    sulla piattaforma;
    \item \textcolor{red}{Piano di abbonamento}: selezionabile dagli utenti, ha proprietà quali \textcolor{orange}{nome},
    \textcolor{orange}{prezzo}, \textcolor{orange}{durata}, \textcolor{orange}{disponibilità di sottoscrizione}.
    Esso raggruppa una lista di \textcolor{red}{servizi} offerti;
    \item \textcolor{red}{Servizio}: caratterizzato da \textcolor{orange}{ID} e \textcolor{orange}{descrizione};
    \item \textcolor{red}{Prodotto multimediale}: oltre agli elementi identificativi, è caratterizzato da \textcolor{orange}{genere},
    \textcolor{orange}{visibilità}, \textcolor{red}{Utente} \textcolor{orange}{proprietario}. Esso può essere di tipo
    \textcolor{violet}{video} o \textcolor{violet}{audio}: nel primo caso è fornito di \textcolor{orange}{traccia audio} e
    \textcolor{orange}{traccia video}, nel secondo solo \textcolor{orange}{traccia audio}, ma con accompagnamento di
    \textcolor{orange}{lyrics} o \textcolor{orange}{video musicale};
    \item \textcolor{red}{Playlist}: contenente una lista di \textcolor{red}{prodotti multimediali}.\\
    Casi particolari di essa sono:
    \begin{enumerate}
        \item \textcolor{red}{Serie TV}: formata solamente da \textcolor{violet}{prodotti video}, tutti pubblicati dallo stesso \textcolor{red}{utente};
        \item \textcolor{red}{Album}: formato solamente da \textcolor{violet}{prodotti audio}, tutti pubblicati dallo stesso \textcolor{red}{utente};
    \end{enumerate}
    \item \textcolor{red}{Visualizzazione}: proveniente da un \textcolor{red}{utente} verso un \textcolor{red}{prodotto multimediale},
    considera anche l'\textcolor{orange}{istante} in cui avviene;
    \item \textcolor{red}{Segnalazione}: relativa ad un \textcolor{red}{prodotto multimediale}, caratterizzata dall' \textcolor{red}{utente} che 
    la effettua e una \textcolor{orange}{motivazione};
    \item \textcolor{red}{Coda di riproduzione}: relativa ad un singolo \textcolor{red}{utente}, composta da una lista di  
    \textcolor{red}{prodotti multimediali};
    \item \textcolor{red}{Amministratore del sistema}: si occupa della gestione del sistema.
\end{enumerate}

\subsubsection{Funzionalità}
\begin{enumerate}
    \item L' \textcolor{red}{amministratore} \textcolor{teal}{effettua} diverse operazioni sugli \textcolor{red}{abbonamenti}.
    Egli \textcolor{teal}{crea} \textcolor{red}{abbonamenti} e \textcolor{cyan}{può} \textcolor{teal}{attivare} e
    \textcolor{teal}{disattivare} alcuni già \textcolor{violet}{creati}, \textcolor{teal}{modificando} la loro \textcolor{orange}{disponibilità}.
    Inoltre, egli \textcolor{teal}{aggiunge} e \textcolor{teal}{rimuove} \textcolor{red}{servizi} da \textcolor{red}{abbonamenti}.
    L' \textcolor{red}{amministratore} \textcolor{cyan}{può} anche \textcolor{teal}{recuperare} \textcolor{orange}{informazioni} riguardanti
    \textcolor{red}{servizi} (\textcolor{violet}{generici} o in un particolare \textcolor{red}{abbonamento}) e \textcolor{red}{piani di abbonamento};
    \item L' \textcolor{red}{amministratore} \textcolor{teal}{effettua} pagamenti verso i \textcolor{red}{partner}.
    \item L' \textcolor{red}{amministratore} \textcolor{cyan}{può} \textcolor{teal}{sospendere} l'account di un \textcolor{red}{utente};
    \item L' \textcolor{red}{utente} \textcolor{teal}{effettua} operazioni sul proprio account. Egli \textcolor{teal}{si registra},
    \textcolor{teal}{fornendo} i propri dati, \textcolor{teal}{modifica} il proprio profilo ed \textcolor{teal}{effettua}
    login e logout per \textcolor{teal}{autenticarsi} sulla piattaforma.
    \item L' \textcolor{red}{utente} \textcolor{teal}{richiede} ricerche al \textcolor{red}{sistema}. Queste sono relative
    ad un particolare contenuto (\textcolor{red}{prodotto} o \textcolor{red}{playlist}) oppure indirizzate a contenuti
    \textcolor{violet}{popolari}. Inoltre, il \textcolor{red}{sistema} \textcolor{teal}{suggerisci} contenuti agli
    \textcolor{red}{utenti}, sulla base delle loro \textcolor{orange}{preferenze};
    \item L' \textcolor{red}{utente} \textcolor{teal}{sottoscrive} nuovi \textcolor{red}{abbonamenti} e ne \textcolor{teal}{disdice} di già
    \textcolor{orange}{sottoscritti}. Egli \textcolor{cyan}{può} \textcolor{teal}{cambiare} un \textcolor{red}{abbonamento} con un altro,
    pagando eventualmente un sovrapprezzo. Il \textcolor{red}{sistema} \textcolor{teal}{gestisce} il rinnovo automatico;
    \item Il \textcolor{red}{partner} \textcolor{teal}{pubblica} nuovi \textcolor{red}{prodotti} sulla piattaforma. Egli \textcolor{teal}{compila}
    le \textcolor{orange}{informazioni di base}. Inoltre, nel caso di un 
    \textcolor{red}{prodotto} \textcolor{violet}{video}, egli \textcolor{teal}{carica} il file video (che comprende audio) e \textcolor{violet}{opzionalmente} sottotitoli.
    Invece, per un \textcolor{red}{prodotto} \textcolor{violet}{audio}, egli \textcolor{teal}{carica} il file audio e \textcolor{violet}{opzionalmente} video musicale o lyrics.
    Egli, \textcolor{cyan}{può} anche \textcolor{teal}{cambiare} lo stato della pubblicazione, tra \textcolor{violet}{pubblico} e \textcolor{violet}{privato};
    \item L' \textcolor{red}{utente} \textcolor{teal}{riproduce} \textcolor{red}{prodotti} (sia video che audio) con il player. Egli \textcolor{cyan}{può} \textcolor{teal}{mettere}
    il player \textcolor{violet}{in pausa} e \textcolor{violet}{in riproduzione} e \textcolor{teal}{spostare} il punto di riproduzione. 
    Egli \textcolor{cyan}{può} anche \textcolor{teal}{riprodurre} audio in background;
    \item L' \textcolor{red}{utente} \textcolor{teal}{effettua} operazioni sulle \textcolor{red}{playlist}. Egli crea nuove \textcolor{red}{playlist}, \textcolor{violet}{vuote};
    \textcolor{teal}{aggiunge} e \textcolor{teal}{rimuove} \textcolor{red}{prodotti} ad/da esse;  \textcolor{teal}{cambia} la loro \textcolor{orange}{visibilità} (\textcolor{violet}{pubblica} o \textcolor{violet}{privata});
    \textcolor{teal}{riproduce} playlist, che \textcolor{cyan}{devono}  \textcolor{teal}{contenere} almeno un  \textcolor{red}{prodotto};
    \item Il \textcolor{red}{partner} \textcolor{teal}{crea} una \textcolor{red}{serie TV};
    \item Il \textcolor{red}{partner} \textcolor{teal}{crea} un \textcolor{red}{album};
    \item L' \textcolor{red}{utente} \textcolor{cyan}{può} \textcolor{teal}{valutare} contenuti. Egli \textcolor{cyan}{può} \textcolor{teal}{votare} il contenuto
    e/o  \textcolor{teal}{commentarlo}. Il commento \textcolor{cyan}{può} \textcolor{teal}{essere eliminato} dallo stesso \textcolor{red}{utente};
    \item L' \textcolor{red}{utente} \textcolor{cyan}{può} \textcolor{teal}{segnalare} \textcolor{red}{prodotti} e l' \textcolor{red}{amministratore} \textcolor{teal}{controllare} le 
    \textcolor{red}{segnalazioni} di un certo \textcolor{red}{utente};
    \item L' \textcolor{red}{utente} \textcolor{teal}{ottiene} la cronologia dei contenuti visualizzati;
    \item L' \textcolor{red}{utente} \textcolor{teal}{effettua} il download di un certo \textcolor{red}{prodotto};
    \item L' \textcolor{red}{utente} \textcolor{teal}{riproduce} spot pubblicitari.
    \item Il \textcolor{red}{sistema} \textcolor{teal}{calcola} il voto di un certo contenuto, sulla base dei voti ricevuti.
    \item L' \textcolor{red}{utente} \textcolor{teal}{effettua} operazioni sulla \textcolor{red}{coda di riproduzione}. Egli \textcolor{teal}{aggiunge}
    e \textcolor{teal}{rimuove} \textcolor{red}{prodotti} da essa. Inoltre, egli \textcolor{cyan}{può} \textcolor{teal}{visualizzare} i 
    \textcolor{red}{prodotti} presenti nella \textcolor{red}{coda}.
\end{enumerate}


%============================ SCHEDE CRC =====================================

\subsection{Schede CRC - Class-Responsability-Collaboration}

Che cosa sono\\

\begin{center}
    \begin{tabular}{ |p{3cm}|p{3cm}|p{3cm}|p{3cm}| }
        Nome & \multicolumn{3}{|p{9cm}|}{cell2} \\\hline
        SuperClassi & \multicolumn{3}{|p{9cm}|}{cell2} \\\hline
        SottoClassi & \multicolumn{3}{|p{9cm}|}{cell2} \\\hline
        Attributi & \multicolumn{3}{|p{9cm}|}{cell2} \\\hline
        \multicolumn{4}{|p{12cm}|}{Responsabilit\'a} \\\hline
        \multicolumn{4}{|p{12cm}|}{Nome} \\\hline
        \multicolumn{2}{|p{6cm}|}{nome\_1} & \multicolumn{2}{|p{6cm}|}{collaboratore\_1} \\\hline
    \end{tabular}
\end{center}
