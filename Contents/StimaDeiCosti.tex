\newcolumntype{P}[1]{>{\centering\arraybackslash}p{#1}}
\captionsetup[table]{name=Tabella}

Presentiamo una stima dei costi, che utilizza la tecnica dei \emph{Use Case Point (UCP)}, necessaria alla realizzazione del progetto Nexify. Nel seguito effettueremo, nell'ordine:
\begin{itemize}
\item una stima della complessità degli attori;
\item una stima della complessità dei casi d'uso;
\item calcoleremo i fattori di aggiustamento.
\end{itemize}

\noindent \\ Vengono inoltre illustrate le tabelle utilizzate ai fini del calcolo.

\begin{table}[hb]
    \centering
\begin{tabular}{ |P{3.5cm}|P{5cm}|P{3.5cm}|  }
\hline
\textbf{Classificazione dell'attore} & \textbf{Tipo dell'attore} & \textbf{Peso} \\\hline
Semplice & Sistema esterno che interagisce col sistema attraverso un'API	 & 1\\\hline
Medio & Sistema esterno che deve interagire col sistema utilizzando protocolli di comunicazione standard (es. TCP/IP, FTP, HTTP, database) & 2\\\hline
Complesso & Attore umano che accede utilizzando un'interfaccia grafica (GUI)	& 3 \\\hline
\end{tabular}
\caption{Unadjusted Actor Weight (UAW)}
\end{table}

\begin{table}[hb]
    \centering
\begin{tabular}{ |P{3.5cm}|P{5cm}|P{3.5cm}|  }
\hline
\textbf{Tipo di use case} & \textbf{No of Transactions} & \textbf{Weighting Factor} \\\hline
Simple & \textless =3 & 5\\\hline
Average & 4 to 7 & 10\\\hline
Complex & \textgreater = 8 & 15\\\hline
\end{tabular}
\caption{Unadjusted Use Case Weight (UUCW)}
\end{table}

\begin{table}[hb]
    \centering
\begin{tabular}{ |P{3.5cm}|P{5cm}|P{3.5cm}|  }
\hline
\textbf{ID} & \textbf{Nome} & \textbf{Peso} \\\hline
A\_1.1 & UtenteNonAutenticato & 3\\
A\_1.2 & UtenteAutenticato & 3\\\hline
A\_2.1.1 & ManagerAbbonamenti & 3\\
A\_2.1.2 & ManagerSegnalazioni & 3\\\hline
A\_3 & Pagamento & 1\\\hline
A\_4 & Database & 2\\\hline
A\_5 & CDN & 2\\\hline
A\_6 & Timer & 1\\\hline
\end{tabular}
\caption{Attori}
\end{table}

\begin{table}[hb]
    \centering
\begin{tabular}{ |P{3.5cm}|P{5cm}|P{3.5cm}|  }
\hline
\textbf{ID} & \textbf{Nome} & \textbf{Peso} \\\hline
UC\_1 & UC\_CreaAbbonamento & 10\\
UC\_2 & UC\_RecuperaAbbonamentiEsistenti & 5\\
UC\_3 & UC\_RecuperaServizi & 5\\
UC\_4 & UC\_DisattivaAbbonamento & 5\\
UC\_5 & UC\_AttivaAbbonamento & 5\\
UC\_6 & UC\_AggiungiServizioAbbonamento & 10\\
UC\_7 & UC\_RimuoviServizioAbbonamento & 10\\
UC\_8 & UC\_RecuperaServiziAbbonamento & 5\\
UC\_9 & UC\_RecuperaPianiAbbonamentoUtente & 5\\
UC\_10 & UC\_EffettuaPagamentoPartner & 10\\
UC\_11 & UC\_CalcolaImportoDaPagare & 5\\
UC\_12 & UC\_SospendiAccount & 5\\\\
UC\_13 & UC\_EffettuaRegistrazione & 10\\\\
UC\_14 & UC\_ModificaProfilo & 10\\
UC\_15 & UC\_EffettuaLogin &1 0\\
UC\_16 & UC\_EffettuaLogout & 5\\
UC\_17 & UC\_SottoscriviAbbonamento & 10\\
UC\_18 & UC\_DisdiciAbbonamento & 5\\
UC\_19 & UC\_CambiaAbbonamento & 15\\
UC\_20 & UC\_CreaProdotto & 10\\
UC\_21 & UC\_ModificaInformazioniDiBase & 10\\
UC\_22 & UC\_CaricaFile & 10\\
UC\_23 & UC\_CambiaStatoPubblicazione & 5\\
UC\_24 & UC\_RiproduciVideo & 10\\
UC\_25 & UC\_RiproduciMusica & 10\\
UC\_26 & UC\_PausaPlayer & 5\\
UC\_27 & UC\_SpostaPuntoRiproduzionePlayer & 5\\
UC\_28 & UC\_SegnalaProdotto & 10\\
UC\_29 & UC\_OttieniSegnalazioni & 5\\
UC\_30 & UC\_ChiudiSegnalazione & 5\\
UC\_31 & UC\_RicercaContenuto & 10\\
UC\_32 & UC\_RicercaPopolari & 15\\
UC\_33 & UC\_SuggerisciContenuti & 15\\
UC\_34 & UC\_VisualizzaPubblicità & 5\\
UC\_35 & UC\_CreaPlaylist & 10\\
UC\_36 & UC\_AggiungiProdottoPlaylist & 5\\
UC\_37 & UC\_RimuoviProdottoPlaylist & 5\\
UC\_38 & UC\_CambiaVisibilitáPlaylist & 5\\
UC\_39 & UC\_RiproduciPlaylist & 5\\
UC\_40 & UC\_AggiungiProdottoAllaCoda & 5\\
UC\_41 & UC\_RimuoviProdottoDallaCoda & 5\\
UC\_42 & UC\_MostraStatoCoda & 5\\
UC\_43 & UC\_RiproduciCoda & 10\\
UC\_44 & UC\_CreaSerieTv & 5\\
UC\_45 & UC\_CreaAlbum & 5\\
UC\_46 & UC\_GestisciScadenzeAbbonamenti & 10\\
UC\_47 & UC\_GestisciAbbonamenti & 10\\
UC\_48 & UC\_CalcolaQualitáContenuto & 10\\
UC\_49 & UC\_OttieniCronologia & 5\\
\end{tabular}
\caption{Casi d'uso}
\end{table}

\begin{table}[b]
\noindent{\large \textbf{Revisioni 9}} \\ \\
\begin{tabular}{|c | c | c | c|} 
 	\hline
	 Numero & Data & Descrizione \\ [0.5ex] 
	\hline\hline
	1 & 20/02/2020 & Stesura iniziale con requisiti funzionali principali \\
	\hline
	2 & 26/02/2020 & Revisione dei principali use case\\
	\hline
\end{tabular}
\end{table}