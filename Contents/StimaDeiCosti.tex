\newcolumntype{P}[1]{>{\centering\arraybackslash}p{#1}}
\captionsetup[table]{name=Tabella}

Presentiamo una stima dei costi, che utilizza la tecnica dei \emph{Use Case Point}, necessaria alla realizzazione del progetto Nexify. Nel seguito effettueremo, nell'ordine:
\begin{itemize}
\item una stima della complessità degli attori;
\item una stima della complessità dei casi d'uso;
\item calcoleremo i fattori di aggiustamento.
\end{itemize}

\noindent \\ Vengono inoltre illustrate le tabelle utilizzate ai fini del calcolo.

\begin{table}[htb]
    \centering
\begin{tabular}{ |P{3.5cm}|P{5cm}|P{3.5cm}|  }
\hline
\textbf{Actor Type} & \textbf{Type of Actor} & \textbf{Weightening factor} \\\hline
Simple & External system that must interact with the system using a well-defined API	 & 1\\\hline
Average & External system that must interact with the system using standard communication protocols (e.g. TCP/IP, FTP, HTTP, database) & 2\\\hline
Complex & Human actor using a GUI application interface	& 3 \\\hline
\end{tabular}
\caption{Unadjusted Actor Weight (UAW)}
\end{table}

\begin{table}[htb]
    \centering
\begin{tabular}{ |P{3.5cm}|P{5cm}|P{3.5cm}|  }
\hline
\textbf{Use Case Type} & \textbf{No of Transactions} & \textbf{Weighting Factor} \\\hline
Simple & \textless =3 & 5\\\hline
Average & 4 to 7 & 10\\\hline
Complex & \textgreater = 8 & 15\\\hline
\end{tabular}
\caption{Unadjusted Use Case Weight (UUCW)}
\end{table}