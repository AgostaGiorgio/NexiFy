\subsection{Tipologia}
I test saranno effettuati in due diverse tipologie:
\begin{itemize}
    \item Black box: valutano le funzionalità dell'applicazione a partire dall'interfaccia utente, che
          sia quella degli amministratori o degli utenti fruitori del servizio. Quindi si ignorano le dinamiche
          interne al sistema. Contengono anche test di esperienza e usabilità utente, ovvero quelli che richiedono
          un interazione umana.
    \item White box: provano la correttezza del sistema con test unitari sui suoi componenti interni,
          così come le interazioni tra i vari componenti.
\end{itemize}

\subsection{Ambiente di test}
Per i test di tipo White Box vengono definiti i seguenti programmi di supporto:
\begin{itemize}
    \item Un programma per intercettare le comunicazioni client-server e server-server, per controllare che le
          comunicazioni avvengano in modo corretto;
    \item Un programma per verificare lo stato ed il contenuto del database;
    \item Un profiler per controllare lo stato delle performance e verificare l’assenza di colli di bottiglia;
\end{itemize}
L’ambiente viene re-inizializzato per ogni test, in quanto eseguito su un container.

\subsection{Dati su cui vengono eseguiti i test}

\subsubsection{Dati non presenti nel database}
\begin{itemize}
    \item Token autenticazione auth\_t\_np\_nv
    \item Nome abbonamento nome\_abb\_np\_nv
    \item Nome abbonamento nome\_abb\_np\_v
    \item Prezzo abbonamento prezzo\_abb\_np\_v
    \item Prezzo abbonamento prezzo\_abb\_np\_nv
    \item Durata abbonamento durata\_abb\_np\_v
    \item Durata abbonamento durata\_abb\_np\_nv
    \item Stringa nome\_np\_v
    \item Stringa cognome\_np\_v
    \item Email email\_np\_v
    \item String password\_np\_v
    \item Commento commento\_np\_v
    \item Commento commento\_np\_nv
    \item String descrizione\_np\_v
    \item String descrizione\_np\_nv
    \item FileVideo file\_video\_np\_v
    \item FileAudio file\_video\_np\_v
    \item Lista<File, Lingua> subs\_np\_v
    \item Lista<FileAudio, Lingua> audios\_np\_v
    \item FileVideo lyrics\_np\_v;
    \item FileVideo video\_np\_v;
    \item Segnalazione segnalazione\_np\_v
    \item String titolo\_np\_v;
    \item \item String titolo\_np\_nv;
    \item Prodotto contenuto\_np\_v

\end{itemize}

\subsubsection{Dati presenti nel database}
\begin{itemize}
    \item Token autenticazione amministratore auth\_admin\_t\_p\_v
    \item Token autenticazione auth\_t\_p\_v
    \item PianoDiAbbonamento piano\_abb\_p\_v
    \item PianoDiAbbonamento piano\_abb\_p\_nv
    \item PianoDiAbbonamento piano\_abb\_p\_nv
    \item AccountUtente utente\_p\_v
    \item Email email\_p\_nv
    \item Stringa password\_p\_nv
    \item Commento commento\_p\_v
    \item Coda di riproduzione coda\_p\_v
    \item Coda di riproduzione coda\_p\_nv
    \item Prodotto contenuto\_p\_v
    \item Prodotto contenuto\_p\_nv
    \item String genere\_p\_v
    \item String genere\_p\_nv
    \item String visibilita\_p\_v
    \item AudioType audio\_type\_p\_v
    \item Playlist playlist\_p\_v
    \item Playlist playlist\_p\_nv
    \item Segnalazione segnalazione\_p\_v
    \item \item String titolo\_p\_v;

\end{itemize}

\subsection{Descrizione priorità dei test}

Vengono definite le seguenti priorità per classificare i vari test in base a quanto importante sia la
loro corretta e periodica esecuzione sull'integrità del sistema.
\begin{itemize}
    \item H (alta priorità): relativa ai test che devono essere eseguiti correttamente per garantire l'integrità
          del sistema;
    \item M (media priorità): relativa ai test che devono essere eseguiti correttamente per garantire l'integrità
          del sistema, ma solo successivamente ai test di priorità H;
    \item L (bassa priorità): relativa ai test che possono essere eseguiti, per garantire una migliore funzionalità
          del sistema, ma solo successivamente ai test di priorità M.
\end{itemize}


\begin{longtable}{| p{.10\textwidth} | p{.70\textwidth} | p{.14\textwidth} |}
    \caption{Casi d'uso}                                                         \\
    \hline
    \textbf{ID TEST} & \textbf{ID UC}                        & \textbf{Priorità} \\\hline
    T\_1             & UC\_GestisciAbbonamenti               & H                 \\\hline
    T\_2             & UC\_CreaAbbonamento                   & H                 \\\hline
    T\_3             & UC\_RecuperaAbbonamentiEsistenti      & M                 \\\hline
    T\_4             & UC\_RecuperaServizi                   & M                 \\\hline
    T\_5             & UC\_DisattivaAbbonamento              & H                 \\\hline
    T\_6             & UC\_AggiungiServizioAbbonamento       & H                 \\\hline
    T\_7             & UC\_RimuoviServizioAbbonamento        & H                 \\\hline
    T\_8             & UC\_RecuperaServiziAbbonamento        & M                 \\\hline
    T\_9             & UC\_RecuperaPianiAbbonamentoUtente    & M                 \\\hline
    T\_10            & UC\_EffettuaPagamentoPartner          & H                 \\\hline
    T\_11            & UC\_CalcolaImportoDaPagare            & H                 \\\hline
    T\_12            & UC\_GestisciSottoscrizioniAbbonamenti & H                 \\\hline
    T\_13            & UC\_SottoscriviAbbonamento            & H                 \\\hline
    T\_14            & UC\_DisdiciAbbonamento                & H                 \\\hline
    T\_15            & UC\_CambiaAbbonamento                 & H                 \\\hline
    T\_16            & UC\_GestisciScadenzeAbbonamenti       & H                 \\\hline
    T\_17            & UC\_SospendiAccount                   & M                 \\\hline
    T\_18            & UC\_RiattivaAccount                   & M                 \\\hline
    T\_19            & UC\_RiattivaAccountAutomatico         & H                 \\\hline
    T\_20            & UC\_EffettuaRegistrazione             & H                 \\\hline
    T\_21            & UC\_ModificaProfilo                   & L                 \\\hline
    T\_22            & UC\_EffettuaLogin                     & M                 \\\hline
    T\_23            & UC\_EffettuaLogout                    & M                 \\\hline
    T\_24            & UC\_OttieniCronologia                 & L                 \\\hline
    T\_25            & UC\_GestisciProdotti                  & H                 \\\hline
    T\_26            & UC\_CreaProdotto                      & H                 \\\hline
    T\_27            & UC\_ModificaInformazioniDiBase        & M                 \\\hline
    T\_28            & UC\_CaricaFile                        & H                 \\\hline
    T\_29            & UC\_CambiaVisibilitàProdotto          & M                 \\\hline
    T\_30            & UC\_RiproduciProdotto                 & M                 \\\hline
    T\_31            & UC\_StreamingVideo                    & M                 \\\hline
    T\_32            & UC\_StreamingMusica                   & M                 \\\hline
    T\_33            & UC\_PausaPlayer                       & M                 \\\hline
    T\_34            & UC\_SpostaPuntoRiproduzionePlayer     & M                 \\\hline
    T\_35            & UC\_DownloadProdotto                  & M                 \\\hline
    T\_36            & UC\_VisualizzaPubblicità              & M                 \\\hline
    T\_37            & UC\_RiproduciAudioInBackground        & M                 \\\hline
    T\_38            & UC\_SegnalaProdotto                   & M                 \\\hline
    T\_39            & UC\_GestisciSegnalazioni              & H                 \\\hline
    T\_40            & UC\_OttieniSegnalazioni               & M                 \\\hline
    T\_41            & UC\_ChiudiSegnalazione                & M                 \\\hline
    T\_42            & UC\_RiapriSegnalazione                & M                 \\\hline
    T\_43            & UC\_VisualizzaProdottoSegnalato       & M                 \\\hline
    T\_44            & UC\_RicercaContenuto                  & M                 \\\hline
    T\_45            & UC\_RicercaPopolari                   & L                 \\\hline
    T\_46            & UC\_SuggerisciContenuti               & M                 \\\hline
    T\_47            & UC\_GestisciPlaylist                  & H                 \\\hline
    T\_48            & UC\_CreaPlaylist                      & M                 \\\hline
    T\_49            & UC\_AggiungiProdottoPlaylist          & M                 \\\hline
    T\_50            & UC\_RimuoviProdottoPlaylist           & M                 \\\hline
    T\_51            & UC\_CambiaVisibilitàPlaylist          & M                 \\\hline
    T\_52            & UC\_RiproduciPlaylist                 & M                 \\\hline
    T\_53            & UC\_CreaSerieTv                       & H                 \\\hline
    T\_54            & UC\_CreaAlbum                         & H                 \\\hline
    T\_55            & UC\_AggiungiProdottoAllaCoda          & M                 \\\hline
    T\_56            & UC\_RimuoviProdottoDallaCoda          & M                 \\\hline
    T\_57            & UC\_MostraStatoCoda                   & L                 \\\hline
    T\_58            & UC\_RiproduciCoda                     & M                 \\\hline
    T\_59            & UC\_CalcolaQualitàContenuto           & M                 \\\hline
    T\_60            & UC\_VotaContenuto                     & L                 \\\hline
    T\_61            & UC\_CommentaContenuto                 & L                 \\\hline
    T\_62            & UC\_RimuoviCommento                   & L                 \\\hline
    T\_63            & UC\_AttivaAbbonamento                 & H                 \\\hline
\end{longtable}

\subsection{Realizzazione dei test}

\begin{table}[hb]
    \centering
    \begin{tabular}{ |p{2cm}|p{10cm}|  }
        \hline
        ID          & T\_1                                                               \\\hline
        Caso d'uso  & UC\_GestisciAbbonamenti                                            \\\hline
        Obbiettivo  & Testare l'interfaccia per la gestione dei piani di abbonamento     \\\hline
        Prerequsiti & L'esecuzione del test è effettuata con privilegi di Amministratore \\\hline
        Tipologia   & White box                                                          \\\hline
        Azioni      &
        TEST1:
        \begin{enumerate}[nosep, topsep=0pt]
            \item richiedere la visualizzazione della gestione abbonamenti, fornendo il token di autenticazione \emph{auth\_t\_p\_nv};
            \item verificare che il sistema restituisca errore di permesso negato;
        \end{enumerate}
        \vspace{0.5cm} TEST2:
        \begin{enumerate}[nosep, topsep=0pt]
            \item richiedere la visualizzazione della gestione abbonamenti, fornendo il token di autenticazione \emph{auth\_t\_p\_v};
            \item verificare che il sistema restituisca errore di permesso negato;
        \end{enumerate}
        \vspace{0.5cm}TEST3:
        \begin{enumerate}[nosep, topsep=0pt]
            \item richiedere la visualizzazione della gestione abbonamenti, fornendo il token di autenticazione \emph{auth\_admin\_t\_p\_v};
            \item verificare che il sistema restituisca la pagina con la lista di abbonamenti richiesti;
        \end{enumerate}
        \\\hline
    \end{tabular}
\end{table}

\begin{table}[hb]
    \centering
    \begin{tabular}{ |p{2cm}|p{10cm}|  }
        \hline
        ID          & T\_2                                                                                                                            \\\hline
        Caso d'uso  & UC\_CreaAbbonamento                                                                                                             \\\hline
        Obbiettivo  & Testare la creazione di un nuovo piano di abbonamento                                                                           \\\hline
        Prerequsiti & Il sistema permette di inserire nuovi piani di abbonamento e l'esecuzione del test è effettuata con privilegi di Amministratore \\\hline
        Tipologia   & White box                                                                                                                       \\\hline
        Azioni      &
        TEST1:
        \begin{enumerate}[nosep, topsep=0pt]
            \item fornire i seguenti valori \emph{nome\_abb\_np\_nv}, \emph{prezzo\_abb\_np\_nv}, \emph{durata\_abb\_np\_nv};
            \item verifica che il sistema restituisca errore nella validazione dei dati;
        \end{enumerate}
        \vspace{0.5cm} TEST2:
        \begin{enumerate}[nosep, topsep=0pt]
            \item fornire i seguenti valori \emph{nome\_abb\_np\_nv}, \emph{prezzo\_abb\_np\_v}, \emph{durata\_abb\_np\_v};
            \item verifica che il sistema restituisca errore nell'inserimento di un piano di abbonamento duplicato;
        \end{enumerate}
        \vspace{0.5cm} TEST3:
        \begin{enumerate}[nosep, topsep=0pt]
            \item fornire i seguenti valori \emph{nome\_abb\_np\_v}, \emph{prezzo\_abb\_np\_v}, \emph{durata\_abb\_np\_v};
            \item verifica che il sistema restituisca errore generico nella crezione di un abbonamento oppure creazione effettuata correttamente;
        \end{enumerate}
        \\\hline
    \end{tabular}
\end{table}

\begin{table}[hb]
    \centering
    \begin{tabular}{ |p{2cm}|p{10cm}|  }
        \hline
        ID          & T\_3                                                                           \\\hline
        Caso d'uso  & UC\_RecuperaAbbonamentiEsistenti                                               \\\hline
        Obbiettivo  & Testare che tutti i piani di abbonamenti esistenti sono recuperati             \\\hline
        Prerequsiti & L'esecuzione del test è effettuata con privilegi di Amministratore o di Utente \\\hline
        Tipologia   & Black box                                                                      \\\hline
        Azioni      &
        TEST1:
        \begin{enumerate}[nosep, topsep=0pt]
            \item richiedere di visualizzare la lista degli abbonamenti esistenti;
            \item verificare che vengano mostrati tutti gli abbonamenti esistenti;
        \end{enumerate}
        \\\hline
    \end{tabular}
\end{table}

\begin{table}[hb]
    \centering
    \begin{tabular}{ |p{2cm}|p{10cm}|  }
        \hline
        ID          & T\_4                                                               \\\hline
        Caso d'uso  & UC\_RecuperaServizi                                                \\\hline
        Obbiettivo  & Testare che tutti i servizi esistenti sono recuperati              \\\hline
        Prerequsiti & L'esecuzione del test è effettuata con privilegi di Amministratore \\\hline
        Tipologia   & Black box                                                          \\\hline
        Azioni      &
        TEST1:
        \begin{enumerate}[nosep, topsep=0pt]
            \item richiedere di visualizzare la lista degli servizi esistenti;
            \item verificare che vengano mostrati tutti i servizi esistenti;
        \end{enumerate}
        \\\hline
    \end{tabular}
\end{table}

\begin{table}[hb]
    \centering
    \begin{tabular}{ |p{2cm}|p{10cm}|  }
        \hline
        ID          & T\_5                                                               \\\hline
        Caso d'uso  & UC\_DisattivaAbbonamento                                           \\\hline
        Obbiettivo  & Testare che un abbonamento non sia più sottoscrivibile             \\\hline
        Prerequsiti & L'esecuzione del test è effettuata con privilegi di Amministratore \\\hline
        Tipologia   & Black box                                                          \\\hline
        Azioni      &
        TEST1:
        \begin{enumerate}[nosep, topsep=0pt]
            \item richiedere la disattivazione di \emph{piano\_abb\_v}
            \item verificare che il sistema ritorni il messaggio di avvenuta disattivazione dell'abbonamento;
        \end{enumerate}
        \\\hline
    \end{tabular}
\end{table}

\begin{table}[hb]
    \centering
    \begin{tabular}{ |p{2cm}|p{10cm}|  }
        \hline
        ID          & T\_6                                                                               \\\hline
        Caso d'uso  & UC\_AggiungiServizioAbbonamento                                                    \\\hline
        Obbiettivo  & Testare l'aggiunta del servizio di abbonamento al piano di abbonamento specificato \\\hline
        Prerequsiti & L'esecuzione del test è effettuata con privilegi di Amministratore                 \\\hline
        Tipologia   & White box                                                                          \\\hline
        Azioni      &
        TEST1:
        \begin{enumerate}[nosep, topsep=0pt]
            \item fornire un piano di abbonamento \emph{piano\_abb\_nv};
            \item verificare che il sistema mostri l'errore "Il piano possiede già tutti i servizi";
        \end{enumerate}
        \vspace{0.5cm} TEST2:
        \begin{enumerate}[nosep, topsep=0pt]
            \item fornire un piano di abbonamento \emph{piano\_abb\_v};
            \item verifica che il sistema restituisca la lista di servizi \emph{L} che possono essere aggiunti;
            \item fornire il servizio richiesto scegliendolo da \emph{L};
            \item verificare che il sistema mostri un errore generico oppure modifica effettuata con successo;
        \end{enumerate}
        \\\hline
    \end{tabular}
\end{table}

\begin{table}[hb]
    \centering
    \begin{tabular}{ |p{2cm}|p{10cm}|  }
        \hline
        ID          & T\_7                                                                                  \\\hline
        Caso d'uso  & UC\_RimuoviServizioAbbonamento                                                        \\\hline
        Obbiettivo  & Testare la rimozione del servizio di abbonamento dal piano di abbonamento specificato \\\hline
        Prerequsiti & L'esecuzione del test è effettuata con privilegi di Amministratore                    \\\hline
        Tipologia   & White box                                                                             \\\hline
        Azioni      &
        TEST1:
        \begin{enumerate}[nosep, topsep=0pt]
            \item fornire un piano di abbonamento \emph{piano\_abb\_nv};
            \item verificare che il sistema mostri l'errore "Non ci sono servizi da rimuovere";
        \end{enumerate}
        \vspace{0.5cm} TEST2:
        \begin{enumerate}[nosep, topsep=0pt]
            \item fornire un piano di abbonamento \emph{piano\_abb\_v};
            \item verifica che il sistema restituisca la lista di servizi \emph{L} che possono essere rimossi;
            \item fornire il servizio richiesto scegliendolo da \emph{L};
            \item verificare che il sistema mostri un errore generico oppure modifica effettuata con successo;
        \end{enumerate}
        \\\hline
    \end{tabular}
\end{table}

\begin{table}[hb]
    \centering
    \begin{tabular}{ |p{2cm}|p{10cm}|  }
        \hline
        ID          & T\_8                                                                                       \\\hline
        Caso d'uso  & UC\_RecuperaServiziAbbonamento                                                             \\\hline
        Obbiettivo  & Testare il recupero di tutti i servizi di abbonamento associati ad un piano di abbonamento \\\hline
        Prerequsiti & L'esecuzione del test è effettuata con privilegi di Amministratore o di Utente             \\\hline
        Tipologia   & Black box                                                                                  \\\hline
        Azioni      &
        TEST1:
        \begin{enumerate}[nosep, topsep=0pt]
            \item fornire un piano di abbonamento \emph{piano\_abb\_v};
            \item verificare che vengano mostrati tutti i servizi esistenti associati al piano di abbonamento specificato;
        \end{enumerate}
        \vspace{0.5cm} TEST2:
        \begin{enumerate}[nosep, topsep=0pt]
            \item fornire un piano di abbonamento \emph{piano\_abb\_nv};
            \item verificare la lista dei servizi restituita sia vuota;
        \end{enumerate}
        \\\hline
    \end{tabular}
\end{table}

\begin{table}[hb]
    \centering
    \begin{tabular}{ |p{2cm}|p{10cm}|  }
        \hline
        ID          & T\_9                                                                                                         \\\hline
        Caso d'uso  & UC\_RecuperaPianiAbbonamentoUtente                                                                           \\\hline
        Obbiettivo  & Testare che la lista degli abbonamenti restituita sia quella degli abbonamenti
        sottoscritti dall'utente                                                                                                   \\\hline
        Prerequsiti & L'esecuzione del test è effettuata con privilegi di Amministratore e \emph{utente\_p\_v} è un account attivo \\\hline
        Tipologia   & Black box                                                                                                    \\\hline
        Azioni      &
        TEST1:
        \begin{enumerate}[nosep, topsep=0pt]
            \item richiedere lista \emph{L} piani di abbonamento sottoscritti da \emph{utente\_p\_v};
            \item verificare che \emph{L} sia la lista degli abbonamenti sottoscritti da \emph{utente\_p\_v};
        \end{enumerate}
        \\\hline
    \end{tabular}
\end{table}

\begin{table}[hb]
    \centering
    \begin{tabular}{ |p{2cm}|p{10cm}|  }
        \hline
        ID          & T\_10                                                              \\\hline
        Caso d'uso  & UC\_EffettuaPagamentoPartner                                       \\\hline
        Obbiettivo  & Verificare che il pagamento verso i partner avvenga correttamente  \\\hline
        Prerequsiti & L'esecuzione del test è effettuata con privilegi di Amministratore \\\hline
        Tipologia   & White box                                                          \\\hline
        Azioni      &
        TEST1:
        \begin{enumerate}[nosep, topsep=0pt]
            \item richiedere la lista di tutti gli utenti \emph{LU};
            \item verificare che \emph{LU} sia la lista di tutti gli utenti;
            \item tra tutti gli utenti in \emph{LU}, controllare tutti coloro che
                  hanno un servizio per pubblicare prodotti e salvarli in \emph{LP};
            \item verificare che \emph{LP} sia la lista di tutti gli utenti che
                  hanno un servizio per pubblicare prodotti;
            \item per ogni utente \emph{U} in \emph{LP}:
                  \begin{itemize}
                      \item ottenere coordinate le bancarie \emph{CB} di \emph{U}
                      \item controllare che \emph{CB} siano valide
                      \item ottenere l'importo da pagare \emph{I}
                      \item effettuare pagamento verso \emph{CB} attraverso il sistema
                            di pagamento.
                      \item controllare che se il pagamento è andato a buon fine, venga aggiornato il giorno di
                            pagamento utente e inviata una mail con messaggio di avvenuto pagamento a \emph{U}
                      \item controllare che se il pagamento non è andato a buon fine, venga inviata una mail
                            con messaggio di errore a \emph{U}
                  \end{itemize}

        \end{enumerate}
        \\\hline
    \end{tabular}
\end{table}

\begin{table}[hb]
    \centering
    \begin{tabular}{ |p{2cm}|p{10cm}|  }
        \hline
        ID          & T\_11                                                                                                        \\\hline
        Caso d'uso  & UC\_CalcolaImportoDaPagare                                                                                   \\\hline
        Obbiettivo  & Testare il corretto funzionamento del calcolo dell'importo da pagare ad un utente                            \\\hline
        Prerequsiti & L'esecuzione del test è effettuata con privilegi di Amministratore e \emph{utente\_p\_v} è un account attivo \\\hline
        Tipologia   & White box                                                                                                    \\\hline
        Azioni      &
        TEST1:
        \begin{enumerate}[nosep, topsep=0pt]
            \item richiedere il calcolo dell'importo dovuto, fornendo \emph{utente\_p\_v};
            \item verificare che il sistema consideri l'intervallo di tempo [\emph{DA}, \emph{A}], dove \emph{DA} è null sse non esistono pagamenti precedenti verso quell'utente;
            \item verificare che il sistema consideri la lista dei prodotti relativa ad \emph{utente\_p\_v} e che il totale dovuto \emph{TOT} sia initializzato a 0;
            \item verificare che il sistema ritorni \emph{R = TOT + V}, con V il numero complessivo di visualizzazioni, assieme alla funzione \emph{F} che indica come calcolare l'importo da pagare a partire da \emph{R};
        \end{enumerate}
        \\\hline
    \end{tabular}
\end{table}

\begin{table}[hb]
    \centering
    \begin{tabular}{ |p{2cm}|p{10cm}|  }
        \hline
        ID          & T\_12                                                                               \\\hline
        Caso d'uso  & UC\_GestisciSottoscrizioniAbbonamenti                                               \\\hline
        Obbiettivo  & Testare l'interfaccia per la gestione delle sottoscrizioni dei piani di abbonamento \\\hline
        Prerequsiti & L'esecuzione del test è effettuata con privilegi di Utente                          \\\hline
        Tipologia   & White box                                                                           \\\hline
        Azioni      &
        TEST1:
        \begin{enumerate}[nosep, topsep=0pt]
            \item richiedere la visualizzazione della gestione delle sottoscrizioni dei piani di abbonamento, fornendo il token di autenticazione \emph{auth\_t\_np\_nv};
            \item verificare che il sistema restituisca errore di permesso negato;
        \end{enumerate}
        \vspace{0.5cm} TEST2:
        \begin{enumerate}[nosep, topsep=0pt]
            \item richiedere la visualizzazione della gestione delle sottoscrizioni dei piani di abbonamento, fornendo il token di autenticazione \emph{auth\_t\_v};
            \item verificare che il sistema restituisca la pagina con la lista delle sottoscrizioni richieste;
        \end{enumerate}
        \\\hline
    \end{tabular}
\end{table}

\begin{table}[hb]
    \centering
    \begin{tabular}{ |p{2cm}|p{10cm}|  }
        \hline
        ID          & T\_13                                                             \\\hline
        Caso d'uso  & UC\_SottoscriviAbbonamento                                        \\\hline
        Obbiettivo  & Testare la sottoscrizione di un abbonamento da parte di un utente \\\hline
        Prerequsiti & L'esecuzione del test è effettuata con privilegi di Utente        \\\hline
        Tipologia   & White box                                                         \\\hline
        Azioni      &
        TEST1:
        \begin{enumerate}[nosep, topsep=0pt]
            \item richiedere la sottoscrizione di un abbonamento, fornendo \emph{piano\_abb\_v} e le coordinate bancarie \emph{CB} dell'utente;
            \item verificare che il pagamento venga registrato;
            \item verificare che l'abbonamento venga aggiunto alle sottoscrizioni dell'utente oppure che il sistema restituisca un errore durante l'aggiunta dell'abbonamento;
        \end{enumerate}
        \\\hline
    \end{tabular}
\end{table}

\begin{table}[hb]
    \centering
    \begin{tabular}{ |p{2cm}|p{10cm}|  }
        \hline
        ID          & T\_14                                                                                                                                   \\\hline
        Caso d'uso  & UC\_DisdiciAbbonamento                                                                                                                  \\\hline
        Obbiettivo  & Testare la disdetta di un piano di abbonamento e la relativa disattivazione del rinnovo automatico                                      \\\hline
        Prerequsiti & L'esecuzione del test è effettuata con privilegi di Utente e \emph{piano\_abb\_v} è un piano di abbonamento attivo dell'utente corrente \\\hline
        Tipologia   & White box                                                                                                                               \\\hline
        Azioni      &
        TEST1:
        \begin{enumerate}[nosep, topsep=0pt]
            \item richiedere la disdetta di un piano di abbonamento, fornendo \emph{piano\_abb\_v};
            \item verificare che il sistema disattivi il rinnovo automatico di \emph{piano\_abb\_v};
            \item verificare che il sistema ritorni la pagina mostrando "Cambiamento avvenuto con successo" oppure "Errore Generico";
        \end{enumerate}
        \\\hline
    \end{tabular}
\end{table}

\begin{table}[hb]
    \centering
    \begin{tabular}{ |p{2cm}|p{10cm}|  }
        \hline
        ID          & T\_15                                                                                                                                   \\\hline
        Caso d'uso  & UC\_CambiaAbbonamento                                                                                                                   \\\hline
        Obbiettivo  & Testare il corretto funzionamento del cambio di piano di abbonamento associato ad un utente                                             \\\hline
        Prerequsiti & L'esecuzione del test è effettuata con privilegi di Utente e \emph{piano\_abb\_v} è un piano di abbonamento attivo dell'utente corrente \\\hline
        Tipologia   & White box                                                                                                                               \\\hline
        Azioni      &
        TEST1:
        \begin{enumerate}[nosep, topsep=0pt]
            \item richiedere il cambiamento di un piano di abbonamento, fornendo \emph{piano\_abb\_v};
            \item verificare che il sistema ritorni tutti i piani di abbonamento sottoscrivibili \emph{L} e che \emph{piano\_abb\_v} non sia contenuto in \emph{L};
            \item fornire al sistema un piano di abbonamento \emph{P2} con prezzo \emph{PRICE2} $\geq$ \emph{PRICE}, dove \emph{PRICE} è il prezzo di \emph{piano\_abb\_v};
            \item verificare che il sistema calcoli l'importo di pagamento \emph{R = PRICE2 - PRICE $\geq$ 0};
            \item fornire le coordinate bancarie \emph{CB} per procedere al pagamento;
            \item verificare che il sistema processi correttamente il pagamento e che l'importo \emph{R} venga correttamente accreditato;
            \item verificare che il sistema registri il pagamento effettuato e che \emph{piano\_abb\_v} venga sostituito col piano di abbonamento \emph{P2};
            \item verificare che il sistema mostri la pagina di cambiamento effettuato oppure di errore generico;
        \end{enumerate}
        \vspace{0.5cm} TEST2:
        \begin{enumerate}[nosep, topsep=0pt]
            \item richiedere il cambiamento di un piano di abbonamento, fornendo \emph{piano\_abb\_v};
            \item verificare che il sistema ritorni tutti i piani di abbonamento sottoscrivibili \emph{L} e che \emph{piano\_abb\_v} non sia contenuto in \emph{L};
            \item fornire al sistema un piano di abbonamento \emph{P2} con prezzo \emph{PRICE2} $<$ \emph{PRICE}, dove \emph{PRICE} è il prezzo di \emph{piano\_abb\_v};
            \item verificare che il sistema calcoli l'importo di pagamento \emph{R = 0};
            \item fornire le coordinate bancarie \emph{CB} per procedere al pagamento;
            \item verificare che il sistema processi correttamente il pagamento e che l'importo \emph{R} venga accreditato;
            \item verificare che il sistema registri il pagamento effettuato e che \emph{piano\_abb\_v} venga sostituito col piano di abbonamento \emph{P2};
            \item verificare che il sistema mostri la pagina di cambiamento effettuato oppure di errore generico;
        \end{enumerate}
        \\\hline
    \end{tabular}
\end{table}

\begin{table}[hb]
    \centering
    \begin{tabular}{ |p{2cm}|p{10cm}|  }
        \hline
        ID          & T\_16                                                                                   \\\hline
        Caso d'uso  & UC\_GestisciScadenzeAbbonamenti                                                         \\\hline
        Obbiettivo  & Testare il corretto funzionamento della gestione di un piano di abbonamento in scadenza \\\hline
        Prerequsiti & L'esecuzione del test è effettuata con privilegi di Amministratore                      \\\hline
        Tipologia   & White box                                                                               \\\hline
        Azioni      &
        TEST1:
        \begin{enumerate}[nosep, topsep=0pt]
            \item verificare che il sistema consideri le sole sottoscrizioni in scadenza nella giornata corrente;
            \item per ogni sottoscrizione \emph{S}, verificare che:
                  \begin{itemize}
                      \item se \emph{S} è ancora sottoscrivibile ed è attivo il rinnovo automatico, allora il sistema processi il pagamento per il rinnovo di \emph{S}. Verificare inoltre che \emph{S} sia rinnovata sse il pagamento va a buon fine, altrimenti verificare che \emph{S} sia rimossa dai piani di abbonamento dell'utente che lo deteneva;
                      \item se \emph{S} non è più sottoscrivibile oppure il rinnovo automatico è disattivo verificare che \emph{S} sia rimossa dai piani di abbonamento dell'utente che lo deteneva;
                  \end{itemize}
        \end{enumerate}

        \\\hline
    \end{tabular}
\end{table}

\begin{table}[hb]
    \centering
    \begin{tabular}{ |p{2cm}|p{10cm}|  }
        \hline
        ID          & T\_17                                                                                                         \\\hline
        Caso d'uso  & UC\_SospendiAccount                                                                                           \\\hline
        Obbiettivo  & Testare il corretto funzionamento della gestione di un piano di abbonamento in scadenza                       \\\hline
        Prerequsiti & L'esecuzione del test è effettuata con privilegi di Amministratore e  \emph{utente\_p\_v} è un account attivo \\\hline
        Tipologia   & White box                                                                                                     \\\hline
        Azioni      &
        TEST1:
        \begin{enumerate}[nosep, topsep=0pt]
            \item fornire la durata \emph{D} per il quale si vuole sospendere l'account \emph{utente\_p\_v};
            \item verificare che il sistema sospenda \emph{utente\_p\_v} per la durata richiesta;
        \end{enumerate}

        \\\hline
    \end{tabular}
\end{table}

\begin{table}[hb]
    \centering
    \begin{tabular}{ |p{2cm}|p{10cm}|  }
        \hline
        ID          & T\_18                                                                                                          \\\hline
        Caso d'uso  & UC\_RiattivaAccount                                                                                            \\\hline
        Obbiettivo  & Testare il corretto funzionamento della funzionalità riattiva abbonamento                                      \\\hline
        Prerequsiti & L'esecuzione del test è effettuata con privilegi di Amministratore e \emph{utente\_p\_v} è un account bloccato \\\hline
        Tipologia   & White box                                                                                                      \\\hline
        Azioni      &
        TEST1:
        \begin{enumerate}[nosep, topsep=0pt]
            \item richiedere la riattivazione dell'account \emph{utente\_p\_v};
            \item verificare che il sistema riattivi l'account richiesto;
        \end{enumerate}
        \\\hline
    \end{tabular}
\end{table}

\begin{table}[hb]
    \centering
    \begin{tabular}{ |p{2cm}|p{10cm}|  }
        \hline
        ID          & T\_19                                                                                                          \\\hline
        Caso d'uso  & UC\_RiattivaAccountAutomatico                                                                                  \\\hline
        Obbiettivo  & Testare il corretto funzionamento della funzionalità riattiva abbonamento automatico                           \\\hline
        Prerequsiti & L'esecuzione del test è effettuata con privilegi di Amministratore e \emph{utente\_p\_v} è un account bloccato \\\hline
        Tipologia   & White box                                                                                                      \\\hline
        Azioni      &
        TEST1:
        \begin{enumerate}[nosep, topsep=0pt]
            \item verificare che il sistema consideri la lista degli account bloccati \emph{B}, previsti da sbloccare nella giornata corrente;
            \item verificare che il sistema sblocchi con successo tutti gli account in \emph{B};
            \item verificare che il sistema ritorni l'esito della procedura che può essere errore generico oppure account riattivati con successo;
        \end{enumerate}
        \\\hline
    \end{tabular}
\end{table}

\begin{table}[hb]
    \centering
    \begin{tabular}{ |p{2cm}|p{10cm}|  }
        \hline
        ID          & T\_20                                                                      \\\hline
        Caso d'uso  & UC\_EffettuaRegistrazione                                                  \\\hline
        Obbiettivo  & Testare il corretto funzionamento della registrazione di un account utente \\\hline
        Prerequsiti & L'esecuzione del test è effettuata con privilegi di Utente                 \\\hline
        Tipologia   & White box                                                                  \\\hline
        Azioni      &
        TEST1:
        \begin{enumerate}[nosep, topsep=0pt]
            \item richiedere la registrazione di un nuovo account utente, fornendo: \emph{nome\_np\_v, cognome\_np\_v, email\_np\_v, password\_np\_v};
            \item verificare che il sistema crei un nuovo account utente con i dati forniti;
            \item verificare che il sistema ritorni una pagina mostrando "utente registrato con successo";
        \end{enumerate}
        \vspace{0.5cm} TEST2:
        \begin{enumerate}[nosep, topsep=0pt]
            \item richiedere la registrazione di un nuovo account utente, fornendo: \emph{nome\_np\_v, cognome\_np\_v, email\_p\_nv, password\_np\_v};
            \item verificare che il sistema non crei un account utente con i dati forniti e che ritorni una pagina di errore;
        \end{enumerate}
        \\\hline
    \end{tabular}
\end{table}

\begin{table}[hb]
    \centering
    \begin{tabular}{ |p{2cm}|p{10cm}|  }
        \hline
        ID          & T\_21                                                                           \\\hline
        Caso d'uso  & UC\_ModificaProfilo                                                             \\\hline
        Obbiettivo  & Testare il corretto funzionamento della funzionalità di modifica profilo utente \\\hline
        Prerequsiti & L'esecuzione del test è effettuata con privilegi di Utente                      \\\hline
        Tipologia   & White box                                                                       \\\hline
        Azioni      &
        TEST1:
        \begin{enumerate}[nosep, topsep=0pt]
            \item richiedere la modifica di un profilo utente, fornendo \emph{utente\_p\_v};
            \item verificare che il sistema ritorni la lista \emph{L} dei campi modificabili per l'account richiesto;
            \item fornire la coppia \emph{K,V} dove \emph{K} è un sottoinsieme dei campi modificabili contenuti in \emph{L}, e \emph{V} sono i valori assegnati;
            \item verificare che il sistema applichi le modifiche all'account \emph{utente\_p\_v};
            \item verificare che il sistema mostri una pagina con un messaggio indicante l'avvenuta conferma dei campi inseriti , oppure un errore generico;
        \end{enumerate}
        \\\hline
    \end{tabular}
\end{table}

\begin{table}[hb]
    \centering
    \begin{tabular}{ |p{2cm}|p{10cm}|  }
        \hline
        ID          & T\_22                                                         \\\hline
        Caso d'uso  & UC\_EffettuaLogin                                             \\\hline
        Obbiettivo  & Testare il corretto funzionamento della funzionalità di login \\\hline
        Prerequsiti & L'esecuzione del test è effettuata con privilegi di Utente    \\\hline
        Tipologia   & White box                                                     \\\hline
        Azioni      &
        TEST1:
        \begin{enumerate}[nosep, topsep=0pt]
            \item richiedere di effettuare l'accesso, fornendo \emph{email\_np\_v, password\_np\_v};
            \item verificare che il sistema validi con successo i dati inseriti e che torni un errore sse l'account è bloccato, altrimenti mostra "login effettuato con successo";
        \end{enumerate}
        \vspace{0.5cm} TEST2:
        \begin{enumerate}[nosep, topsep=0pt]
            \item richiedere di effettuare l'accesso, fornendo \emph{email\_p\_nv, password\_p\_nv};
            \item verificare che il sistema ritorni un errore in fase di validazione dei dati, l'accesso al sistema non è consentito;
        \end{enumerate}
        \\\hline
    \end{tabular}
\end{table}

\begin{table}[hb]
    \centering
    \begin{tabular}{ |p{2cm}|p{10cm}|  }
        \hline
        ID          & T\_23                                                                                                                         \\\hline
        Caso d'uso  & UC\_EffettuaLogout                                                                                                            \\\hline
        Obbiettivo  & Testare il corretto funzionamento della funzionalità di logout                                                                \\\hline
        Prerequsiti & L'esecuzione del test è effettuata con privilegi di Utente e \emph{utente\_p\_v} è un account attualmente connesso al sistema \\\hline
        Tipologia   & White box                                                                                                                     \\\hline
        Azioni      &
        TEST1:
        \begin{enumerate}[nosep, topsep=0pt]
            \item richiedere di effettuare il logout, fornendo l'account \emph{utente\_p\_v};
            \item verificare che l'account sia correttamente disconnesso dal sistema e che venga mostrata una pagina di disconnessione effettuata con successo;
        \end{enumerate}
        \\\hline
    \end{tabular}
\end{table}

\begin{table}[hb]
    \centering
    \begin{tabular}{ |p{2cm}|p{10cm}|  }
        \hline
        ID          & T\_24                                                                      \\\hline
        Caso d'uso  & UC\_OttieniCronologia                                                      \\\hline
        Obbiettivo  & Testare il corretto funzionamento della funzionalità di ottieni cronologia \\\hline
        Prerequsiti & L'esecuzione del test è effettuata con privilegi di Utente                 \\\hline
        Tipologia   & White box                                                                  \\\hline
        Azioni      &
        TEST1:
        \begin{enumerate}[nosep, topsep=0pt]
            \item richiedere di ottenere la cronologia dell'account \emph{utente\_p\_v};
            \item verificare che il sistema consideri la lista \emph{L} dei soli prodotti visualizzati dall'utente indicato, ritornando infine una pagina per mostrare l'elenco dei prodotti in \emph{L}
        \end{enumerate}
        \\\hline
    \end{tabular}
\end{table}

\begin{table}[hb]
    \centering
    \begin{tabular}{ |p{2cm}|p{10cm}|  }
        \hline

        ID          & T\_25                                                                         \\\hline
        Caso d'uso  & UC\_GestisciProdotti                                                          \\\hline
        Obbiettivo  & Testare il corretto funzionamento della schermata per visualizzare i prodotti \\\hline
        Prerequsiti & L'esecuzione del test è effettuata con privilegi di Utente                    \\\hline
        Tipologia   & White box                                                                     \\\hline
        Azioni      &
        TEST1:
        \begin{enumerate}[nosep, topsep=0pt]
            \item richiedere di ottenere la lista dei prodotti creati, fornendo il token di autenticazione \emph{auth\_t\_v};
            \item verificare che il sistema validi la richiesta di visualizzazione prodotti;
            \item verificare che il sistema ritorni la lista \emph{L} dei prodotti relativi all'account indicati e che popoli la pagina di risultati con elementi di \emph{L};
        \end{enumerate}
        \vspace{0.5cm} TEST2:
        \begin{enumerate}[nosep, topsep=0pt]
            \item richiedere di ottenere la lista dei prodotti creati, fornendo il token di autenticazione \emph{auth\_t\_np\_nv};
            \item verificare che il sistema neghi la richiesta di visualizzazione prodotti e che mostri la relativa pagina di errore;
        \end{enumerate}
        \\\hline
    \end{tabular}
\end{table}

\begin{table}[hb]
    \centering
    \begin{tabular}{ |p{2cm}|p{10cm}|  }
        \hline
        ID          & T\_26                                                                                    \\\hline
        Caso d'uso  & UC\_CreaProdotto                                                                         \\\hline
        Obbiettivo  & Testare il corretto funzionamento della creazione di nuovi prodotti da parte dell'utente \\\hline
        Prerequsiti & L'esecuzione del test è effettuata con privilegi di Utente                               \\\hline
        Tipologia   & White box                                                                                \\\hline
        Azioni      &
        TEST1:
        \begin{enumerate}[nosep, topsep=0pt]
            \item richiedere la creazione di un nuovo prodotto di tipo video, fornendo \emph{nome\_np\_v} e il token di autenticazione \emph{auth\_t\_v};
            \item verificare che il token di autenticazione sia correttamente validato dal sistema;
            \item verificare che il sistema crei il prodotto e che venga conseguentemente mostrata una pagina con l'esito dell'operazione;
        \end{enumerate}
        \vspace{0.5cm} TEST2:
        \begin{enumerate}[nosep, topsep=0pt]
            \item richiedere la creazione di un nuovo prodotto di tipo video, fornendo \emph{nome\_nv} e il token di autenticazione \emph{auth\_t\_np\_nv};
            \item verificare che il sistema ritorni un errore in fase di validazione oppure nella creazione di un contenuto già esistente;
            \item verificare che il sistema mostri una pagina esplicativa dell'errore avvenuto;
        \end{enumerate}
        \vspace{0.5cm}
        TEST3:
        \begin{enumerate}[nosep, topsep=0pt]
            \item richiedere la creazione di un nuovo prodotto di tipo audio, fornendo \emph{nome\_np\_v} e il token di autenticazione \emph{auth\_t\_v};
            \item verificare che il token di autenticazione sia correttamente validato dal sistema;
            \item verificare che il sistema crei il prodotto e che venga conseguentemente mostrata una pagina con l'esito dell'operazione;
        \end{enumerate}
        \vspace{0.5cm} TEST4:
        \begin{enumerate}[nosep, topsep=0pt]
            \item richiedere la creazione di un nuovo prodotto di tipo audio, fornendo \emph{nome\_nv} e il token di autenticazione \emph{auth\_t\_np\_nv};
            \item verificare che il sistema ritorni un errore in fase di validazione oppure nella creazione di un contenuto già esistente;
            \item verificare che il sistema mostri una pagina esplicativa dell'errore avvenuto;
        \end{enumerate}
        \\\hline
    \end{tabular}
\end{table}

\begin{table}[hb]
    \centering
    \begin{tabular}{ |p{2cm}|p{10cm}|  }
        \hline
        ID          & T\_27                                                                                                          \\\hline
        Caso d'uso  & UC\_ModificaInformazioniDiBase                                                                                 \\\hline
        Obbiettivo  & Testare il corretto funzionamento della funzionalità per la modifica della informazioni di base di un prodotto \\\hline
        Prerequsiti & L'esecuzione del test è effettuata con privilegi di Utente                                                     \\\hline
        Tipologia   & White box                                                                                                      \\\hline
        Azioni      &
        TEST1:
        \begin{enumerate}[nosep, topsep=0pt]
            \item richiedere di modificare le informazioni di base di un prodotto \emph{contenuto\_p\_v}, fornendo \emph{descrizione\_np\_v} e \emph{genere\_p\_v};
            \item verificare che il sistema validi la richiesta e che le nuove informazioni vengano memorizzate;
            \item verificare che il sistema presenti un questionario a domande chiuse per la valutazione dell'età minima per visionarlo/riprodurlo;
            \item fornire le risposte al questionario;
            \item verificare che il sistema calcoli e aggiorni l'età minima per visionare/riprodurre il prodotto;
            \item verificare che il sistema mostri la pagina riepilogativa con le modifiche apportate;
        \end{enumerate}
        \vspace{0.5cm} TEST2:
        \begin{enumerate}[nosep, topsep=0pt]
            \item richiedere di modificare le informazioni di base di un prodotto \emph{contenuto\_p\_v}, fornendo \emph{descrizione\_np\_nv} e \emph{genere\_p\_nv};
            \item verificare che il sistema mostri a schermo un errore di validazione delle informazioni fornite;
        \end{enumerate}
        \\\hline
    \end{tabular}
\end{table}

\begin{table}[hb]
    \centering
    \begin{tabular}{ |p{2cm}|p{10cm}|  }
        \hline
        ID          & T\_28                                                                          \\\hline
        Caso d'uso  & UC\_CaricaFile                                                                 \\\hline
        Obbiettivo  & Testare il corretto funzionamento della funzionalità di caricamento file       \\\hline
        Prerequsiti & L'esecuzione del test è effettuata con privilegi di Amministratore o di Utente \\\hline
        Tipologia   & White box                                                                      \\\hline
        Azioni      &
        TEST1:
        \begin{enumerate}[nosep, topsep=0pt]
            \item richiedere il caricamento di un file video, fornendo: \emph{auth\_t\_p\_v}, \emph{file\_video\_np\_v}, \emph{subs\_np\_v}, \emph{audios\_np\_v};
            \item verificare che il sistema validi correttamente la richiesta;
            \item verificare che il sistema elabori elabori tutte le informazioni fornite in un file MKV \emph{F};
            \item verificare che nel database sia contenuto \emph{F};
            \item verificare che le CDN siano aggiornate rispetto a \emph{F};
            \item verificare che il sistema mostri una pagina di conferma del caricamento effettuato;
        \end{enumerate}
        \vspace{0.5cm} TEST2:
        \begin{enumerate}[nosep, topsep=0pt]
            \item richiedere il caricamento di un file audio, fornendo: \emph{auth\_t\_p\_v}, \emph{file\_audio\_np\_v}, \emph{lyrics\_np\_v}, \emph{video\_np\_v};
            \item verificare che il sistema validi correttamente la richiesta;
            \item verificare che il sistema elabori elabori tutte le informazioni fornite in una tripla di file MKV: \emph{F1}, \emph{F2} ed \emph{F3}.;
            \item verificare che \emph{F1} contenga i lyrics e la traccia musicale, che \emph{F2} contenga la traccia video e quella musicale e che \emph{F3} contenga la traccia musicale;
            \item verificare che nel database siano contenuti \emph{F1}, \emph{F2} ed \emph{F3};
            \item verificare che le CDN siano aggiornate rispetto a \emph{F1}, \emph{F2} ed \emph{F3};
            \item verificare che il sistema mostri una pagina di conferma dei caricamenti effettuati;
        \end{enumerate}
        \\\hline
    \end{tabular}
\end{table}

\begin{table}[hb]
    \centering
    \begin{tabular}{ |p{2cm}|p{10cm}|  }
        \hline
        ID          & T\_29                                                                                      \\\hline
        Caso d'uso  & UC\_CambiaVisibilitàProdotto                                                               \\\hline
        Obbiettivo  & Testare il corretto funzionamento della funzionalità cambiare la visibilità di un prodotto \\\hline
        Prerequsiti & L'esecuzione del test è effettuata con privilegi di Utente                                 \\\hline
        Tipologia   & White box                                                                                  \\\hline
        Azioni      &
        TEST1:
        \begin{enumerate}[nosep, topsep=0pt]
            \item richiedere di modificare la visibilità di un prodotto, fornendo \emph{contenuto\_p\_v} e \emph{visibilita\_p\_v};
            \item verificare che nel database lo stato del prodotto \emph{contenuto\_p\_v} sia impostato a \emph{visibilita\_p\_v};
            \item verificare che il sistema ritorni un messaggio di modifica effettuata con successo oppure di errore generico;
        \end{enumerate}
        \\\hline
    \end{tabular}
\end{table}

\begin{table}[hb]
    \centering
    \begin{tabular}{ |p{2cm}|p{10cm}|  }
        \hline
        ID          & T\_30                                                               \\\hline
        Caso d'uso  & UC\_RiproduciProdotto                                               \\\hline
        Obbiettivo  & Testare il corretto funzionamento della riproduzione di un prodotto \\\hline
        Prerequsiti & L'esecuzione del test è effettuata con privilegi di Utente          \\\hline
        Tipologia   & White box                                                           \\\hline
        Azioni      &
        TEST1:
        \begin{enumerate}[nosep, topsep=0pt]
            \item richiedere di iniziare la riproduzione di un prodotto, fornendo \emph{auth\_t\_p\_nv} e \emph{contenuto\_p\_v};
            \item verificare che la connessione ad internet sia disponibile;
            \item verificare che il sistema rifiuti la richiesta di riproduzione di \emph{contenuto\_p\_v}, di tipologia che non è inclusa nel piano dell'utente associato a \emph{auth\_t\_p\_nv};
            \item verificare che il sistema mostri una pagina con l'errore "non si dispone del servizio necessario a riprodurre il contenuto";
        \end{enumerate}
        \vspace{0.5cm} TEST2:
        \begin{enumerate}[nosep, topsep=0pt]
            \item richiedere di iniziare la riproduzione di un prodotto, fornendo \emph{auth\_t\_p\_v} e \emph{contenuto\_p\_v};
            \item verificare che il sistema valuti nel flag \emph{INTERNET} se la connessione ad internet è presente;
            \item verificare che il sistema validi la richiesta di riproduzione di \emph{contenuto\_p\_v}, di tipologia che è inclusa nel piano dell'utente associato a \emph{auth\_t\_p\_v};
            \item verificare che il sistema aggiunga la riproduzione generata al database e valuti nel flag \emph{ADS} se inserire spot pubblicitari durante la riproduzione;
            \item verificare che il sistema, dopo aver valutato nel flag \emph{DOWNLOADED} se il prodotto è stato o meno scaricato, agisca come segua:
                  \begin{itemize}
                      \item riproduca il prodotto offline sse il flag \emph{DOWNLOADED} è vero, inserendo spot pubblicitari sse \emph{ADS} è vero;
                      \item mostri una pagina con errore "è necessaria una connessione ad internet" sse sia \emph{DOWNLOADED} che \emph{INTERNET} sono falsi;
                      \item riproduca il prodotto in streaming sse \emph{DOWNLOADED} è falso ma \emph{INTERNET} è vero, inserendo spot pubblicitari sse \emph{ADS} è vero;
                  \end{itemize}
        \end{enumerate}
        \\\hline
    \end{tabular}
\end{table}

\begin{table}[hb]
    \centering
    \begin{tabular}{ |p{2cm}|p{10cm}|  }
        \hline
        ID          & T\_31                                                                   \\\hline
        Caso d'uso  & UC\_StreamingVideo                                                      \\\hline
        Obbiettivo  & Testare il corretto funzionamento della funzionalità di streaming video \\\hline
        Prerequsiti & L'esecuzione del test è effettuata con privilegi di Utente              \\\hline
        Tipologia   & White box                                                               \\\hline
        Azioni      &
        TEST1:
        \begin{enumerate}[nosep, topsep=0pt]
            \item richiedere di visualizzare in streaming un prodotto video, fornendo \emph{contenuto\_p\_v};
            \item verificare che il sistema reperisca correttamente la \emph{URL} associata al contenuto video indicato, l'\emph{IP} dell'utente richiedente e la durata \emph{DUR};
            \item verificare che il sistema generi una policy di visualizzazione \emph{P} per l'utente richiedente, a partire da \emph{IP} e \emph{DUR};
            \item verificare che il sistema generi una \emph{SIGNED\_URL} a partire da \emph{URL} e \emph{P} che garantisca all'utente di visionare il video;
            \item verificare che il sistema player inizi la riproduzione del contenuto richiesto utilizzando \emph{SIGNED\_URL};
        \end{enumerate}
        \\\hline
    \end{tabular}
\end{table}

\begin{table}[hb]
    \centering
    \begin{tabular}{ |p{2cm}|p{10cm}|  }
        \hline
        ID          & T\_32                                                                    \\\hline
        Caso d'uso  & UC\_StreamingMusica                                                      \\\hline
        Obbiettivo  & Testare il corretto funzionamento della funzionalità di streaming musica \\\hline
        Prerequsiti & L'esecuzione del test è effettuata con privilegi di Utente               \\\hline
        Tipologia   & White box                                                                \\\hline
        Azioni      &
        TEST1:
        \begin{enumerate}[nosep, topsep=0pt]
            \item richiedere di visualizzare in streaming un prodotto video, fornendo \emph{contenuto\_p\_v} e il tipo di riproduzione preferito \emph{audio\_type\_p\_v};
            \item verificare che il sistema reperisca correttamente la \emph{URL} associata al contenuto video indicato e a \emph{audio\_type\_p\_v}, l'\emph{IP} dell'utente richiedente e la durata \emph{DUR};
            \item verificare che il sistema generi una policy di visualizzazione \emph{P} per l'utente richiedente, a partire da \emph{IP} e \emph{DUR};
            \item verificare che il sistema generi una \emph{SIGNED\_URL} a partire da \emph{URL} e \emph{P} che garantisca all'utente di visionare il video;
            \item verificare che il sistema player inizi la riproduzione del contenuto richiesto utilizzando \emph{SIGNED\_URL};
        \end{enumerate}
        \\\hline
    \end{tabular}
\end{table}

\begin{table}[hb]
    \centering
    \begin{tabular}{ |p{2cm}|p{10cm}|  }
        \hline
        ID          & T\_33                                                                                                                                   \\\hline
        Caso d'uso  & UC\_PausaPlayer                                                                                                                         \\\hline
        Obbiettivo  & Testare il corretto funzionamento della funzionalità di pausa della riproduzione                                                        \\\hline
        Prerequsiti & L'esecuzione del test è effettuata con privilegi di Utente e \emph{contenuto\_p\_v} attualmente in riproduzione sul player multimediale \\\hline
        Tipologia   & White box                                                                                                                               \\\hline
        Azioni      &
        TEST1:
        \begin{enumerate}[nosep, topsep=0pt]
            \item richiedere la messa in pausa della riproduzione di \emph{contenuto\_p\_v};
            \item verificare che il sistema blocchi la riproduzione del contenuto sul player multimediale, mostrando il pulsante di play al posto di quello di pausa;
            \item verificare che la CDN continui ad inviare pacchetti sino alla saturazione del buffer di riproduzione;
        \end{enumerate}
        \\\hline
    \end{tabular}
\end{table}

\begin{table}[hb]
    \centering
    \begin{tabular}{ |p{2cm}|p{10cm}|  }
        \hline
        ID          & T\_34                                                                                                                               \\\hline
        Caso d'uso  & UC\_SpostaPuntoRiproduzionePlayer                                                                                                   \\\hline
        Obbiettivo  & Testare il corretto funzionamento della funzionalità di spostamento del punto di riproduzione del player                            \\\hline
        Prerequsiti & L'esecuzione del test è effettuata con privilegi di Utente e \emph{contenuto\_p\_v} attualmente in caricato nel player multimediale \\\hline
        Tipologia   & White box                                                                                                                           \\\hline
        Azioni      &
        TEST1:
        \begin{enumerate}[nosep, topsep=0pt]
            \item richiedere lo spostamento del punto di riproduzione di \emph{contenuto\_p\_v} nel player multimediale;
            \item verificare che il sistema comunichi alla CDN il nuovo punto di riproduzione affinchè quest'ultimo possa inviare i pacchetti desiderati;
            \item verificare che il player multimediale riprenda l'esecuzione non appena riceve un numero sufficiente di pacchetti, oppure che attenda la ricezione di ulteriori pacchetti;
        \end{enumerate}
        \\\hline
    \end{tabular}
\end{table}

\begin{table}[hb]
    \centering
    \begin{tabular}{ |p{2cm}|p{10cm}|  }
        \hline
        ID          & T\_35                                                                                        \\\hline
        Caso d'uso  & UC\_DownloadProdotto                                                                         \\\hline
        Obbiettivo  & Testare il corretto funzionamento della funzionalità di download di un prodotto multimediale \\\hline
        Prerequsiti & L'esecuzione del test è effettuata con privilegi di Utente                                   \\\hline
        Tipologia   & White box                                                                                    \\\hline
        Azioni      &
        TEST1:
        \begin{enumerate}[nosep, topsep=0pt]
            \item richiedere il dowload di un prodotto multimediale \emph{contenuto\_p\_v}, fornendo \emph{auth\_t\_p\_v};
            \item verificare che il sistema validi correttamente la richiesta di download del prodotto indicato;
            \item verificare che il sistema contatti un nodo per la ricezione del file richiesto;
            \item verificare che il sistema salvi sul dispositivo il prodotto scaricato, se questo può essere scaricato;
            \item verificare che il sistema mostri un messaggio per indicare lo scaricamento del prodotto, oppure un errore esplicativo di problemi avvenuti nel download.
        \end{enumerate}
        \vspace{0.5cm} TEST2:
        \begin{enumerate}[nosep, topsep=0pt]
            \item richiedere il dowload di un prodotto multimediale \emph{contenuto\_p\_nv}, fornendo \emph{auth\_t\_p\_v};
            \item verificare che il sistema rifiuti la richiesta di download del prodotto indicato;
            \item verificare che il sistema mostri un errore di "prodotto non disponibile per il download";
        \end{enumerate}
        \\\hline
    \end{tabular}
\end{table}

\begin{table}[hb]
    \centering
    \begin{tabular}{ |p{2cm}|p{10cm}|  }
        \hline
        ID          & T\_36                                                                              \\\hline
        Caso d'uso  & UC\_VisualizzaPubblicità                                                           \\\hline
        Obbiettivo  & Testare il corretto funzionamento della visualizzazione degli annunci pubblicitari \\\hline
        Prerequsiti & L'esecuzione del test è effettuata con privilegi di Utente                         \\\hline
        Tipologia   & White box                                                                          \\\hline
        Azioni      &
        TEST1:
        \begin{enumerate}[nosep, topsep=0pt]
            \item richiedere la visualizzazione di contenuti pubblicitari, fornendo \emph{auth\_t\_p\_v};
            \item verificare che il sistema validi correttamente la richiesta di visualizzazione;
            \item verificare che il sistema calcoli in \emph{COUNT} il numero di riproduzioni di contenuti multimediali necessarie alla riproduzione di uno spot pubblicitario;
            \item verificare che il sistema ripristini il numero di riproduzioni di contenuti multimediali necessarie alla riproduzione di uno spot pubblicitario al valore di default sse \emph{COUNT = 0};
            \item verificare che il sistema riproduca uno spot pubblicitario qualsiasi sse \emph{COUNT = 0};
        \end{enumerate}
        \\\hline
    \end{tabular}
\end{table}

\begin{table}[hb]
    \centering
    \begin{tabular}{ |p{2cm}|p{10cm}|  }
        \hline
        ID          & T\_37                                                                                                                                                                  \\\hline
        Caso d'uso  & UC\_RiproduciAudioInBackground                                                                                                                                         \\\hline
        Obbiettivo  & Testare il corretto funzionamento della riproduzione di audio in background                                                                                            \\\hline
        Prerequsiti & L'esecuzione del test è effettuata con privilegi di Utente, è in corso la riproduzione del prodotto audio \emph{contenuto\_p\_v} e il client viene messo in background \\\hline
        Tipologia   & White box                                                                                                                                                              \\\hline
        Azioni      &
        TEST1:
        \begin{enumerate}[nosep, topsep=0pt]
            \item verificare che il sistema sposti la riproduzione del contenuto multimediale \emph{contenuto\_p\_v} in modalità riproduzione in background sse l'utente richiedente ha associato il servizio di riproduzione di audio in background, altrimenti nel blocca la riproduzione;
        \end{enumerate}
        \\\hline
    \end{tabular}
\end{table}

\begin{table}[hb]
    \centering
    \begin{tabular}{ |p{2cm}|p{10cm}|  }
        \hline
        ID          & T\_38                                                                              \\\hline
        Caso d'uso  & UC\_SegnalaProdotto                                                           \\\hline
        Obbiettivo  & Testare il corretto funzionamento della funzionalità di segnalazione di un prodotto \\\hline
        Prerequsiti & L'esecuzione del test è effettuata con privilegi di Utente                         \\\hline
        Tipologia   & White box                                                                          \\\hline
        Azioni      &
        TEST1:
        \begin{enumerate}[nosep, topsep=0pt]
            \item richiedere l'apertura di una segnalazione \emph{segnalazione\_np\_v} relativa ad un prodotto \emph{contenuto\_p\_v};
            \item verificare che nel database sia presente \emph{segnalazione\_np\_v};
            \item verificare che il sistema mostri un messaggio di errore durante l'inserimento della segnalazione oppure di segnalazione inserita;
        \end{enumerate}
        \\\hline
    \end{tabular}
\end{table}

\begin{table}[hb]
    \centering
    \begin{tabular}{ |p{2cm}|p{10cm}|  }
        \hline
        ID          & T\_39                                                                              \\\hline
        Caso d'uso  & UC\_GestisciSegnalazioni                                                           \\\hline
        Obbiettivo  & Testare il corretto funzionamento della funzionalità di gestione delle segnalazioni \\\hline
        Prerequsiti & L'esecuzione del test è effettuata con privilegi di Amministratore                         \\\hline
        Tipologia   & White box                                                                          \\\hline
        Azioni      &
        TEST1:
        \begin{enumerate}[nosep, topsep=0pt]
            \item richiedere la visualizzazione delle segnalazioni presenti nel database;
            \item fornisce l'indicazione sulla segnalazione \emph{segnalazione\_p\_v} scelta e su cui effettuare un'azione \emph{A}, a scelta tra: chiudi segnalazione, riapri segnalazione, sospendere account con segnalazione aperta oppure visualizza prodotto segnalato;
            \item verificare che il sistema mostri la pagina relativa alla richiesta descritta dalla coppia \emph{segnalazione\_p\_v}, \emph{A};
        \end{enumerate}
        \\\hline
    \end{tabular}
\end{table}

\begin{table}[hb]
    \centering
    \begin{tabular}{ |p{2cm}|p{10cm}|  }
        \hline
        ID          & T\_40                                                                              \\\hline
        Caso d'uso  & UC\_OttieniSegnalazioni                                                           \\\hline
        Obbiettivo  & Testare il corretto funzionamento della funzionalità ottieni segnalazioni \\\hline
        Prerequsiti & L'esecuzione del test è effettuata con privilegi di Amministratore                         \\\hline
        Tipologia   & White box                                                                          \\\hline
        Azioni      &
        TEST1:
        \begin{enumerate}[nosep, topsep=0pt]
            \item richiedere la visualizzazione delle segnalazioni presenti nel database relative ad un prodotto \emph{contenuto\_p\_v};
            \item verificare che il sistema mostri tutte le segnalazioni relative al prodotto indicato;
        \end{enumerate}
        \\\hline
    \end{tabular}
\end{table}


\begin{table}[hb]
    \centering
    \begin{tabular}{ |p{2cm}|p{10cm}|  }
        \hline
        ID          & T\_41                                                                              \\\hline
        Caso d'uso  & UC\_ChiudiSegnalazione                                                           \\\hline
        Obbiettivo  & Testare il corretto funzionamento della funzionalità di chiusura delle segnalazioni \\\hline
        Prerequsiti & L'esecuzione del test è effettuata con privilegi di Amministratore e \emph{segnalazione\_p\_v} è una segnalazione aperta                         \\\hline
        Tipologia   & White box                                                                          \\\hline
        Azioni      &
        TEST1:
        \begin{enumerate}[nosep, topsep=0pt]
            \item richiedere la chiusura di una segnalazione \emph{segnalazione\_p\_v}, fornendo una descrizione \emph{descrizione\_np\_v};
            \item verificare che il database registri l'avvenuta chiusura della segnalazione \emph{segnalazione\_p\_v};
            \item verificare che il sistema mostri un messaggio di avvenuta chiusura della segnalazione oppure di errore generico;
        \end{enumerate}
        \\\hline
    \end{tabular}
\end{table}

\begin{table}[hb]
    \centering
    \begin{tabular}{ |p{2cm}|p{10cm}|  }
        \hline
        ID          & T\_42                                                                              \\\hline
        Caso d'uso  & UC\_RiapriSegnalazione                                                           \\\hline
        Obbiettivo  & Testare il corretto funzionamento della funzionalità di riapertura delle segnalazioni \\\hline
        Prerequsiti & L'esecuzione del test è effettuata con privilegi di Amministratore e \emph{segnalazione\_p\_v} è una segnalazione chiusa                         \\\hline
        Tipologia   & White box                                                                          \\\hline
        Azioni      &
        TEST1:
        \begin{enumerate}[nosep, topsep=0pt]
            \item richiedere la riapertura di una segnalazione \emph{segnalazione\_p\_v};
            \item verificare che il database registri l'avvenuta riapertura della segnalazione \emph{segnalazione\_p\_v};
            \item verificare che il sistema mostri un messaggio di avvenuta riapertura della segnalazione oppure di errore generico;
        \end{enumerate}
        \\\hline
    \end{tabular}
\end{table}

\begin{table}[hb]
    \centering
    \begin{tabular}{ |p{2cm}|p{10cm}|  }
        \hline
        ID          & T\_43                                                                              \\\hline
        Caso d'uso  & UC\_VisualizzaProdottoSegnalato                                                           \\\hline
        Obbiettivo  & Testare il corretto funzionamento della funzionalità di visualizzazione dei prodotti segnalati \\\hline
        Prerequsiti & L'esecuzione del test è effettuata con privilegi di Amministratore e \emph{segnalazione\_p\_v} è una segnalazione aperta                         \\\hline
        Tipologia   & White box                                                                          \\\hline
        Azioni      &
        TEST1:
        \begin{enumerate}[nosep, topsep=0pt]
            \item richiedere la visualizzazione del prodotto segnalato relativo alla segnalazione \emph{segnalazione\_p\_v};
            \item verificare che il sistema recuperi correttamente il prodotto \emph{P} a partire dalla segnalazione indicata;
            \item verificare che il prodotto venga visualizzato mediante player multimediale;
            \item verificare che il sistema, al termine della riproduzione, mostri un messaggio di termine visualizzazione;
        \end{enumerate}
        \\\hline
    \end{tabular}
\end{table}

\begin{table}[hb]
    \centering
    \begin{tabular}{ |p{2cm}|p{10cm}|  }
        \hline
        ID          & T\_44                                                                              \\\hline
        Caso d'uso  & UC\_RicercaContenuto                                                           \\\hline
        Obbiettivo  & Testare il corretto funzionamento della funzionalità di ricerca contenuti \\\hline
        Prerequsiti & L'esecuzione del test è effettuata con privilegi di Utente                         \\\hline
        Tipologia   & White box                                                                          \\\hline
        Azioni      &
        TEST1:
        \begin{enumerate}[nosep, topsep=0pt]
            \item richiedere la ricerca di contenuti che contengano la stringa \emph{titolo\_np\_v};
            \item verificare che il sistema validi correttamente l'input fornito;
            \item verificare che il sistema ritorni la lista \emph{L} di prodotti che corrispondono alla stringa indicata;
            \item verificare che il sistema mostri i contenuti presenti in \emph{L};
        \end{enumerate}
        \vspace{0.5cm} TEST2:
        \begin{enumerate}[nosep, topsep=0pt]
            \item richiedere la ricerca di contenuti che contengano la stringa \emph{titolo\_np\_v};
            \item verificare che il sistema validi correttamente l'input fornito;
            \item verificare che il sistema mostri una pagina priva di risultati;
        \end{enumerate}
        \vspace{0.5cm} TEST3:
        \begin{enumerate}[nosep, topsep=0pt]
            \item richiedere la ricerca di contenuti che contengano la stringa \emph{titolo\_np\_nv};
            \item verificare che il sistema non validi l'input fornito e mostri un messaggio di errore;
        \end{enumerate}
        \\\hline
    \end{tabular}
\end{table}

\begin{table}[hb]
    \centering
    \begin{tabular}{ |p{2cm}|p{10cm}|  }
        \hline
        ID          & T\_45                                                                              \\\hline
        Caso d'uso  & UC\_RicercaPopolari                                                           \\\hline
        Obbiettivo  & Testare il corretto funzionamento della funzionalità di ricerca contenuti popolari \\\hline
        Prerequsiti & L'esecuzione del test è effettuata con privilegi di Utente                         \\\hline
        Tipologia   & Black box                                                                          \\\hline
        Azioni      &
        TEST1:
        \begin{enumerate}[nosep, topsep=0pt]
            \item richiedere la ricerca dei contenuti popolari;
            \item verificare che il il sistema mostri la lista dei contenuti trovati;
        \end{enumerate}
        \\\hline
    \end{tabular}
\end{table}

\begin{table}[hb]
    \centering
    \begin{tabular}{ |p{2cm}|p{10cm}|  }
        \hline
        ID          & T\_46                                                                              \\\hline
        Caso d'uso  & UC\_SuggerisciContenuti                                                           \\\hline
        Obbiettivo  & Testare il corretto funzionamento della funzionalità di suggerisci contenuti \\\hline
        Prerequsiti & L'esecuzione del test è effettuata con privilegi di Utente                         \\\hline
        Tipologia   & White box                                                                          \\\hline
        Azioni      &
        TEST1:
        \begin{enumerate}[nosep, topsep=0pt]
            \item richiedere la visualizzazione dei contenuti suggeriti;
            \item verificare che il il sistema ritorni la lista dei contenuti trovati \emph{L};
            \item verificare che i contenuti in \emph{L} soddisfino i seguenti requisiti:
            \begin{itemize}
                \item 5 prodotti video sono del genere video preferito dell'utente;
                \item 5 serie tv sono del genere video preferito dell'utente;
                \item 5 prodotti musicali sono del genere audio preferito dell'utente;
                \item 5 album musicali sono del genere audio preferito dell'utente;
            \end{itemize}
        \end{enumerate}
        \\\hline
    \end{tabular}
\end{table}

\begin{table}[hb]
    \centering
    \begin{tabular}{ |p{2cm}|p{10cm}|  }
        \hline
        ID          & T\_47                                                                              \\\hline
        Caso d'uso  & UC\_GestisciPlaylist                                                           \\\hline
        Obbiettivo  & Testare il corretto funzionamento della funzionalità di gestione playlist \\\hline
        Prerequsiti & L'esecuzione del test è effettuata con privilegi di Utente                         \\\hline
        Tipologia   & White box                                                                          \\\hline
        Azioni      &
        TEST1:
        \begin{enumerate}[nosep, topsep=0pt]
            \item richiedere la visualizzazione della pagina di gestione delle playlist;
            \item verificare che il sistema mostri un elenco di playlist \emph{L} su cui inteprendere un'azione;
            \item fornisce la playlist selezionata \emph{P} e l'azione da intraprendere \emph{A}, a scelta tra: crea nuova playlist, rimuovere prodotto da playlist, cambiare stato pubblicazione playlist o rendere una playslist un album o una serie tv;
            \item verificare che il sistema mostri la pagina relativa alla richiesta descritta dalla coppia \emph{P}, \emph{A};
        \end{enumerate}
        \\\hline
    \end{tabular}
\end{table}

\begin{table}[hb]
    \centering
    \begin{tabular}{ |p{2cm}|p{10cm}|  }
        \hline
        ID          & T\_48                                                                              \\\hline
        Caso d'uso  & UC\_CreaPlaylist                                                           \\\hline
        Obbiettivo  & Testare il corretto funzionamento della funzionalità di creazione playlist \\\hline
        Prerequsiti & L'esecuzione del test è effettuata con privilegi di Utente                         \\\hline
        Tipologia   & White box                                                                          \\\hline
        Azioni      &
        TEST1:
        \begin{enumerate}[nosep, topsep=0pt]
            \item richiedere creazione di una playlist vuota, fornendo come titolo \emph{titolo\_np\_v};
            \item verificare che il sistema crei una playlist vuota \emph{P};
            \item verificare che il database contenga \emph{P};
            \item verificare che il sistema mostri un messaggio riepilogativo della creazione della playlist oppure un errore generico;
        \end{enumerate}
        \vspace{0.5cm} TEST2:
        \begin{enumerate}[nosep, topsep=0pt]
            \item richiedere creazione di una playlist vuota, fornendo come titolo \emph{titolo\_p\_v};
            \item verificare che il sistema non crei una playlist vuota \emph{P};
            \item verificare che il sistema mostri un messaggio di errore per mostrare che la playlist con il nome fornito esiste già;
        \end{enumerate}
        \\\hline
    \end{tabular}
\end{table}

\begin{table}[hb]
    \centering
    \begin{tabular}{ |p{2cm}|p{10cm}|  }
        \hline
        ID          & T\_49                                                                              \\\hline
        Caso d'uso  & UC\_AggiungiProdottoPlaylist                                                           \\\hline
        Obbiettivo  & Testare il corretto funzionamento della funzionalità di aggiunta di un prodotto a playlist \\\hline
        Prerequsiti & L'esecuzione del test è effettuata con privilegi di Utente                         \\\hline
        Tipologia   & White box                                                                          \\\hline
        Azioni      &
        TEST1:
        \begin{enumerate}[nosep, topsep=0pt]
            \item richiedere l'aggiunta di prodotto a playlist;
            \item verificare che il sistema ritorni la lista \emph{L} delle playlist associate all'utente corrente;
            \item fornire una playlist \emph{P} scelta in \emph{L} ed un prodotto \emph{contenuto\_np\_v};
            \item verificare che il sistema validi la richiesta di inserimento del prodotto nella playlist indicata;
            \item verificare che il database contenga \emph{contenuto\_np\_v} tra i prodotti della plyalist \emph{P};
            \item verificare che il sistema mostri un messaggio di avvenuto inserimento del prodotto nella playlist oppure di errore esplicativo;
        \end{enumerate}
        \vspace{0.5cm} TEST2:
        \begin{enumerate}[nosep, topsep=0pt]
            \item richiedere l'aggiunta di prodotto a playlist;
            \item verificare che il sistema ritorni la lista \emph{L} delle playlist associate all'utente corrente;
            \item fornire una playlist \emph{P} scelta in \emph{L} ed un prodotto \emph{contenuto\_np\_nv};
            \item verificare che il sistema non validi la richiesta di inserimento del prodotto nella playlist indicata;
            \item verificare che il sistema mostri un messaggio di errore di prodotto non adatto alla playlist;
        \end{enumerate}
        \\\hline
    \end{tabular}
\end{table}

\begin{table}[hb]
    \centering
    \begin{tabular}{ |p{2cm}|p{10cm}|  }
        \hline
        ID          & T\_50                                                                 \\\hline
        Caso d'uso  & UC\_RimuoviProdottoPlaylist                                                    \\\hline
        Obbiettivo  & Verificare che il prodotto venga rimosso dalla playlist con successo \\\hline
        Prerequsiti & L'esecuzione del test è effettuata con privilegi di Utente, \emph{playlist\_p\_v}
        è una playlist scelta \\\hline
        Tipologia   & White box                                                             \\\hline
        Azioni      &
        TEST1:
        \begin{enumerate}[nosep, topsep=0pt]
            \item richiedere al sistema la lista \emph{L} dei prodotti in \emph{playlist\_p\_v}
            \item richiedere al sistema la rimozione di un prodotto \emph{P} in \emph{L} da \emph{playlist\_p\_v};
            \item verificare che il messaggio ricevuto dal sistema sia di successo;
            \item verificare che nel database \emph{P} non sia in \emph{playlist\_p\_v};
        \end{enumerate}
        \\\hline
    \end{tabular}
\end{table}

\begin{table}[hb]
    \centering
    \begin{tabular}{ |p{2cm}|p{10cm}|  }
        \hline
        ID          & T\_51                                                                 \\\hline
        Caso d'uso  & UC\_CambiaVisibilitàPlaylist                                                    \\\hline
        Obbiettivo  & Verificare che la visibilità della playlist venga cambiata correttamente \\\hline
        Prerequsiti & L'esecuzione del test è effettuata con privilegi di Utente, \emph{playlist\_p\_v}
        è una playlist scelta \\\hline
        Tipologia   & White box                                                             \\\hline
        Azioni      &
        TEST1:
        \begin{enumerate}[nosep, topsep=0pt]
            \item richiedere di impostare la visibilità di \emph{playlist\_p\_v} a pubblica;
            \item verificare che il sistema risponda con un messaggio di successo;
            \item verificare che nel database \emph{playlist\_p\_v} sia pubblica;
        \end{enumerate}
        TEST2:
        \begin{enumerate}[nosep, topsep=0pt]
            \item richiedere di impostare la visibilità di \emph{playlist\_p\_v} a privata;
            \item verificare che il sistema risponda con un messaggio di successo;
            \item verificare che nel database \emph{playlist\_p\_v} sia privata;
        \end{enumerate}
        \\\hline
    \end{tabular}
\end{table}

\begin{table}[hb]
    \centering
    \begin{tabular}{ |p{2cm}|p{10cm}|  }
        \hline
        ID          & T\_52                                                                 \\\hline
        Caso d'uso  & UC\_RiproduciPlaylist                                                    \\\hline
        Obbiettivo  & Verificare che i prodotti nella playlist vengano aggiunti alla coda di riproduzione correttamente \\\hline
        Prerequsiti & L'esecuzione del test è effettuata con privilegi di Utente, \emph{playlist\_p\_v}
        è una playlist scelta \\\hline
        Tipologia   & White box                                                             \\\hline
        Azioni      &
        TEST1:
        \begin{enumerate}[nosep, topsep=0pt]
            \item richiedere la riproduzione di \emph{playlist\_p\_v};
            \item verificare che il player avvi la riproduzione dei prodotti della playlist;
            \item verificare che nel database in \emph{coda\_p\_v} contenga tutti e soli i prodotti di \emph{playlist\_p\_v};
        \end{enumerate}
        \\\hline
    \end{tabular}
\end{table}

\begin{table}[hb]
    \centering
    \begin{tabular}{ |p{2cm}|p{10cm}|  }
        \hline
        ID          & T\_53                                                                              \\\hline
        Caso d'uso  & UC\_CreaSerieTv                                                             \\\hline
        Obbiettivo  & Verificare la playlist venga convertita in serie tv correttamente               \\\hline
        Prerequsiti & L'esecuzione del test è effettuata con privilegi di Utente, \emph{playlist\_p\_v} e \emph{playlist\_p\_nv}
        sono playlist scelte                                                                            \\\hline
        Tipologia   & White box                                                                          \\\hline
        Azioni      &
        TEST1:
        \begin{enumerate}[nosep, topsep=0pt]
            \item richiedere la conversione di \emph{playlist\_p\_v} (dove tutti i prodotti sono video, appartengono
            all'utente stesso e non esiste alcuna serie tv che contiene neanche uno di questi) in Serie tv;
            \item verificare che il sistema ritorni il messaggio di successo;
            \item verificare che nel database \emph{playlist\_p\_v} sia una serie tv;
        \end{enumerate}
        \vspace{0.5cm} TEST2:
        \begin{enumerate}[nosep, topsep=0pt]
            \item richiedere la conversione di \emph{playlist\_p\_nv} (dove un prodotto non appartiene all'utente stesso) in Serie tv;
            \item verificare che il sistema ritorni il messaggio di errore per prodotto non posseduto;
            \item verificare che nel database \emph{playlist\_p\_nv} non sia una serie tv;
        \end{enumerate}
        \vspace{0.5cm} TEST3:
        \begin{enumerate}[nosep, topsep=0pt]
            \item richiedere la conversione di \emph{playlist\_p\_nv} (dove un prodotto non è video) in Serie tv;
            \item verificare che il sistema ritorni il messaggio di errore per prodotto non video;
            \item verificare che nel database \emph{playlist\_p\_nv} non sia una serie tv;
        \end{enumerate}
        \vspace{0.5cm} TEST4:
        \begin{enumerate}[nosep, topsep=0pt]
            \item richiedere la conversione di \emph{playlist\_p\_nv} (dove un prodotto appartiene ad un'altra serie tv) in Serie tv;
            \item verificare che il sistema ritorni il messaggio di errore per prodotto appartenente ad altra serie tv;
            \item verificare che nel database \emph{playlist\_p\_nv} non sia una serie tv;
        \end{enumerate}
        \\\hline
    \end{tabular}
\end{table}

\begin{table}[hb]
    \centering
    \begin{tabular}{ |p{2cm}|p{10cm}|  }
        \hline
        ID          & T\_54                                                                              \\\hline
        Caso d'uso  & UC\_CreaAlbum                                                             \\\hline
        Obbiettivo  & Verificare la playlist venga convertita in album correttamente               \\\hline
        Prerequsiti & L'esecuzione del test è effettuata con privilegi di Utente, \emph{playlist\_p\_v} e \emph{playlist\_p\_nv}
        sono playlist scelte                                                                            \\\hline
        Tipologia   & White box                                                                          \\\hline
        Azioni      &
        TEST1:
        \begin{enumerate}[nosep, topsep=0pt]
            \item richiedere la conversione di \emph{playlist\_p\_v} (dove tutti i prodotti sono musicali, appartengono
            all'utente stesso e non esiste alcun album che contiene neanche uno di questi) in Album;
            \item verificare che il sistema ritorni il messaggio di successo;
            \item verificare che nel database \emph{playlist\_p\_v} sia un Album;
        \end{enumerate}
        \vspace{0.5cm} TEST2:
        \begin{enumerate}[nosep, topsep=0pt]
            \item richiedere la conversione di \emph{playlist\_p\_nv} (dove un prodotto non appartiene all'utente stesso) in Album;
            \item verificare che il sistema ritorni il messaggio di errore per prodotto non posseduto;
            \item verificare che nel database \emph{playlist\_p\_nv} non sia un Album;
        \end{enumerate}
        \vspace{0.5cm} TEST3:
        \begin{enumerate}[nosep, topsep=0pt]
            \item richiedere la conversione di \emph{playlist\_p\_nv} (dove un prodotto non è musicale) in Album;
            \item verificare che il sistema ritorni il messaggio di errore per prodotto non musicale;
            \item verificare che nel database \emph{playlist\_p\_nv} non sia un Album;
        \end{enumerate}
        \vspace{0.5cm} TEST4:
        \begin{enumerate}[nosep, topsep=0pt]
            \item richiedere la conversione di \emph{playlist\_p\_nv} (dove un prodotto appartiene ad un altro album) in Album;
            \item verificare che il sistema ritorni il messaggio di errore per prodotto appartenente ad altro album;
            \item verificare che nel database \emph{playlist\_p\_nv} non sia un Album;
        \end{enumerate}
        \\\hline
    \end{tabular}
\end{table}

\begin{table}[hb]
    \centering
    \begin{tabular}{ |p{2cm}|p{10cm}|  }
        \hline
        ID          & T\_55                                                                 \\\hline
        Caso d'uso  & UC\_AggiungiProdottoAllaCoda                                                    \\\hline
        Obbiettivo  & Verificare che il prodotto venga aggiunto alla coda di riproduzione con successo \\\hline
        Prerequsiti & L'esecuzione del test è effettuata con privilegi di Utente, contenuto\_p\_v
        e contenuto\_p\_nv sono i contenuti forniti           \\\hline
        Tipologia   & White box                                                             \\\hline
        Azioni      &
        TEST1:
        \begin{enumerate}[nosep, topsep=0pt]
            \item richiedere al sistema l'aggiunta di \emph{contenuto\_p\_v} alla coda\_p\_v;
            \item verificare che il messaggio ricevuto dal sistema sia di successo;
            \item verificare che nel database \emph{contenuto\_p\_v} sia in coda\_p\_v;
        \end{enumerate}
        \vspace{0.5cm} TEST2:
        \begin{enumerate}[nosep, topsep=0pt]
            \item richiedere al sistema l'aggiunta di \emph{contenuto\_p\_nv} alla coda\_p\_v;
            \item verificare che il messaggio ricevuto dal sistema contenga l'errore per
            prodotto fornito non valido;
            \item verificare che nel database \emph{contenuto\_p\_nv} non sia in coda\_p\_v;
        \end{enumerate}
        \\\hline
    \end{tabular}
\end{table}

\begin{table}[hb]
    \centering
    \begin{tabular}{ |p{2cm}|p{10cm}|  }
        \hline
        ID          & T\_56                                                                 \\\hline
        Caso d'uso  & UC\_RimuoviProdottoDallaCoda                                                    \\\hline
        Obbiettivo  & Verificare che il prodotto venga rimosso dalla coda di riproduzione con successo \\\hline
        Prerequsiti & L'esecuzione del test è effettuata con privilegi di Utente, contenuto\_p\_v
        e contenuto\_p\_nv sono i contenuti forniti           \\\hline
        Tipologia   & White box                                                             \\\hline
        Azioni      &
        TEST1:
        \begin{enumerate}[nosep, topsep=0pt]
            \item richiedere al sistema la rimozione di \emph{contenuto\_p\_v} da coda\_p\_v;
            \item verificare che il messaggio ricevuto dal sistema sia di successo;
            \item verificare che nel database \emph{contenuto\_p\_v} non sia in coda\_p\_v;
        \end{enumerate}
        \vspace{0.5cm} TEST2:
        \begin{enumerate}[nosep, topsep=0pt]
            \item richiedere al sistema la rimozione di \emph{contenuto\_p\_nv} da coda\_p\_v;
            \item verificare che il messaggio ricevuto dal sistema contenga l'errore per
            contenuto non presente in coda;
        \end{enumerate}
        \\\hline
    \end{tabular}
\end{table}

\begin{table}[hb]
    \centering
    \begin{tabular}{ |p{2cm}|p{10cm}|  }
        \hline
        ID          & T\_57                                                                 \\\hline
        Caso d'uso  & UC\_MostraStatoCoda                                                    \\\hline
        Obbiettivo  & Verificare che vengano visualizzati tutti e soli gli elementi nella coda \\\hline
        Prerequsiti & L'esecuzione del test è effettuata con privilegi di Utente            \\\hline
        Tipologia   & Black box                                                             \\\hline
        Azioni      &
        TEST1:
        \begin{enumerate}[nosep, topsep=0pt]
            \item richiedere la lista \emph{L} dei prodotti in coda\_p\_v;
            \item verificare che gli elementi in \emph{L} siano tutti e soli i prodotti che sono nella
            coda di riproduzione;
        \end{enumerate}
        \\\hline
    \end{tabular}
\end{table}

\begin{table}[hb]
    \centering
    \begin{tabular}{ |p{2cm}|p{10cm}|  }
        \hline
        ID          & T\_58                                                                 \\\hline
        Caso d'uso  & UC\_RiproduciCoda                                                     \\\hline
        Obbiettivo  & Verificare che la riproduzione della coda venga avviata correttamente \\\hline
        Prerequsiti & L'esecuzione del test è effettuata con privilegi di Utente            \\\hline
        Tipologia   & Black box                                                             \\\hline
        Azioni      &
        TEST1:
        \begin{enumerate}[nosep, topsep=0pt]
            \item richiedere la riproduzione della coda coda\_p\_v che contiene almeno un prodotto;
            \item verificare che il player inizi a riprodurre il prossimo prodotto in coda\_p\_v;
            \item verificare che, dopo la riproduzione del prodotto, il player inizi a riprodurre
                  il prossimo prodotto, se esso esiste (e si ripete il punto 3), oppure termini\_p\_v;
        \end{enumerate}
        \vspace{0.5cm} TEST2:
        \begin{enumerate}[nosep, topsep=0pt]
            \item richiedere la riproduzione della coda coda\_p\_nv che non contiene prodotti;
            \item verificare che il sistema restituisca un messaggio di errore per coda vuota;
        \end{enumerate}
        \\\hline
    \end{tabular}
\end{table}

\begin{table}[hb]
    \centering
    \begin{tabular}{ |p{2cm}|p{10cm}|  }
        \hline
        ID          & T\_59                                                                              \\\hline
        Caso d'uso  & UC\_CalcolaQualitàContenuto                                                        \\\hline
        Obbiettivo  & Verificare che il voto del contenuto sia corretto, in base ai voti ricevuti        \\\hline
        Prerequsiti & L'esecuzione del test è effettuata con privilegi di Utente, \emph{contenuto\_p\_v}
        è un contenuto scelto                                                                            \\\hline
        Tipologia   & Black box                                                                          \\\hline
        Azioni      &
        TEST1:
        \begin{enumerate}[nosep, topsep=0pt]
            \item richiedere il voto \emph{V} della qualità di \emph{contenuto\_p\_v};
            \item se \emph{contenuto\_p\_v} è un prodotto Video o Musicale,
                  verificare che il sistema restituisca \emph{V} = \emph{contenuto\_p\_v.sommavoti} / \emph{contenuto\_p\_v.voti\_totali}
            \item se \emph{contenuto\_p\_v} è una Playlist,
                  verificare che il sistema restituisca \emph{V} =
                  $\sum_{p\ in\ all\ Prodotto\ in\ \emph{contenuto\_p\_v}} \emph{p.sommavoti}$
                  /
                  $\sum_{p in all Prodotto in \emph{contenuto\_p\_v}} \emph{p.voti\_totali}$
        \end{enumerate}
        \\\hline
    \end{tabular}
\end{table}

\begin{table}[hb]
    \centering
    \begin{tabular}{ |p{2cm}|p{10cm}|  }
        \hline
        ID          & T\_60                                                                              \\\hline
        Caso d'uso  & UC\_VotaContenuto                                                                  \\\hline
        Obbiettivo  & Verificare che il feedback venga aggiunto al contenuto correttamente               \\\hline
        Prerequsiti & L'esecuzione del test è effettuata con privilegi di Utente, \emph{contenuto\_p\_v}
        è un contenuto scelto                                                                            \\\hline
        Tipologia   & White box                                                                          \\\hline
        Azioni      &
        TEST1:
        \begin{enumerate}[nosep, topsep=0pt]
            \item fornire un numero \emph{NUM} \textgreater 0 e \textless 6 al sistema
            \item verificare che il sistema ritorni il messaggio di avvenuta pubblicazione
                  di un feedback con voto \emph{NUM} per \emph{contenuto\_p\_v};
            \item verificare che nel database esista un feedback con voto \emph{NUM} associato a
                  \emph{contenuto\_p\_v};
        \end{enumerate}
        \vspace{0.5cm} TEST2:
        \begin{enumerate}[nosep, topsep=0pt]
            \item fornire un numero \emph{NUM} \textless 1 o \textgreater 5 al sistema
            \item verificare che il sistema ritorni il messaggio di errore per input invalido;
            \item verificare che nel database non esista un feedback con voto \emph{NUM} associato a
                  \emph{contenuto\_p\_v};
        \end{enumerate}
        \\\hline
    \end{tabular}
\end{table}

\begin{table}[hb]
    \centering
    \begin{tabular}{ |p{2cm}|p{10cm}|  }
        \hline
        ID          & T\_61                                                                              \\\hline
        Caso d'uso  & UC\_CommentaContenuto                                                              \\\hline
        Obbiettivo  & Verificare che il commento venga aggiunto al contenuto correttamente               \\\hline
        Prerequsiti & L'esecuzione del test è effettuata con privilegi di Utente, \emph{contenuto\_p\_v}
        è un contenuto scelto                                                                            \\\hline
        Tipologia   & White box                                                                          \\\hline
        Azioni      &
        TEST1:
        \begin{enumerate}[nosep, topsep=0pt]
            \item fornire \emph{commento\_np\_v} al sistema
            \item verificare che il sistema ritorni il messaggio di avvenuta pubblicazione
                  di \emph{commento\_np\_v} per \emph{contenuto\_p\_v};
            \item verificare che nel database esista \emph{commento\_np\_v} associato a
                  \emph{contenuto\_p\_v};
        \end{enumerate}
        \vspace{0.5cm} TEST2:
        \begin{enumerate}[nosep, topsep=0pt]
            \item fornire \emph{commento\_np\_nv} al sistema
            \item verificare che il sistema ritorni il messaggio di errore per input non valido;
            \item verificare che nel database non esista \emph{commento\_np\_nv} associato a
                  \emph{contenuto\_p\_v};
        \end{enumerate}
        \\\hline
    \end{tabular}
\end{table}

\begin{table}[hb]
    \centering
    \begin{tabular}{ |p{2cm}|p{10cm}|  }
        \hline
        ID          & T\_62                                                      \\\hline
        Caso d'uso  & UC\_RimuoviCommento                                        \\\hline
        Obbiettivo  & Verificare che il commento venga rimosso correttamente     \\\hline
        Prerequsiti & L'esecuzione del test è effettuata con privilegi di Utente \\\hline
        Tipologia   & White box                                                  \\\hline
        Azioni      &
        TEST1:
        \begin{enumerate}[nosep, topsep=0pt]
            \item richiedere la rimozione di \emph{commento\_p\_v}
            \item verificare che il sistema ritorni il messaggio di avvenuta cancellazione di \emph{commento\_p\_v};
            \item verificare che nel database non esista \emph{commento\_p\_v};
        \end{enumerate}
        \\\hline
    \end{tabular}
\end{table}


\begin{table}[hb]
    \centering
    \begin{tabular}{ |p{2cm}|p{10cm}|  }
        \hline
        ID          & T\_63                                                              \\\hline
        Caso d'uso  & UC\_AttivaAbbonamento                                              \\\hline
        Obbiettivo  & Testare che un abbonamento diventi sottoscrivibile                 \\\hline
        Prerequsiti & L'esecuzione del test è effettuata con privilegi di Amministratore \\\hline
        Tipologia   & Black box                                                          \\\hline
        Azioni      &
        TEST1:
        \begin{enumerate}[nosep, topsep=0pt]
            \item richiedere l'attivazione di \emph{piano\_abb\_p\_v}
            \item verificare che il sistema ritorni il messaggio di avvenuta attivazione dell'abbonamento;
        \end{enumerate}
        \\\hline
    \end{tabular}
\end{table}