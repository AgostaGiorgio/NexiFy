\subsection*{Tipologia}
I test saranno effettuati in due diverse tipologie:
\begin{itemize}
    \item Black box: valutano le funzionalità dell'applicazione a partire dall'interfaccia utente, che
    sia quella degli amministratori o degli utenti fruitori del servizio. Quindi si ignorano le dinamiche
    interne al sistema. Contengono anche test di esperienza e usabilità utente, ovvero quelli che richiedono
    un interazione umana.
    \item White box: provano la correttezza del sistema con test unitari sui suoi componenti interni, 
    così come le interazioni tra i vari componenti.
\end{itemize}

\subsection*{Ambiente di test}
Per i test di tipo White Box vengono definiti i seguenti programmi di supporto:
\begin{itemize}
    \item Un programma per intercettare le comunicazioni client-server e server-server, per controllare che le
    comunicazioni avvengano in modo corretto;
    \item Un programma per verificare lo stato ed il contenuto del database;
    \item Un profiler per controllare lo stato delle performance e verificare l’assenza di colli di bottiglia;
\end{itemize}
L’ambiente viene re-inizializzato per ogni test, in quanto eseguito su un container.

\subsection*{Dati su cui vengono eseguiti i test}

\subsubsection*{Dati non presenti nel database}
\begin{itemize}
    \item utente\_signup
    \begin{itemize}
        \item username
        \item password
    \end{itemize}
\end{itemize}

\subsubsection*{Dati presenti nel database}
\begin{itemize}
    \item utente\_login
    \begin{itemize}
        \item username
        \item password
    \end{itemize}
\end{itemize}

\subsection*{Descrizione priorità dei test}

Vengono definite le seguenti priorità per classificare i vari test in base a quanto importante sia la 
loro corretta e periodica esecuzione sull'integrità del sistema.
\begin{itemize}
    \item H (alta priorità): relativa ai test che devono essere eseguiti correttamente per garantire l'integrità
    del sistema;
    \item M (media priorità): relativa ai test che devono essere eseguiti correttamente per garantire l'integrità
    del sistema, ma solo successivamente ai test di priorità H;
    \item L (bassa priorità): relativa ai test che possono essere eseguiti, per garantire una migliore funzionalità
    del sistema, ma solo successivamente ai test di priorità M.
\end{itemize}
