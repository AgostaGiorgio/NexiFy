\subsection{Aspetto tecnologico}
Per la realizzazione del sistema sarà necessario l'utilizzo di CDN, in modo da distribuire i contenuti
multimediali in maniera efficiente e mantenendoli sempre accessibili per gli utenti. Saranno anche necessarie
basi di dati cloud-based di diversi tipi:
\begin{itemize}
	\item relazionali per mantenere informazioni su dati utenti
	(oltre alle informazioni di base, anche le eventuali playlist musicali create, ecc) e sui listing dei contenuti.
	\item NoSQL per mantenere dati sulle richieste effettuate dagli utenti
	(per creare statistiche, classifiche, suggerimenti e altre funzionalità), anonimi ove possibile.
\end{itemize}
Questa suddivisione rende possibile trattare i dati personali degli utenti con maggiore sicurezza.
Alcune CDN mantengono già informazioni sulle richieste degli utenti,
ma per mantenere il sistema il più indipendente possibile dalla tecnologia utilizzata, sarà NexiFy stesso a
tenere traccia di tali dati. \\
La fattibilità del progetto dal punto di vista tecnologico segue dalla grande disponibilità di aziende che offrono servizi cloud, tra cui CDN: per esempio AWS CloudFront. Questi servizi includono anche delle API per interfacciarsi in maniera efficiente con la CDN. Notiamo che questi servizi vengono già usati da piattaforme simili a NexiFy, e che sono risultate in grado di scalare a livello mondiale (ad esempio Prime Video usa AWS CloudFront). Anche per quanto riguarda i database esistono numerose soluzioni cloud affidabili.
%Il mantenimento dei dati riguardanti gli utenti sarà conforme alle politiche GDPR per garantire la privacy secondo le leggi europee.\\


%Attraverso le richieste ricevute dal server, che risponderà con dei manifest file per indicare al client quali sono gli indirizzi cdn a lui più convenienti, verranno salvati nel database dei dati sull’accesso degli utenti ai singoli contenuti. In questo modo si potranno generare analytics per un costante miglioramento del servizio. 


%Per la realizzazione del sistema sarà necessario l'utilizzo di CDN, in modo da distribuire i contenuti multimediali in maniera efficiente e mantenendoli sempre accessibili per gli utenti. Deve inoltre essere possibile, in maniera efficiente, estrarre dati e statistiche dalla CDN (come per esempio quante volte è stato richiesto un certo video in una certa zona). Saranno necessarie basi di dati NoSQL cloud-based per mantenere informazioni su dati utenti (oltre alle informazioni di base, anche le eventuali playlist musicali create, ecc), e implementando funzionalità lato server per gestire le richieste degli utenti, la loro autenticazione e altro. Verrà inoltre utilizzato un ORM per garantire semplicità di migrazione ad un'altra tecnologia di memorizzazione dati permettendo di rendere indipendente il codice dalla base di dati.\\

\subsection{Aspetto Economico}
\subsubsection{Vantaggi del sistema}

NexiFy sarà in grado di acquisire utenti grazie ai diversi vantaggi offerti:
    \begin{itemize}
       	\item La disponibilità di contenuti sia video che musicali, non presente in piattaforme quali Netflix e Spotify. Questo darà la possibilità agli utenti interessati di visualizzare film, serie tv e ascoltare brani musicali sottoscrivendo un singolo abbonamento, comportando un risparmio di denaro, in quanto non è necessario iscriversi a diversi servizi il cui costo totale ammonterebbe ad una quota mensile maggiore;
        	\item La possibilità per piccoli creatori di pubblicare autonomamente contenuti, ma senza degradare la qualità dei contenuti (come invece avviene in piattaforme quali YouTube).
	\item\label{VantaggioSistema_cambioPiano} Oltre al poter disdire un abbonamento, per poi attivarne un altro; verrà data la possibilità agli utenti che fruiscono della
	piattaforma di poter cambiare piano di abbonamento con un altro piano
	offerto, mentre questo è ancora in corso di validità.
	In questo modo, qualora l'utente avesse ripensamenti potrebbe cambiare piano evitando di rimanere insoddisfatto.
	In seguito ad un cambiamento di piano, avverrà un'interruzione del rinnovo del piano
	precedente e l'attivazione del nuovo piano scelto, con rinnovo che inizia nel giorno stesso. Questa funzionalità sarà utilizzabile solamente in casi speciali (ad esempio non si può passare ad un piano di abbonamento di costo inferiore).
	\end{itemize}
\subsubsection{Stima dei costi}

NexiFy inizialmente supporterà un numero modesto di utenti e di contenuti multimediali; questo per avere dei costi iniziali sostenibili, e man mano che si acquisiranno utenti (e quindi sottoscrizioni di abbonamenti) sarà possibile ampliare la CDN e i database (dal punto di vista software NexiFy sarà già progettato per supportare un carico maggiore del carico iniziale).\\ \\

% \textbf{TODO... conti precisi}
\noindent{\LARGE \textbf{Revisioni 1-3}} \\ \\
\begin{tabular}{|c | c | c | c|} 
 	\hline
	 Numero & Data & Descrizione \\ [0.5ex] 
	\hline\hline
	1 & 21/11/2019 & Stesura iniziale \\ 
	\hline
	2 & 27/11/2019 & Modifica sulla gestione dei dati per analytics \\
	\hline
	3 & 1/12/2019 & Passaggio a db NoSQL con conseguente rimozione di ORM \\
	\hline
\end{tabular}