\setlength{\arrayrulewidth}{.5mm}
\setlength{\tabcolsep}{5pt}
\renewcommand{\arraystretch}{2}
\renewcommand{\labelenumii}{\theenumii}
\renewcommand{\theenumii}{\theenumi.\arabic{enumii}.}

\subsection{Introduzione}
La seguente sezione mira all'identificazione dei casi d'uso presenti nel progetto. Di seguito vengono identificati oltre ai casi d'uso, anche gli attori coinvolti nel loro utilizzo.
Lo scopo principale del documento, è quello di avere un riferimento sui casi d'uso leggibile da chiunque, anche a personale non interno al progetto. Questo è utile al fine di permettere a tutti una comprensione e quindi una discussione sui casi d'uso.

\subsection{Legenda}
\large{\textbf{Attori}} \\
\begin{itemize}[]
	\item \textbf{ID}: rappresenta l'identificatore univoco di un attore. La sintassi è del tipo A\_X dove X è una stringa di interi separati da punto che rappresenta la gerarchia di relazioni padre-figlio.
	\item \textbf{Nome}: nome dell'attore, deve essere comunque univoco e deve dare una prima idea di cosa rappresenta quel preciso attore.
	\item \textbf{Genitore}: indica l'identificatore del primo antenato nella gerarchia degli attori.
	\item \textbf{Livello}: indica il grado di rilevanza dell'attore all'interno dell'intero sistema. Può assumere il valore di {Primario, Secondario, Di Supporto}.
	\item \textbf{Tipologia}: classifica l'attore in {Umano, Sistema}
	\item \textbf{Descrizione}: descrive brevemente cosa rappresenta all'interno del sistema l'attore.\
\end{itemize}


\noindent \large{\textbf{Use Case}} \\
\begin{itemize}[]
	\item \textbf{ID}: identificativo univoco del caso d'uso, è della forma UC\_X dove X è una stringa di interi separati da punto che rappresenta la gerarchia di relazioni padre-figlio.
	\item \textbf{Categoria}: Macroarea di appartenza della funzionalità. Può assumere i valori {User, Subscriptions, Management}
	\item \textbf{Nome}: Nome identificativo del caso d'uso, serve anche per dare un'idea del tipo di attività.
	\item \textbf{Priorità}: Livello di priorità del caso d'uso. Può assumere i valori {High, Medium, Low}. 
	\item \textbf{Attori}: Lista di identificativi degli attori coinvolti nel caso d'uso.
	\item \textbf{Descrizione}: Breve descrizione della funzione svolta dal caso d'uso.
	\item \textbf{Input}: Lista di cosa riceve in input il caso d'uso (se non indicato, si assume che non riceve niente in input)
	\item \textbf{Output}: Lista di cosa restituisce il caso d'uso (se non è indicato, si assume che non restituisce niente)
	\item \textbf{Pre-condizioni}: Condizioni che devono essere verificate prima che il caso d'uso inizi, altrimenti non potrà essere svolto.
	\item \textbf{Post-condizioni}: Condizioni che devono essere verificate subito dopo l'esecuzione del caso d'uso.
	\item \textbf{Flusso}: Passi logici eseguiti dal caso d'uso per ottenere il risultato desiderato.
	%\item \textbf{Flusso Alternativo}: 

\end{itemize}

% ============================== ATTORI ====================================

\subsection{Specifica Attori}

\begin{center}

\begin{tabular}{ |p{2cm}|p{10cm}|  }
\hline
ID & A\_1 \\\hline
Nome & Utente\\\hline
Genitore & - \\\hline
Livello &  Primario \\\hline
Tipologia & Umano \\\hline
Descrizione &  L' Utente è il ruolo assunto da tutti coloro che usufruiranno del servizio di streaming offerto dalla piattaforma. \\\hline
\end{tabular}
\label{table_attore:1}\newline

\begin{tabular}{ |p{2cm}|p{10cm}|  }
\hline
ID & A\_1.1 \\\hline
Nome & UtenteNonAutenticato\\\hline
Genitore & A\_1 \\\hline
Livello &  Primario \\\hline
Tipologia & Umano \\\hline
Descrizione &  Questo ruolo è assunto da tutti coloro che utilizzano la piattaforma senza aver effettuato l'accesso ad un profilo utente e che quindi avranno accesso solo a determinate funzionalità \\\hline
\end{tabular}
\label{table_attore:1.1}\newline

\begin{tabular}{ |p{2cm}|p{10cm}|  }
\hline
ID & A\_1.2 \\\hline
Nome & UtenteAutenticato\\\hline
Genitore & A\_1 \\\hline
Livello &  Primario \\\hline
Tipologia & Umano \\\hline
Descrizione &  Questo ruolo è assunto da tutti gli utenti che hanno sottoscritto un contratto con la piattaforma e si sono autenticati in fase di accesso ad essa. \\\hline
\end{tabular}
\label{table_attore:1.2}\newline

\begin{tabular}{ |p{2cm}|p{10cm}|  }
\hline
ID & A\_2 \\\hline
Nome & Staff\\\hline
Genitore & -\\\hline
Livello &  Primario \\\hline
Tipologia & Umano \\\hline
Descrizione &  Rappresenta il ruolo di staff della piattaforma \\\hline
\end{tabular}
\label{table_attore:2}\newline

\begin{tabular}{ |p{2cm}|p{10cm}|  }
\hline
ID & A\_2.1 \\\hline
Nome & Amministratore\\\hline
Genitore & A\_2\\\hline
Livello &  Primario \\\hline
Tipologia & Umano \\\hline
Descrizione &  Rappresenta il ruolo amministratore della piattaforma, cioè un membro dello staff incaricato della gestione di vari aspetti della piattaforma \\\hline
\end{tabular}
\label{table_attore:2.1}\newline

\begin{tabular}{ |p{2cm}|p{10cm}|  }
\hline
ID & A\_2.1.1 \\\hline
Nome & ManagerAbbonamenti\\\hline
Genitore & A\_2.1\\\hline
Livello &  Primario \\\hline
Tipologia & Umano \\\hline
Descrizione &  Rappresenta il ruolo di chi gestisce i servizi offerti dalla piattaforma \\\hline
\end{tabular}
\label{table_attore:2.1.1}\newline

\begin{tabular}{ |p{2cm}|p{10cm}|  }
\hline
ID & A\_2.1.2 \\\hline
Nome & ManagerSegnalazioni\\\hline
Genitore & A\_2.1\\\hline
Livello &  Primario \\\hline
Tipologia & Umano \\\hline
Descrizione &  Rappresenta il ruolo di chi gestisce tutte le segnalazioni verso prodotti o utenti \\\hline
\end{tabular}
\label{table_attore:2.1.2}\newline

\begin{tabular}{ |p{2cm}|p{10cm}|  }
\hline
ID & A\_3 \\\hline
Nome & Pagamento\\\hline
Genitore & - \\\hline
Livello &  Primario \\\hline
Tipologia & Sistema \\\hline
Descrizione &  Rappresenta il sistema di pagamento esterno al sistema della piattaforma \\\hline
\end{tabular}
\label{table_attore:3}\newline

\begin{tabular}{ |p{2cm}|p{10cm}|  }
\hline
ID & A\_4 \\\hline
Nome & Database\\\hline
Genitore & - \\\hline
Livello &  Primario \\\hline
Tipologia & Sistema \\\hline
Descrizione &  Rappresenta il sistema usato per la memorizzazione di informazioni. \\\hline
\end{tabular}
\label{table_attore:4}\newline

\begin{tabular}{ |p{2cm}|p{10cm}|  }
\hline
ID & A\_5 \\\hline
Nome & CDN\\\hline
Genitore & - \\\hline
Livello &  Primario \\\hline
Tipologia & Sistema \\\hline
Descrizione &  Rappresenta il sistema di distribuzione dei contenuti. \\\hline
\end{tabular}
\label{table_attore:5}\newline

\begin{tabular}{ |p{2cm}|p{10cm}|  }
\hline
ID & A\_6 \\\hline
Nome & Time\\\hline
Genitore & - \\\hline
Livello &  Primario \\\hline
Tipologia & Sistema \\\hline
Descrizione &  Rappresenta la componente responsabile dello scheduling delle operazioni \\\hline
\end{tabular}
\label{table_attore:6}\newline


\end{center}

%==== gestione errori di default ====
\subsection{Gestione Fallimenti}
Specifichiamo qui le politiche standard di gestione dei fallimenti di sistemi esterni. Se negli use case non viene specificato esplicitamente come gestire un fallimento di questi sistemi, si assume che entrino in atto queste politiche.

\begin{tabular}{ |p{2cm}|p{10cm}|  }
\hline
ID & F\_1\\\hline
Nome & F\_FallimentoDatabase\\\hline
Attore & Database\\\hline
Descrizione & una richiesta al database produce un errore, e non viene quindi restituito il risultato richiesto\\\hline
Gestione &  La procedura che ha effettuato la richiesta al database viene abortita, riportando un messaggio di errore \\\hline
\end{tabular}
\label{table_fail:1}\newline

\begin{tabular}{ |p{2cm}|p{10cm}|  }
\hline
ID & F\_2\\\hline
Nome & F\_FallimentoCDN\\\hline
Attore & CDN\\\hline
Descrizione & una richiesta alla CDN produce un errore\\\hline
Gestione &  La procedura che ha effettuato la richiesta alla CDN viene abortita, riportando un messaggio di errore \\\hline
\end{tabular}
\label{table_fail:2}\newline

% ================ USE CASE CHE VANNO BENE ===========
\subsection{Specifica Use Case}

\begin{center}

%======= UC relativi ai requisiti funzionali 1. ==========
\begin{table}[bp]
    \centering
    \addtolength{\leftskip} {-2cm}
\begin{tabular}{ |p{2.6cm}|p{13cm}|  }
\hline
ID & UC\_\lastUC \\\hline
Categoria & Management\\\hline
Nome & UC\_CreaAbbonamento\\\hline
Priorità & High \\\hline
Attori &  ManagerAbbonamenti \\\hline
Descrizione & Crea un nuovo piano di abbonamento senza nessun servizio associato.\\\hline
Pre-condizioni &  \'E possibile creare un nuovo piano di abbonamento\\\hline
Post-condizioni &  Esiste un piano di abbonamento con il nome fornito dal manager degli abbonamenti\\\hline
Flusso &  	\begin{enumerate}
			\item Il manager degli abbonamenti richiede la creazione di un nuovo piano di abbonamento;
			\item Il sistema richiede al manager degli abbonamenti il nome, il prezzo e la durata del nuovo piano di abbonamento;
			\item Il manager degli abbonamenti fornisce i dati;
			\item Il sistema valida gli input, e:
				\begin{enumerate}[  ]
				\item \textbf{\# alt1}: Se non avvengono errori durante la validazione,  il sistema ricerca se esistono altri piani di abbonamento con il nome inserito:
					\begin{enumerate}[label*=\arabic*.]
					\item \textbf{\# alt1}: Se non esistono altri piani di abbonamento con il nome inserito: il piano di abbonamento viene creato, con i dati inseriti dal manager degli abbonamenti, la lista dei servizi inizialmente vuota e inizialmente non sottoscrivibile
					\item \textbf{\# alt2}: Se esistono altri piani di abbonamento con il nome inserito: la creazione viene annullata e viene comunicato al manager degli abbonamenti che esiste già un piano di abbonamento con il nome inserito	
					\end{enumerate}
				\item \textbf{\# alt2}: Se avvengono errori durante la validazione: il piano di abbonamento non viene creato e l'errore viene comunicato al manager degli abbonamenti
				\end{enumerate}
			
			\end{enumerate}\\\hline
\end{tabular}
\label{table_use_case:\lastUC}\newline
\end{table}

\begin{table}[bp]
    \centering
    \addtolength{\leftskip} {-2cm}
\begin{tabular}{ |p{2.6cm}|p{13cm}|  }
\hline
ID & UC\_\nextUC \\\hline
Categoria & Management\\\hline
Nome & UC\_RecuperaAbbonamentiEsistenti\\\hline
Priorità & High \\\hline
Attori &  Staff \\\hline
Descrizione & Recupera tutti i piani di abbonamento attualmente esistenti.\\\hline
Input &  - \\\hline
Output &  Tutti i piani di abbonamento esistenti\\\hline
Pre-condizioni &  Nessuna \\\hline
Post-condizioni &  Tutti i piani di abbonamento esistenti vengono restituiti\\\hline
Flusso &  	\vspace{-5mm} \begin{enumerate}
			\item Il sistema effettua una richiesta al sistema Database per richiedere tutti i piani di abbonamento
			\item viene restituita la lista dei piani di abbonamento trovati
		\end{enumerate}\\\hline
\end{tabular}
\label{table_use_case:\lastUC}\newline
\end{table}

\begin{table}[bp]
    \centering
    \addtolength{\leftskip} {-2cm}
\begin{tabular}{ |p{2.6cm}|p{13cm}|  }
\hline
ID & UC\_\nextUC \\\hline
Categoria & Management\\\hline
Nome & UC\_RecuperaServizi\\\hline
Priorità & High \\\hline
Attori &  Staff \\\hline
Descrizione & Permette di recuperare la lista di tutti i servizi esistenti.\\\hline
Input &  - \\\hline
Output &  Tutti i servizi esistenti\\\hline
Pre-condizioni &  Nessuna.\\\hline
Post-condizioni &  Nessuna\\\hline
Flusso &  	\vspace{-5mm} \begin{enumerate}
			\item Il sistema effettua una richiesta al sistema Database per richiedere tutti i servizi
			\item viene restituita la lista dei servizi trovati
		\end{enumerate}\\\hline
\end{tabular}
\label{table_use_case:\lastUC}\newline
\end{table}

\begin{table}[bp]
    \centering
    \addtolength{\leftskip} {-2cm}
\begin{tabular}{ |p{2.6cm}|p{13cm}|  }
\hline
ID & UC\_\nextUC \\\hline
Categoria & Management\\\hline
Nome & UC\_DisattivaAbbonamento\\\hline
Priorità & High \\\hline
Attori &  ManagerAbbonamenti \\\hline
Descrizione & Rende non più sottoscrivibile un piano di abbonamento esistente.\\\hline
Pre-condizioni &  Il manager ha indicato la disattivazione di uno specifico piano di abbonamento\\\hline
Post-condizioni &  Il piano di abbonamento specificato non è più sottoscrivibile\\\hline
Flusso &  	\vspace{-5mm} \begin{enumerate}
		\item Il sistema richiede al Database di salvare che il piano di abbonamento specificato non è più sottoscrivibile\newline
		\end{enumerate}\\\hline
\end{tabular}
\label{table_use_case:\lastUC}\newline
\end{table}

\begin{table}[bp]
    \centering
    \addtolength{\leftskip} {-2cm}
\begin{tabular}{ |p{2.6cm}|p{13cm}|  }
\hline
ID & UC\_\nextUC \\\hline
Categoria & Management\\\hline
Nome & UC\_AttivaAbbonamento\\\hline
Priorità & High \\\hline
Attori &  ManagerAbbonamenti \\\hline
Descrizione & Rende sottoscrivibile un piano di abbonamento esistente.\\\hline
Pre-condizioni &  Il manager ha indicato l'attivazione di uno specifico piano di abbonamento\\\hline
Post-condizioni &  Il piano di abbonamento specificato è sottoscrivibile\\\hline
Flusso &  	\vspace{-5mm} \begin{enumerate}
		\item Il sistema richiede al Database di salvare che il piano di abbonamento specificato è sottoscrivibile\newline
		\end{enumerate}\\\hline
\end{tabular}
\label{table_use_case:\lastUC}\newline
\end{table}


\begin{table}[bp]
    \centering
    \addtolength{\leftskip} {-2cm}
\begin{tabular}{ |p{2.6cm}|p{13cm}|  }
\hline
ID & UC\_\nextUC \\\hline
Categoria & Management\\\hline
Nome & UC\_AggiungiServizioAbbonamento\\\hline
Priorità & High \\\hline
Attori &  ManagerAbbonamenti \\\hline
Descrizione & Permette di aggiungere un servizio all'insieme dei servizi disponibili in un piano di abbonamento.\\\hline
Pre-condizioni & Il manager ha indicato uno specifico piano di abbonamento al quale vuole aggiungere un servizio\\\hline
Post-condizioni & Un nuovo servizio viene aggiunto al piano di abbonamento indicato\\\hline
Flusso &  	\vspace{-5mm} \begin{enumerate}
		\item Il sistema mostra la lista dei servizi che possono essere aggiunti al piano di abbonamento indicato		
			\begin{enumerate}[label*=\arabic*.]
			\item (UC\_3) Il sistema recupera tutti i servizi disponibili
			\item (UC\_8) Il sistema recupera tutti i servizi associati al piano di abbonamento indicato
			\item La lista dei servizi sarà la differenza tra il primo e il secondo insieme calcolati
			\end{enumerate}
		\item Il manager degli abbonamenti sceglie dalla lista, il servizio da aggiungere al piano di abbonamento
		\item Il sistema richiede al Database di aggiungere il servizio scelto alla lista dei servizi per il piano di abbonamento indicato
		\end{enumerate}\\\hline
\end{tabular}
\label{table_use_case:\lastUC}\newline
\end{table}

\begin{table}[bp]
    \centering
    \addtolength{\leftskip} {-2cm}
\begin{tabular}{ |p{2.6cm}|p{13cm}|  }
\hline
ID & UC\_\nextUC \\\hline
Categoria & Management\\\hline
Nome & UC\_RimuoviServizioAbbonamento\\\hline
Priorità & High \\\hline
Attori &  ManagerAbbonamenti \\\hline
Descrizione & Permette la rimozione di uno dei servizi forniti da un piano abbonamento.\\\hline
Pre-condizioni &  Il manager ha indicato uno specifico piano di abbonamento al quale vuole rimuovere un servizio \\\hline
Post-condizioni &  Il servizio scelto non \'e pi\'u associato al piano di abbonamento indicato\\\hline
Flusso &  	\vspace{-5mm} \begin{enumerate}
		\item Il sistema mostra la lista dei servizi che possono essere rimossi dal piano di abbonamento indicato
			\begin{enumerate}[label*=\arabic*.]
			\item (UC\_8) Il sistema recupera tutti i servizi associati al piano di abbonamento indicato
			\end{enumerate}
		\item Il manager degli abbonamenti sceglie dalla lista, il servizio da rimuovere
		\item Il sistema richiede al Database di rimuovere il servizio scelto alla lista dei servizi per il piano di abbonamento indicato
		\end{enumerate}\\\hline
\end{tabular}
\label{table_use_case:\lastUC}\newline
\end{table}

\begin{table}[bp]
    \centering
    \addtolength{\leftskip} {-2cm}
\begin{tabular}{ |p{2.6cm}|p{13cm}|  }
\hline
ID & UC\_\nextUC \\\hline
Categoria & Management\\\hline
Nome & UC\_RecuperaServiziAbbonamento\\\hline
Priorità & High \\\hline
Attori &  Staff \\\hline
Descrizione & Permette, dato un piano di abbonamento, di reperire tutti i servizi associati ad esso.\\\hline
Input &  Un piano di abbonamento \\\hline
Output &  I servizi associati al piano di abbonamento fornito\\\hline
Pre-condizioni &  Nessuna\\\hline
Post-condizioni &  Nessuna \\\hline
Flusso &  	\vspace{-5mm} \begin{enumerate}
			\item Il sistema effettua una richiesta al sistema Database per richiedere tutti i servizi associati al piano di abbonamento indicato
			\item Viene restituita la lista dei servizi trovata
		\end{enumerate}\\\hline
\end{tabular}
\label{table_use_case:\lastUC}\newline
\end{table}

\begin{table}[bp]
    \centering
    \addtolength{\leftskip} {-2cm}
\begin{tabular}{ |p{2.6cm}|p{13cm}|  }
\hline
ID & UC\_\nextUC \\\hline
Categoria & Management\\\hline
Nome & UC\_RecuperaPianiAbbonamentoUtente\\\hline
Priorità & High \\\hline
Attori &  Staff \\\hline
Descrizione & Permette di recuperare la lista dei piani di abbonamento sottoscritti da un utente.\\\hline
Input &  Un utente \\\hline
Output &  La lista dei piani di abbonamento sottoscritti dall'utente\\\hline
Pre-condizioni &  Nessuna\\\hline
Post-condizioni &  Nessuna\\\hline
Flusso &  	\vspace{-5mm} \begin{enumerate}	
		\item Il sistema effettua una richiesta al sistema Database per richiedere tutti i piani di abbonamento sottoscritti dall'utente indicato
		\item Viene restituita la lista dei piani di abbonamento trovata
		\end{enumerate}\\\hline
\end{tabular}
\label{table_use_case:\lastUC}\newline
\end{table}
%==============================


%======= UC relativi a 2 e 3 ========
\begin{table}[bp]
    \centering
    \addtolength{\leftskip} {-2cm}
\begin{tabular}{ |p{2.6cm}|p{13cm}|  }
\hline
ID & UC\_\nextUC \\\hline
Categoria & Management\\\hline
Nome & UC\_EffettuaPagamentoPartner\\\hline
Priorità & High \\\hline
Attori &  Time \\\hline
Descrizione & Effettua i pagamenti verso gli utente che hanno il servizio di pubblicare prodotti\\\hline
Pre-condizioni &  Nessuno\\\hline
Post-condizioni &  Nessuno\\\hline
Flusso &  	
		\vspace{-5mm} \begin{enumerate}	
		\item Il sistema, alla fine di ogni mese, recupera la lista degli utenti che possono ricevere un pagamento
			\begin{enumerate}[  ]
			\item Il sistema recupera tutti gli utenti
			\item (UC\_9) per ogni utente il sistema recupera i suoi piani di abbonamento e da questi (UC\_8) recupera i servizi dell'utente
			\item Vengono restituiti gli utenti che hanno il servizio di pubblicare prodotti e il cui account è attivo
			\end{enumerate}
		\item Per ogni utente individuato:
		\begin{enumerate}[label*=\arabic*.]
			\item Il sistema individua il sistema di pagamento esterno adatto ad effettuare il pagamento (sulla base dei metodi di pagamento forniti dall'utente)
			\item (UC\_11) Il sistema calcola l'importo da pagare all'utente
			\begin{enumerate}[label*=\arabic*.]
				\item \textbf{\# alt1}: Se UC\_11 non produce errori, allora il sistema richiede al sistema di pagamento di effettuare un pagamento all'utente (con i dati inseriti dall'utente e l'importo calcolato dal sistema)
				\begin{enumerate}[label*=\arabic*.]
					\item \textbf{\# alt1}: Se il pagamento va a buon fine: viene aggiornato l'ultimo giorno di pagamento per l'utente, viene inviata una mail all'utente comunicandogli che il pagamento è stato effettuato
					\item \textbf{\# alt2}: Se il pagamento non va a buon fine: viene inviata una mail all'utente comunicandogli che un tentativo di pagamento non è andato a buon fine
				\end{enumerate}
				\item \textbf{\# alt2}: Se UC\_11 produce errori, viene inviata una mail all'utente comunicandogli che un tentativo di pagamento non è andato a buon fine
			\end{enumerate}
		\end{enumerate}
		\end{enumerate}\\\hline
\end{tabular}
\label{table_use_case:\lastUC}\newline
\end{table}

\begin{table}[bp]
    \centering
    \addtolength{\leftskip} {-2cm}
\begin{tabular}{ |p{2.6cm}|p{13cm}|  }
\hline
ID & UC\_\nextUC \\\hline
Categoria & Management\\\hline
Nome & UC\_CalcolaImportoDaPagare\\\hline
Priorità & High \\\hline
Attori &  Time \\\hline
Descrizione & Calcola l'importo da pagare a un utente per il prossimo pagamento\\\hline
Input & L'utente per cui si vuole calcolare l'importo\\\hline
Output & L'importo da pagare all'utente\\\hline
Pre-condizioni &  L'utente ha il servizio che permette di pubblicare prodotti e l'account dell'utente è attivo\\\hline
Post-condizioni &  Nessuna\\\hline
Flusso &  	
		\vspace{-5mm} \begin{enumerate}
		\item Il sistema ottiene l'ultimo istante in cui l'utente ha ricevuto un pagamento, e considera l'intervallo:\newline $[max(adesso - 1mese, istante\ ultimo\ pagamento), adesso]$ (dove adesso indica l'istante corrente, quando viene eseguito l'use-case)
		\item Il sistema effettua una richiesta al Database per calcolare il numero di visualizzazioni complessive ricevute dai prodotti dell'utente nell'intervallo di tempo considerato
		\item il sistema restituisce f(numero visualizzazioni complessive), dove f è una funzione che indica come calcolare l'importo da pagare in funzione del numero di visualizzazioni
		\end{enumerate}\\\hline
\end{tabular}
\label{table_use_case:\lastUC}\newline
\end{table}

\begin{table}[bp]
    \centering
    \addtolength{\leftskip} {-2cm}
\begin{tabular}{ |p{2.6cm}|p{13cm}|  }
\hline
ID & UC\_\nextUC \\\hline
Categoria & Management\\\hline
Nome & UC\_SospendiAccount\\\hline
Priorità & High \\\hline
Attori &  ManagerSegnalazioni \\\hline
Descrizione & Sospende l'account di un utente\\\hline
Pre-condizioni &  Il manager ha indicato un utente che vuole sospendere\\\hline
Post-condizioni &  L'account dell'utente specificato è sospeso\\\hline
Flusso &  	
		\vspace{-5mm} \begin{enumerate}
		\item Il sistema richiede al Database di memorizzare che l'account dell'utente specificato è sospeso\newline
		\end{enumerate}\\\hline
\end{tabular}
\label{table_use_case:\lastUC}\newline
\end{table}
%=======================


%======= UC relativi a 4 ===========
\begin{table}[bp]
    \centering
    \addtolength{\leftskip} {-2cm}
\begin{tabular}{ |p{2.6cm}|p{13cm}|  }
\hline
ID & UC\_\nextUC \\\hline
Categoria & Autenticazione \\\hline
Nome & UC\_EffettuaRegistrazione \\\hline
Priorità & High \\\hline
Attori &  UtenteNonAutenticato \\\hline
Descrizione & Permette a un utente non autenticato di registrarsi\\\hline
Pre-condizioni &  Nessuna\\\hline
Post-condizioni &  Viene creato un nuovo account\\\hline
Flusso &  	\vspace{-5mm} \begin{enumerate}
			\item L'utente richiede di registrarsi alla piattaforma;
			\item Il sistema richiede all' utente non registrato informazioni quali: anagrafiche, email, email di recupero, password, conferma della password;
			\item L'utente non registrato riempie i campi richiesti e li invia al sistema;
			\item Il sistema verifica la validità dei dati:
			\begin{enumerate}[  ]
				\item\textbf{\# alt1}: Se il sistema accetta i dati, il sistema richiede al Database di registrare l'utente sulla piattaforma
				\item\textbf{\# alt2}: Se il sistema non accetta i dati, la procedura fallisce e l'utente viene avvisato
			\end{enumerate}
		\end{enumerate}\\\hline
\end{tabular}
\label{table_use_case:\lastUC}\newline
\end{table}

\begin{table}[bp]
    \centering
    \addtolength{\leftskip} {-2cm}
\begin{tabular}{ |p{2.6cm}|p{13cm}|  }
\hline
ID & UC\_\nextUC\\\hline
Categoria & Autenticazione \\\hline
Nome & UC\_ModificaProfilo \\\hline
Priorità & Medium \\\hline
Attori &  UtenteAutenticato \\\hline
Descrizione & Permette la modifica da parte di un utente di alcuni dati inseriti in fase di registrazione relativi all'account dell'utente stesso  \\\hline
Pre-condizioni & Nessuna\\\hline
Post-condizioni & Le informazioni relative all'utente sono aggiornate secondo i nuovi valori forniti dall'utente.\\\hline
Flusso &  	\vspace{-5mm} \begin{enumerate}
			\item L'utente richiede la modifica dei suoi dati
			\item Il sistema fornisce i valori attualmente memorizzati sul sistema relativi ai dati modificabili, facendo una richiesta al Database
			\item L'utente comunica al sistema i nuovi valori da memorizzare
			\item Il sistema valida i dati proposti dall'utente:
			\begin{enumerate}[  ]
				\item\textbf{\# alt1}: Se i nuovi valori sono validi, il sistema richiede al Database di aggiornare i dati dell'utente con quelli inseriti
				\item\textbf{\# alt}: Se i nuovi valori non sono validi, il salvataggio viene bloccato, l'utente viene avvisato e invitato a fornire dei valori validi
			\end{enumerate}
		\end{enumerate}\\\hline
\end{tabular}
\label{table_use_case:\lastUC}\newline
\end{table}

\begin{table}[bp]
    \centering
    \addtolength{\leftskip} {-2cm}
\begin{tabular}{ |p{2.6cm}|p{13cm}|  }
\hline
ID & UC\_\nextUC \\\hline
Categoria & Autenticazione \\\hline
Nome & UC\_EffettuaLogin \\\hline
Priorità & High \\\hline
Attori &  UtenteNonAutenticato \\\hline
Descrizione & Permette ad un utente registrato di accedere al proprio profilo fornendo le opportune credenziali \\\hline
Pre-condizioni & Nessuna\\\hline
Post-condizioni & L'utente è autenticato\\\hline
Flusso &  	\vspace{-5mm} \begin{enumerate}
			\item L'utente richiede di accedere alla piattaforma
			\item Il sistema richiede all' utente le credenziali di accesso quali: email e password
			\item L'utente riempie i campi richiesti e li invia al sistema
			\item Il sistema valida i dati e avvia la procedura di autenticazione interfacciandosi con il Database:
			\begin{enumerate}[  ]
				\item \textbf{\# alt1}: se le credenziali sono corrette: l'utente è autenticato all'interno del proprio account
				\item \textbf{\# alt2}: se le credenziali non sono corrette: l'utente non viene autenticato ed è reindirizzato alla schermata di login
			\end{enumerate}
		\end{enumerate}\\\hline
\end{tabular}
\label{table_use_case:\lastUC}\newline
\end{table}

\begin{table}[bp]
    \centering
    \addtolength{\leftskip} {-2cm}
\begin{tabular}{ |p{2.6cm}|p{13cm}|  }
\hline
ID & UC\_\nextUC\\\hline
Categoria & Autenticazione \\\hline
Nome & UC\_EffettuaLogout \\\hline
Priorità & High \\\hline
Attori &  UtenteAutenticato \\\hline
Descrizione & Permette ad un utente autenticato di uscire dal proprio profilo \\\hline
Pre-condizioni &  Nessuna\\\hline
Post-condizioni &  L'utente non è più autenticato.\\\hline
Flusso &  	\vspace{-5mm} \begin{enumerate}
			\item L'utente richiede di uscire dalla piattaforma
			\item L'utente non è più autenticato sulla piattaforma e viene reindirizzato alla schermata principale
			%\item Il sistema elimina la sessione dell'utente
		\end{enumerate}\\\hline
\end{tabular}
\label{table_use_case:\lastUC}\newline
\end{table}
%==============================

%============ UC relativi a 6 =============
\begin{table}[bp]
    \centering
    \addtolength{\leftskip} {-2cm}
\begin{tabular}{ |p{2.6cm}|p{13cm}|  }
\hline
ID & UC\_\nextUC \\\hline
Categoria & Risorse\\\hline
Nome & UC\_SottoscriviAbbonamento\\\hline
Priorità & High \\\hline
Attori &  UtenteAutenticato \\\hline
Descrizione & Permette a un utente di sottoscrivere un nuovo abbonamento con la piattaforma.\\\hline
Pre-condizioni & L'utente non ha sottoscritto tutti i i piani di abbonamento\\\hline
Post-condizioni &  L'utente ha accesso a tutti i servizi offerti dal nuovo abbonamento sottoscritto.\\\hline
Flusso &  	\vspace{-5mm} \begin{enumerate}
			\item L'utente richiede di sottoscrivere un piano di abbonamento;
			\item Il sistema fornisce una lista dei piani di abbonamento sottoscrivibili e non ancora sottoscritti dall'utente
				\begin{enumerate}[  ]
				\item (UC\_2): il sistema recupera i piani di abbonamento esistenti
				\item (UC\_9): il sistema recupera i piani di abbonamento sottoscritti da un utente
				\item vengono forniti nella lista solo i piani di abbonamento sottoscrivibili che non sono sottoscritti dall'utente
				\end{enumerate}
			\item L'utente sceglie il piano di abbonamento
			\item Il sistema reinderizza l'utente al sistema esterno di pagamento (sulla base del metodo di pagamento scelto in fase di registrazione), e attende risposta dal sistema esterno
			\begin{enumerate}[  ]
				\item \textbf{\# alt1}: Se il pagamento va a buon fine: il sistema aggiunge il piano di abbonamento alla lista degli abbonamenti attivi dell'utente, salvando il timestamp in cui il piano è stato attivato e attivando inizialmente il rinnovo automatico
				\item \textbf{\# alt2}: Se il pagamento non va a buon fine: il sistema comunica l'errore all'utente e la procedura termina
			\end{enumerate}
		\end{enumerate}\\\hline
\end{tabular}
\label{table_use_case:\lastUC}\newline
\end{table}

\begin{table}[bp]
    \centering
    \addtolength{\leftskip} {-2cm}
\begin{tabular}{ |p{2.6cm}|p{13cm}|  }
\hline
ID & UC\_\nextUC \\\hline
Categoria & Risorse\\\hline
Nome & UC\_DisdiciAbbonamento\\\hline
Priorità & High \\\hline
Attori &  UtenteAutenticato \\\hline
Descrizione & Permette a un utente di disdire un abbonamento sottoscritto, evitando che questo si rinnovi automaticamente nella data prevista\\\hline
Pre-condizioni & L'utente ha indicato un piano di abbonamento sottoscritto che intende disdire\\\hline
Post-condizioni &  Il rinnovo automatico per il piano di abbonamento indicato è disattivato\\\hline
Flusso &  	\vspace{-5mm} \begin{enumerate}
			\item Il sistema richiede al Database di memorizzare l'annullamento del rinnovo automatico per il piano di abbonamento indicato\newline
		\end{enumerate}\\\hline
\end{tabular}
\label{table_use_case:\lastUC}\newline
\end{table}

\begin{table}[bp]
    \centering
    \addtolength{\leftskip} {-2cm}
\begin{tabular}{ |p{2.6cm}|p{13cm}|  }
\hline
ID & UC\_\nextUC \\\hline
Categoria & Risorse\\\hline
Nome & UC\_CambiaAbbonamento\\\hline
Priorità & High \\\hline
Attori &  UtenteAutenticato \\\hline
Descrizione & Permette a un utente di sottoscrivere un nuovo abbonamento in sostituzione di uno gi\'a attivo.\\\hline
Pre-condizioni & L'utente ha indicato un piano di abbonamento sottoscritto che intende cambiare\\\hline
Post-condizioni &  L'utente ha sottoscritto il nuovo abbonamento e non possiede pi\'u il precedente.\\\hline
Flusso &  	\vspace{-5mm} \begin{enumerate}
			\item Il sistema fornisce una lista dei piani di abbonamento sottoscrivibili, in sostituzione a quello indicato:
			\begin{enumerate}[  ]
				\item Il sistema ottiene la lista di tutti i piani di abbonamento sottoscrivibili (UC\_2)
				\item Il sistema ottiene la lista di tutti i piani sottoscritti dall'utente (UC\_9)
				\item Il sistema fornisce solo i piani di abbonamento sottoscrivibili non ancora sottoscritti dall'utente
			\end{enumerate}
			\item L'utente sceglie il nuovo abbonamento da sottoscrivere;
			\item Il sistema controlla le politiche di cambio:
			\begin{enumerate}[  ]
				\item \textbf{\# alt1}: Se le politiche permettono il cambio di abbonamento:
				\begin{enumerate}[label*=\arabic*.]
					\item \textbf{\# alt1}: Se il prezzo del nuovo abbonamento è minore o uguale al vecchio abbonamento, il cambio verra effettuato immediatamente
					\item \textbf{\# alt2}: Altrimenti, il sistema reinderizza l'utente al sistema esterno di pagamento per far pagare all'utente la differenza tra i due piano di abbonamento:
						\begin{enumerate}[label*=\arabic*.]
							\item \textbf{\# alt1}: Se il pagamento va a buon fine: viene effettuato il cambio, rimuovendo il vecchio piano di abbonamento e sottoscrivendo il nuovo piano di abbonamento
							\item \textbf{\# alt2}: Se il pagamento non va a buon fine: il sistema comunica l'errore all'utente e la procedura termina
						\end{enumerate}
				\end{enumerate}
				\item \textbf{\# alt2}: Altrimenti, l'operazione viene annullata e l'utente viene informato che non è possibile effettuare il cambio
			\end{enumerate}
			\end{enumerate}\\\hline
\end{tabular}
\label{table_use_case:\lastUC}\newline
\end{table}

%========================



%======= UC relativi a 7 ===========
\begin{table}[bp]
    \centering
    \addtolength{\leftskip} {-2cm}
\begin{tabular}{ |p{2.6cm}|p{13cm}|  }
\hline
ID & UC\_\nextUC\\\hline
Categoria & Prodotto \\\hline
Nome & UC\_CreaProdotto \\\hline
Priorità & High \\\hline
Attori &  UtenteAutenticato \\\hline
Descrizione & Permette ad un utente autenticato di caricare un nuovo prodotto \\\hline
Pre-condizioni &  L'utente autenticato ha il servizio che permette di pubblicare prodotti\\\hline
Post-condizioni &  Viene creato un nuovo prodotto con proprietario l'utente autenticato\\\hline
Flusso &  	\vspace{-5mm} \begin{enumerate}
			\item L'utente richiede di creare un nuovo prodotto
			\item Il sistema chiede all'utente di fornire il titolo del prodotto e il tipo di prodotto (prodotto video o prodotto musicale)
			\item L'utente fornisce le informazioni richieste
			\item Il sistema verifica che non esiste gi\'a un prodotto dello stesso tipo con il nome scelto dall'utente:
			\begin{enumerate}[  ]
				\item \textbf{\# alt1}: Se non esiste il sistema lo crea
					\begin{enumerate}[label*=\arabic*.]
						\item La visibilità del prodotto è inizialmente impostata come "privato"
						\item Il proprietario del prodotto è l'utente che ha effettuato la richiesta di creazione
					\end{enumerate}
				\item \textbf{\# alt2}: Se esiste l'operazione viene annullata e l'utente viene avvisato
			\end{enumerate}
		\end{enumerate}\\\hline
\end{tabular}
\label{table_use_case:\lastUC}\newline
\end{table}

\begin{table}[bp]
    \centering
    \addtolength{\leftskip} {-2cm}
\begin{tabular}{ |p{2.6cm}|p{13cm}|  }
\hline
ID & UC\_\nextUC\\\hline
Categoria & Prodotto \\\hline
Nome & UC\_ModificaInformazioniDiBase \\\hline
Priorità & High \\\hline
Attori &  UtenteAutenticato \\\hline
Descrizione & Permette ad un utente autenticato di inserire informazioni di base su un prodotto \\\hline
Pre-condizioni &  L'utente autenticato ha il servizio che permette di pubblicare prodotti, l'utente ha selezionato il prodotto da modificare\\\hline
Post-condizioni &  I campi base del prodotto sono compilati\\\hline
Flusso &  	\vspace{-5mm} \begin{enumerate}
			\item Il sistema richiede all'utente di inserire descrizione del prodotto, genere del prodotto (tra una lista di generi disponibili)
			\item L'utente inserisce i dati per il prodotto
			\item Il sistema valida i dati e:
			\begin{enumerate}[  ]
				\item \textbf{\# alt1}: Se non si verificano errori: i dati vengono correttamente memorizzati, e:
				\begin{enumerate}[label*=\arabic*.]
					\item Il sistema mostra all'utente un questionario a domande chiuse, atto a calcolare l'età minima per visionare il prodotto
					\item L'utente compila il questionario
					\item Sulla base delle risposte dell'utente, il sistema calcola l'età minima per visionare il prodotto
				\end{enumerate}
				\item \textbf{\# alt2}: Se si verificano errori, viene informato l'utente e viene riportato alla schermata per ricompilare i campi
			\end{enumerate}
		\end{enumerate}\\\hline
\end{tabular}
\label{table_use_case:\lastUC}\newline
\end{table}

\begin{table}[bp]
    \centering
    \addtolength{\leftskip} {-2cm}
\begin{tabular}{ |p{2.6cm}|p{13cm}|  }
\hline
ID & UC\_\nextUC\\\hline
Categoria & Prodotto \\\hline
Nome & UC\_CaricaFile \\\hline
Priorità & High \\\hline
Attori &  UtenteAutenticato \\\hline
Descrizione & Viene completata la creazione di un prodotto, che è stato scelto come prodotto video \\\hline
Pre-condizioni &  L'utente autenticato ha il servizio che permette di pubblicare prodotti, l'utente ha selezionato il prodotto sul quale caricare i/il file \\\hline
Post-condizioni & I file multimediali relativi al prodotto sono pubblicati \\\hline
Flusso &  	\vspace{-5mm}
			\textbf{\# alt1}:Se il prodotto \'e di tipo Video allora: 
				\begin{enumerate}[label*=\arabic*.]
					\item Il sistema richiede di caricare un file video (obbligatorio), almeno un file audio e per ogni file audio deve essere associata una lingua (scelta da una lista di lingue conosciute), un insieme di file dei sottotitoli (opzionali, ma per ogni file dei sottotitoli deve essere associata la lingua)
					\item L'utente carica i file sulla piattaforma
					\item Il sistema carica i file sulla CDN esterna e li associa al prodotto creato dall'utente
				\end{enumerate}
			\textbf{\# alt2}:Se il prodotto \'e di tipo Musicale allora: 
				\begin{enumerate}[label*=\arabic*.]
					\item Il sistema richiede un file audio obbligatorio, un file dei lyrics (opzionale), e un file del video musicale (opzionale)
					\item L'utente carica i file sulla piattaforma
					\item Il sistema carica i file sulla CDN esterna e li associa al prodotto creato dall'utente
				\end{enumerate}
\end{tabular}
\label{table_use_case:\lastUC}\newline
\end{table}

\begin{table}[bp]
    \centering
    \addtolength{\leftskip} {-2cm}
\begin{tabular}{ |p{2.6cm}|p{13cm}|  }
\hline
ID & UC\_\nextUC\\\hline
Categoria & Prodotto \\\hline
Nome & UC\_CambiaStatoPubblicazione \\\hline
Priorità & High \\\hline
Attori &  UtenteAutenticato \\\hline
Descrizione & Permette a un utente autenticato di cambiare lo stato di pubblicazione di uno dei suoi prodotti\\\hline
Pre-condizioni &  L'utente deve avere il servizio per la pubblicazione di prodotti, l'utente ha selezionato la pubblicazione tramite la pagina delle pubblicazioni o tramite la pagina di dettaglio della pubblicazione.\\\hline
Post-condizioni &  Lo stato di pubblicazione del prodotto viene cambiato con quello scelto dall'utente\\\hline
Flusso &  	\vspace{-5mm} \begin{enumerate}
			\item Il sistema richiede all'utente il nuovo stato di pubblicazione per il prodotto
			\item L'utente sceglie il nuovo stato di pubblicazione tra "privato", "pubblico", e:
				\begin{enumerate}[  ]
					\item \textbf{\# alt1:} Se l'utente sceglie privato: il sistema cambia lo stato di pubblicazione in "privato" e il prodotto diventa accessibile solamente all'utente stesso
					\item \textbf{\# alt2:} Se l'utente sceglie pubblico: il sistema cambia lo stato di pubblicazione in "pubblico" e il prodotto diventa accessibile a tutti gli utenti che hanno il servizio adatto a visionare quel tipo di prodotto
				\end{enumerate}
		\end{enumerate}\\\hline
\end{tabular}
\label{table_use_case:\lastUC}\newline
\end{table}
%====================


%============ UC relativi a 8 ============= 
\begin{table}[bp]
    \centering
    \addtolength{\leftskip} {-2cm}
\begin{tabular}{ |p{2.6cm}|p{13cm}|  }
\hline
ID & UC\_\nextUC \\\hline
Categoria & Risorse\\\hline
Nome & UC\_RiproduciVideo\\\hline
Priorità & High \\\hline
Attori &  UtenteAutenticato \\\hline
Descrizione & Permette di riprodurre un prodotto video.\\\hline
Pre-condizioni & L'utente ha il servizio che permette la riproduzione video e ha fornito uno specifico prodotto video che intende riprodurre\\\hline
Post-condizioni & Nessuna\\\hline
Flusso &  	\vspace{-5mm} \begin{enumerate}
			\item Il sistema fornisce al client le informazioni sul nodo della CDN dal quale reperire il prodotto
			\item Il client contatta il nodo e stabiliscono una connessione
			\item Inizia lo stream del prodotto
			\item I pacchetti ricevuti dalla CDN vengono riprodotti dal player
			\end{enumerate}
			\\\hline
\end{tabular}
\label{table_use_case:\lastUC}\newline
\end{table}

\begin{table}[bp]
    \centering
    \addtolength{\leftskip} {-2cm}
\begin{tabular}{ |p{2.6cm}|p{13cm}|  }
\hline
ID & UC\_\nextUC \\\hline
Categoria & Risorse\\\hline
Nome & UC\_RiproduciMusica\\\hline
Priorità & High \\\hline
Attori &  UtenteAutenticato \\\hline
Descrizione & riproduce un prodotto musicale, con eventuale traccia video adeguata (lyrics o video musicale).\\\hline
Pre-condizioni & L'utente ha il servizio che permette la riproduzione di musica e  ha fornito uno specifico prodotto musicale che intende riprodurre\\\hline
Post-condizioni & Nessuna\\\hline
Flusso &  	\vspace{-5mm} \begin{enumerate}
			\item Il sistema fornisce al client le informazioni sul nodo della CDN dal quale reperire il prodotto
			\item Il client contatta il nodo e stabiliscono una connessione
			\begin{enumerate}[label*=\arabic*.]
				\item \textbf{\# alt1}: Se il prodotto ha associato un file di lyrics: verr\'a riprodotto il video lyrics insieme al file audio;
				\item \textbf{\# alt2}: Altrimenti, se il prodotto non ha associato un file lyrics, ma ha associato un file video musicale: verr\'a riprodotto il video musicale insieme al file audio
				\item \textbf{\# alt3}: Altrimenti, se il prodotto non ha assocaito nè un file lyrics nè un file video musicale: verr\'a riprodotta solo la traccia audio
			\end{enumerate}
			\item Inizia lo stream del file audio e dell'eventuale file video
			\item I pacchetti ricevuti dalla CDN vengono riprodotti dal player
			\end{enumerate}
			\\\hline
\end{tabular}
\label{table_use_case:\lastUC}\newline
\end{table}

\begin{table}[bp]
    \centering
    \addtolength{\leftskip} {-2cm}
\begin{tabular}{ |p{2.6cm}|p{13cm}|  }
\hline
ID & UC\_\nextUC \\\hline
Categoria & Risorse\\\hline
Nome & UC\_PausaPlayer\\\hline
Priorità & High \\\hline
Attori &  UtenteAutenticato \\\hline
Descrizione & Mette in pausa la visualizzazione del prodotto.\\\hline
Pre-condizioni & L'utente sta riproducento un prodotto.\\\hline
Post-condizioni & Nessuna\\\hline
Flusso &  	\vspace{-5mm} \begin{enumerate}
			\item L'utente richiede la pausa di riproduzione al player
			\item Il player ferma la riproduzione del prodotto, ma continua a ricevere pacchetti dalla CDN (fino ad un'eventuale saturazione del buffer)
			\end{enumerate}
			\\\hline
\end{tabular}
\label{table_use_case:\lastUC}\newline
\end{table}

\begin{table}[bp]
    \centering
    \addtolength{\leftskip} {-2cm}
\begin{tabular}{ |p{2.6cm}|p{13cm}|  }
\hline
ID & UC\_\nextUC \\\hline
Categoria & Risorse\\\hline
Nome & UC\_SpostaPuntoRiproduzionePlayer\\\hline
Priorità & High \\\hline
Attori &  UtenteAutenticato \\\hline
Descrizione & Cambia il punto di riproduzione del prodotto (pu\'o essere spostata avanti o indietro).\\\hline
Pre-condizioni & L'utente sta riproducento un prodotto.\\\hline
Post-condizioni & La riproduzione continua dall'istante selezionato\\\hline
Flusso &  	\vspace{-5mm} \begin{enumerate}
			\item L'utente seleziona lo spostamento di riproduzione in un certo punto compreso tra 0 e lunghezza del prodotto
			\item Il client comunica al nodo della CDN che sta fornendo il prodotto il nuovo istante di tempo dal quale proseguire
			\item Il player riprende la riproduzione dal punto selezionato (eventualmente usando i pacchetti che aveva già ricevuto dalla CDN, oppure aspettando i nuovi pacchetti)
		\end{enumerate}
			\\\hline
\end{tabular}
\label{table_use_case:\lastUC}\newline
\end{table}
%====================


%============ UC relativi a 13 ============= 
\begin{table}[bp]
    \centering
    \addtolength{\leftskip} {-2cm}
\begin{tabular}{ |p{2.6cm}|p{13cm}|  }
\hline
ID & UC\_\nextUC \\\hline
Categoria & Prodotto\\\hline
Nome & UC\_SegnalaProdotto\\\hline
Priorità & High \\\hline
Attori &  UtenteAutenticato, Staff \\\hline
Descrizione & Permette di segnalare un prodotto (con relativa motivazione della segnalazione).\\\hline
Pre-condizioni & L'utente ha indicato il prodotto che intende segnalare\\\hline
Post-condizioni & Il prodotto selezionato ha una segnalazione aperta\\\hline
Flusso &  	\vspace{-5mm} \begin{enumerate}
			\item Il sistema richiede la motivazione della segnalazione
			\item L'utente comunica una motivazione.
			\item Il sistema provvede ad aprire la segnalazione
			\item Viene memorizzato in modo persistente l'utente segnalatore, il prodotto segnalato, la motivazione e l'istante in cui è stata sottomessa.
			\item L'utente viene avvisato dell'esito della sengalazione
			\begin{enumerate}[label= ]
				\item \textbf{\# alt1}: Se il la segnalazione è andata a buon fine, l'utente riceverà un responso positivo
				\item \textbf{\# alt2}: Se il la segnalazione non è andata a buon fine, l'utente riceverà un responso negativo
			\end{enumerate}
			\end{enumerate}
			\\\hline
\end{tabular}
\label{table_use_case:\lastUC}\newline
\end{table}

\begin{table}[bp]
    \centering
    \addtolength{\leftskip} {-2cm}
\begin{tabular}{ |p{2.6cm}|p{13cm}|  }
\hline
ID & UC\_\nextUC \\\hline
Categoria & Prodotto\\\hline
Nome & UC\_OttieniSegnalazioni\\\hline
Priorità & High \\\hline
Attori &  Staff \\\hline
Descrizione & Permette ai membri dello staff di ottenere le segnalazioni.\\\hline
Pre-condizioni & Nessuna\\\hline
Post-condizioni & Nessuna\\\hline
Flusso &  	\vspace{-5mm} \begin{enumerate}
			\item L'operatore dello staff richiede al sistema la lista delle segnalazioni.
			\item Il sistema ricerca le segnalazioni e fornisce la lista di segnalazioni all'operatore richiedente.
			\end{enumerate}
			\\\hline
\end{tabular}
\label{table_use_case:\lastUC}\newline
\end{table}

\begin{table}[bp]
    \centering
    \addtolength{\leftskip} {-2cm}
\begin{tabular}{ |p{2.6cm}|p{13cm}|  }
\hline
ID & UC\_\nextUC \\\hline
Categoria & Prodotto\\\hline
Nome & UC\_ChiudiSegnalazione\\\hline
Priorità & High \\\hline
Attori &  ManagerSegnalazioni \\\hline
Descrizione & Permette ai membri dello staff di chiudere una segnalazione.\\\hline
Pre-condizioni & Esiste almeno una segnalazione su un prodotto aperta.\\\hline
Post-condizioni & La segnalazione selezionata è chiusa ed ha un esito\\\hline
Flusso &  	\vspace{-5mm} \begin{enumerate}
			\item L'operatore dello staff richiede al sistema la lista delle segnalazioni secondo UC\_29.
%			POTREBBE ESSERE UN ALTRO USE CASE
%			\item L'operatore potra richiedere di visionare un elemento della lista:
%				\begin{enumerate}
%				\item \textbf{\# alt1}: Se è stato segnalato un profilo utente si potr\'a visionare il dettaglio di esso.
%				\item \textbf{\# alt2}: Se è stata segnalata una risorsa sar\'a visionabile la risorsa in questione.
%				\end{enumerate}
			\item L'operatore seleziona il prodotto da chiudere fornendo un esito.
			\item Il sistema riceve la segnalazione da chiudere, l'esito e chiude la segnalazione: 
			memorizza in maniera persistente l'esito della segnalazione. 
			\end{enumerate}
			\\\hline
\end{tabular}
\label{table_use_case:\lastUC}\newline
\end{table}
%===========================

%================UC relativi a 5==========
\begin{table}[bp]
    \centering
    \addtolength{\leftskip} {-2cm}
\begin{tabular}{ |p{2.6cm}|p{13cm}|  }
\hline
ID & UC\_\nextUC \\\hline
Categoria & Ricerca \\\hline
Nome & UC\_RicercaContenuto\\\hline
Priorità & High \\\hline
Attori &  Utente \\\hline
Descrizione & Cerca contenuti (sia prodotti che playlist) coerenti con una stringa digitata dall'utente\\\hline
Pre-condizioni & Nessuna\\\hline
Post-condizioni & Nessuna\\\hline
Flusso &  	\vspace{-5mm} \begin{enumerate}
			\item L'utente richiede di ricercare un contenuto;
			\item Il sistema richiede all' utente la stringa di ricerca;
			\item L'utente fornisce al sistema la stringa;
			\item Il sistema valida l'input e:
			\begin{enumerate}[label*=\arabic*.]
				\item \textbf{\# alt1:} Se l'input supera la fase di validazione: il sistema effettua una richiesta al Database, richiedendo tutti i prodotti pubblici che contengono la stringa inserita dall'utente nel proprio nome, e tutte le serie tv e album pubblici che contengono la stringa inserita dall'utente nel proprio nome
					\begin{enumerate}[label*=\arabic*.]
						\item \textbf{\# alt1:} Se il Database risponde con successo, il sistema mostra i risultati ottenuti all'utente
						\item \textbf{\# alt2:} Se si verificano errori nella richiesta, la procedura viene annullata e si comunicano gli errori all'utente
					\end{enumerate}
				\item \textbf{\# alt2:} Altrimenti: la procedura viene annullata e viene comunicato all'utente che l'input non è valido
			\end{enumerate}
			
		\end{enumerate}\\\hline
\end{tabular}
\label{table_use_case:\lastUC}\newline
\end{table}

\begin{table}[bp]
    \centering
    \addtolength{\leftskip} {-2cm}
\begin{tabular}{ |p{2.6cm}|p{13cm}|  }
\hline
ID & UC\_\nextUC \\\hline
Categoria & Ricerca \\\hline
Nome & UC\_RicercaPopolari\\\hline
Priorità & Medium \\\hline
Attori &  Utente \\\hline
Descrizione & Cerca contenuti (sia prodotti che playlist) ritenuti popolari\\\hline
Pre-condizioni & Nessuna\\\hline
Post-condizioni &  Nessuna\\\hline
Flusso &  	\vspace{-5mm} \begin{enumerate}
		\item Il sistema fa una richiesta al Database per ottenere i primi 10 prodotti video (non appartententi a serie tv) per numero di visualizzazioni ricevute negli ultimi 7 giorni:
				\begin{enumerate}[label*=\arabic*.]
					\item \textbf{\# alt1:} Se il Database risponde con successo, il sistema mostra i video popolari all'utente
					\item \textbf{\# alt2:} Se si verificano errori nella richiesta, si comunica che non sono disponibili video popolari al momento
				\end{enumerate}
		\item Il sistema fa una richiesta al Database per ottenere i primi 10 prodotti musicali per numero di visualizzazioni ricevute negli ultimi 7 giorni:
				\begin{enumerate}[label*=\arabic*.]
					\item \textbf{\# alt1:} Se il Database risponde con successo, il sistema mostra le canzoni popolari all'utente
					\item \textbf{\# alt2:} Se si verificano errori nella richiesta, si comunica che non sono disponibili musicali popolari al momento
				\end{enumerate}
		\item Il sistema fa una richiesta al Database per ottenere le prime 10 serie tv per numero di visualizzazioni (le visualizzazioni di una serie tv sono la somma delle visualizzazioni dei singoli episodi) ricevute negli ultimi 7 giorni:
				\begin{enumerate}[label*=\arabic*.]
					\item \textbf{\# alt1:} Se il Database risponde con successo, il sistema mostra le serie tv popolari all'utente
					\item \textbf{\# alt2:} Se si verificano errori nella richiesta, si comunica che non sono disponibili serie tv popolari al momento
				\end{enumerate}
		\item Il sistema fa una richiesta al Database per ottenere i primi 10 album per numero di visualizzazioni (le visualizzazioni di un album sono la somma delle visualizzazioni delle singole canzoni) ricevute negli ultimi 7 giorni:
				\begin{enumerate}[label*=\arabic*.]
					\item \textbf{\# alt1:} Se il Database risponde con successo, il sistema mostra gli album popolari all'utente
					\item \textbf{\# alt2:} Se si verificano errori nella richiesta, si comunica che non sono disponibili album popolari al momento
				\end{enumerate}
		\end{enumerate}\\\hline
\end{tabular}
\label{table_use_case:\lastUC}\newline
\end{table}

\begin{table}[bp]
    \centering
    \addtolength{\leftskip} {-2cm}
    \vspace{-4.35cm}
\begin{tabular}{ |p{2.6cm}|p{13cm}|  }
\hline
ID & UC\_\nextUC \\\hline
Categoria & Ricerca \\\hline
Nome & UC\_SuggerisciContenuti\\\hline
Priorità & Medium \\\hline
Attori &  UtenteAutenticato \\\hline
Descrizione & Suggerisce contenuti (sia prodotti che playlist) all'utente sulla base delle ultime visualizzazioni\\\hline
Pre-condizioni &  Nessuna\\\hline
Post-condizioni &  Nessuna\\\hline
Flusso &  	\vspace{-5mm} \begin{enumerate}
		\item Il sistema richiede al Database il genere più visualizzato dall'utente considerando gli ultimi 20 prodotti video riprodotti (o tutti i prodotti video riprodotti se l'utente ha visualizzato meno di 20 prodotti video),:
		\begin{enumerate}[label*=\arabic*.]
			\item \textbf{\# alt1:} Se il Database risponde con successo, e il genere più visualizzato è @genere:
			\begin{enumerate}[label*=\arabic*.]
			\item Il sistema fa una richiesta al Database per ottenere 5 prodotti video (non appartententi a una serie tv) con genere @genere
				\begin{enumerate}[label*=\arabic*.]
					\item \textbf{\# alt1:} Se il Database risponde con successo, il sistema mostra i prodotti video all'utente
					\item \textbf{\# alt2:} Se si verificano errori nella richiesta, non vengono suggeriti prodotti video
				\end{enumerate}
			\item Il sistema fa una richiesta al Database per ottenere 5 serie tv con genere @genere
				\begin{enumerate}[label*=\arabic*.]
					\item \textbf{\# alt1:} Se il Database risponde con successo, il sistema mostra le serie tv all'utente
					\item \textbf{\# alt2:} Se si verificano errori nella richiesta, non vengono suggerite serie tv
				\end{enumerate}
			\end{enumerate}
			\item \textbf{\# alt2:} Se si verificano errori nella richiesta (per esempio se l'utente non ha prodotti video visualizzati), vengono suggeriti solo i prodotti video restituiti dallo UC\_32
		\end{enumerate}
		\item Il sistema richiede al Database il genere più visualizzato dall'utente considerando gli ultimi 20 prodotti musicali riprodotti (o tutti i prodotti musicali riprodotti se l'utente ha visualizzato meno di 20 prodotti musicali),:
		\begin{enumerate}[label*=\arabic*.]
			\item \textbf{\# alt1:} Se il Database risponde con successo, e il genere più visualizzato è @genere:
			\begin{enumerate}[label*=\arabic*.]
			\item Il sistema fa una richiesta al Database per ottenere 5 prodotti musicali con genere @genere
				\begin{enumerate}[label*=\arabic*.]
					\item \textbf{\# alt1:} Se il Database risponde con successo, il sistema mostra i prodotti musicali all'utente
					\item \textbf{\# alt2:} Se si verificano errori nella richiesta, non vengono suggeriti prodotti musicali
				\end{enumerate}
			\item Il sistema fa una richiesta al Database per ottenere 5 album con genere @genere
				\begin{enumerate}[label*=\arabic*.]
					\item \textbf{\# alt1:} Se il Database risponde con successo, il sistema mostra gli album all'utente
					\item \textbf{\# alt2:} Se si verificano errori nella richiesta, non vengono suggeriti album
				\end{enumerate}
			\end{enumerate}
			\item \textbf{\# alt2:} Se si verificano errori nella richiesta (per esempio se l'utente non ha prodotti musicali visualizzati), vengono suggeriti solo i prodotti musicali restituiti dallo UC\_32
		\end{enumerate}
		\end{enumerate}\\\hline
\end{tabular}
\label{table_use_case:\lastUC}\newline
\end{table}
%=============================

%================UC relativi a 16==========
\begin{table}[bp]
    \centering
    \addtolength{\leftskip} {-2cm}
\begin{tabular}{ |p{2.6cm}|p{13cm}|  }
\hline
ID & UC\_\nextUC \\\hline
Categoria & Prodotto \\\hline
Nome & UC\_VisualizzaPubblicità\\\hline
Priorità & High \\\hline
Attori & Timer \\\hline
Descrizione & Riproduce uno spot pubblicitario\\\hline
Pre-condizioni & Il piano di abbonamento associato all'utente non lo esenta dalla riproduzione periodica di spot pubblicitari\\\hline
Post-condizioni & Nessuna\\\hline
%immagino di avere una variabile che mi permette di sapere se mandare 
%la pubblicità oppure no. Se la mando ovviamente la variabile va resettata altrimenti decremento e riproduco 
%ciò che mi è stato chiesto
Flusso &  	\vspace{-5mm} \begin{enumerate}
		\item Il sistema controlla quante riproduzioni mancano all'avvio dello spot pubblicitario:
			\begin{enumerate}[label*=\arabic*.]
				\item \textbf{\# alt1}: Se mancano 0 riproduzioni allora il sistema resetta questo valore a quello di default e riproduce uno spot pubblicitario qualsiasi
				\item \textbf{\# alt2}: Se mancano 1 o più riproduzioni allora il sistema decrementa questo valore di 1 e non riproduce nessuno spot.
			\end{enumerate}
		\end{enumerate}\\\hline
\end{tabular}
\label{table_use_case:\lastUC}\newline
\end{table}
%======================================

%================UC relativi a 9==========
\begin{table}[bp]
    \centering
    \addtolength{\leftskip} {-2cm}
\begin{tabular}{ |p{2.6cm}|p{13cm}|  }
\hline
ID & UC\_\nextUC \\\hline
Categoria & Prodotto \\\hline
Nome & UC\_CreaPlaylist\\\hline
Priorità & High \\\hline
Attori &  UtenteAutenticato \\\hline
Descrizione & Permette di creare una nuova playlist (vuota)\\\hline
Pre-condizioni & L'utente ha il servizio che permette la creazione di playlist\\\hline
Post-condizioni & Una nuova playlist vuota è asscociata all'utente che l'ha creata\\\hline
Flusso &  	\vspace{-5mm} \begin{enumerate}
	\item L'utente chiede al sistema la creazione di una nuova playlist;
	\item Il sistema richiede all'utente il nome della nuova playlist;
	\item L'utente fornisce l'informazione richiesta;
	\item Il sistema, dopo aver validato l'informazione fornitagli dall'utente, controlle se è già esistente una playlist con quel nome tra quelle create dallo stesso utente:
		\begin{enumerate}[label*=\arabic*.]
			\item \textbf{\# alt1}: Se non ci sono altre playlist, il sistema procede con la creazione di una nuova con il nome fornitogli dall'utente
			\item \textbf{\# alt2}: Se esiste almeno 1 playlist associata a quell'utente, l'operazione fallisce e il sistema chiede all'utente un nuovo nome diverso d quello appena fornito.
		\end{enumerate}
	\end{enumerate}\\\hline
\end{tabular}
\label{table_use_case:\lastUC}\newline
\end{table}

\begin{table}[bp]
    \centering
    \addtolength{\leftskip} {-2cm}
\begin{tabular}{ |p{2.6cm}|p{13cm}|  }
\hline
ID & UC\_\nextUC \\\hline
Categoria & Prodotto \\\hline
Nome & UC\_AggiungiProdottoPlaylist\\\hline
Priorità & High \\\hline
Attori &  UtenteAutenticato \\\hline
Descrizione & Permette di agguingere un prodotto all'interno di una playlist esistente\\\hline
Pre-condizioni & L'utente ha il servizio che permette la creazione di playlist e ha gi\'a scelto il prodotto da aggiungere\\\hline
Post-condizioni & Il prodotto scelto fa parte della playlist scelta\\\hline
Flusso &  	\vspace{-5mm} \begin{enumerate}
	\item Il sistema fornisce la lista delle playlist associate all'utente;
	\item L'utente sceglie la playlist e la comunica al sistema;	
	\item Il sistema procede con l'aggiunta del prodotto scelto all'interno della playist scelta e avvisa l'utente della riuscita dell'operazione.
	\end{enumerate}\\\hline
\end{tabular}
\label{table_use_case:\lastUC}\newline
\end{table}

\begin{table}[bp]
    \centering
    \addtolength{\leftskip} {-2cm}
\begin{tabular}{ |p{2.6cm}|p{13cm}|  }
\hline
ID & UC\_\nextUC\\\hline
Categoria & Prodotto \\\hline
Nome & UC\_RimuoviProdottoPlaylist\\\hline
Priorità & High \\\hline
Attori &  UtenteAutenticato \\\hline
Descrizione & Permette di rimuovere un prodotto all'interno di una playlist esistente\\\hline
Pre-condizioni & L'utente ha scelto la Playlist da cui rimuovere una risorsa tra quelle di cui \'e proprietario.\\\hline
Post-condizioni & Il prodotto scelto non far\'a pi\'u parte della playlist scelta\\\hline
Flusso &  	\vspace{-5mm} \begin{enumerate}
	\item Il sistema fornisce la lista di prodotti associati alla playlist scelta;
	\item L'utente sceglie il prodotto e lo comunica al sistema;
	\item Il sistema procede con la rimozione del prodotto all'interno della playist scelta e avvisa l'utente della riuscita dell'operazione.
	\end{enumerate}\\\hline
\end{tabular}
\label{table_use_case:\lastUC}\newline
\end{table}

\begin{table}[bp]
    \centering
    \addtolength{\leftskip} {-2cm}
\begin{tabular}{ |p{2.6cm}|p{13cm}|  }
\hline
ID & UC\_\nextUC\\\hline
Categoria & Prodotto \\\hline
Nome & UC\_CambiaVisibilit\'aPlaylist\\\hline
Priorità & High \\\hline
Attori &  UtenteAutenticato \\\hline
Descrizione & Permette di rendere pubblica, privata una playlist\\\hline
Pre-condizioni & L'utente ha scelto la playlist da modificare tra quelle di cui \'e proprietario\\\hline
Post-condizioni & La playlist avr\'a visibilit\'a aggiornata a quella fornita dall'utente\\\hline
Flusso &  	\vspace{-5mm} \begin{enumerate}
		\item Il sistema chiede se impostare la visibilit\'a a pubblica o privata;
		\item L'utente sceglie la visibilità e la comunica al sistema;
		\item Il sistema procede con l'aggiornamento dell'informaizone secondo le scelte fatte dall'utente.
	\end{enumerate}\\\hline
\end{tabular}
\label{table_use_case:\lastUC}\newline
\end{table}

\begin{table}[bp]
    \centering
    \addtolength{\leftskip} {-2cm}
\begin{tabular}{ |p{2.6cm}|p{13cm}|  }
\hline
ID & UC\_\nextUC\\\hline
Categoria & Prodotto \\\hline
Nome & UC\_RiproduciPlaylist\\\hline
Priorità & High \\\hline
Attori &  UtenteAutenticato \\\hline
Descrizione & Aggiunge i prodotti associati alla playlist sulla coda di riproduzione\\\hline
Pre-condizioni & L'utente ha scelto la playlist da riprodurre.\\\hline
Post-condizioni & Tutti i prodotti associati alla playlist sono inseriti all'interno della coda di riproduzione\\\hline
Flusso &  \vspace{-5mm}	\begin{enumerate}
		\item Il sistema per ogni prodotto nella coda di riproduzione (UC\_42) procede rimuovendolo da questa coda (UC\_41);
		\item Il sistema per ogni prodotto associata alla playlist scelta dall'utente e lo aggiunge alla coda di riproduzione (UC\_40);
		\item Viene avviata la riproduzione della coda (UC\_43)
		\end{enumerate}\\\hline
\end{tabular}
\label{table_use_case:\lastUC}\newline
\end{table}
%=====================================

%================UC relativi a 18==========

\begin{table}[bp]
    \centering
    \addtolength{\leftskip} {-2cm}
\begin{tabular}{ |p{2.6cm}|p{13cm}|  }
\hline
ID & UC\_\nextUC\\\hline
Categoria & Prodotto \\\hline
Nome & UC\_AggiungiProdottoAllaCoda\\\hline
Priorità & High \\\hline
Attori &  UtenteAutenticato \\\hline
Descrizione & Permette di aggiungere alla coda di riproduzione un prodotto\\\hline
Pre-condizioni & L'utente ha indicato un prodotto che intende aggiungere alla coda di riproduzione, l'utente ha il servizio per riprodurre il prodotto, il prodotto è pubblico\\\hline
Post-condizioni & Il prodotto scelto è inserito in fondo alla coda di riproduzione\\\hline
Flusso &    \vspace{-5mm} \begin{enumerate}
    \item Il sistema aggiunge il prodotto indicato alla fine della coda di riproduzione \newline
    \end{enumerate}\\\hline
\end{tabular}
\label{table_use_case:\lastUC}\newline
\end{table}

\begin{table}[bp]
    \centering
    \addtolength{\leftskip} {-2cm}
\begin{tabular}{ |p{2.6cm}|p{13cm}|  }
\hline
ID & UC\_\nextUC\\\hline
Categoria & Prodotto \\\hline
Nome & UC\_RimuoviProdottoDallaCoda\\\hline
Priorità & High \\\hline
Attori &  UtenteAutenticato \\\hline
Descrizione & Permette di rimuovere dalla coda di riproduzione un prodotto\\\hline
Pre-condizioni & L'utente ha indicato un prodotto della coda di riproduzione che intende rimuovere\\\hline
Post-condizioni & Il prodotto scelto non \'e pi\'u  nella coda di riproduzione\\\hline
Flusso &    \vspace{-5mm} \begin{enumerate}
    \item Il sistema rimuove il prodotto dalla coda di riproduzione \newline \newline
    \end{enumerate}\\\hline
\end{tabular}
\label{table_use_case:\lastUC}\newline
\end{table}

\begin{table}[bp]
    \centering
    \addtolength{\leftskip} {-2cm}
\begin{tabular}{ |p{2.6cm}|p{13cm}|  }
\hline
ID & UC\_\nextUC\\\hline
Categoria & Prodotto \\\hline
Nome & UC\_MostraStatoCoda\\\hline
Priorità & High \\\hline
Attori &  UtenteAutenticato \\\hline
Descrizione & Permette di visualizzare tutti i prodotti nella coda di riproduzione\\\hline
Input & - \\\hline
Output & La lista dei prodotti nella coda di riproduzione\\\hline
Pre-condizioni & Nessuna\\\hline
Post-condizioni & Nessuna\\\hline
Flusso &    \vspace{-5mm} \begin{enumerate}
    \item Viene restituita la lista dei prodotti nella coda di riproduzione \newline \newline
    \end{enumerate}\\\hline
\end{tabular}
\label{table_use_case:\lastUC}\newline
\end{table}


\begin{table}[bp]
    \centering
    \addtolength{\leftskip} {-2cm}
\begin{tabular}{ |p{2.6cm}|p{13cm}|  }
\hline
ID & UC\_\nextUC\\\hline
Categoria & Prodotto \\\hline
Nome & UC\_RiproduciCoda\\\hline
Priorità & High \\\hline
Attori &  UtenteAutenticato \\\hline
Descrizione & Riproduce la coda di riproduzione\\\hline
Pre-condizioni & La coda non è vuota\\\hline
Post-condizioni & Nessuna\\\hline
Flusso &    \vspace{-5mm} \begin{enumerate}
	\item L'utente richiede la riproduzione della coda
 	\item Il sistema estrae il primo prodotto dalla coda di riproduzione
	\item Il sistema rimuove il primo prodotto dalla coda di riproduzione
	\item Il sistema controlla il tipo del prodotto estratto:
		\begin{enumerate}[label*=\arabic*.]
			\item \textbf{\# alt1}: Se il prodotto è di tipo video: viene riprodotto il video (UC\_24)
			\item \textbf{\# alt2}: Se il prodotto è di tipo musicale: viene riprodotto il prodotto musicale (UC\_25)
		\end{enumerate}
	\item Dopo la riproduzione del prodotto (o se la riproduzione del prodotto genera errori),
		\begin{enumerate}[label*=\arabic*.]
			\item \textbf{\# alt1}: Se la coda di riproduzione è vuota, la riproduzione termina
			\item \textbf{\# alt2}: Se la coda di riproduzione non è vuota, si torna al punto 2.
		\end{enumerate}
    \end{enumerate}\\\hline
\end{tabular}
\label{table_use_case:\lastUC}\newline
\end{table}
%======================================

%================UC relativi a 10, 11==========
\begin{table}[bp]
    \centering
    \addtolength{\leftskip} {-2cm}
\begin{tabular}{ |p{2.6cm}|p{13cm}|  }
\hline
ID & UC\_\nextUC\\\hline
Categoria & Prodotto \\\hline
Nome & UC\_CreaSerieTv\\\hline
Priorità & High \\\hline
Attori &  UtenteAutenticato \\\hline
Descrizione & Permette di trasformare una playlist in una serie tv\\\hline
Pre-condizioni & L'utente ha il servizio per creare serie tv, e ha indicato una playlist che intende trasformare in serie tv\\\hline
Post-condizioni & Se non avvengono errori, la playlist indicata diventa una serie tv\\\hline
Flusso &    \vspace{-5mm} \begin{enumerate}
		\item Il sistema verifica che:
			\begin{itemize}
			\item Tutti i prodotti della playlist siano prodotti video
			\item Tutti i prodotti della playlist appartengono all'utente stesso
			\item Per ogni prodotto della playlist, non esiste un'altra playlist dell'utente che contiene quel prodotto
			\end{itemize}
			 \begin{enumerate}[label*=\arabic*.]
				\item \textbf{\# alt1}: Se tutte le condizioni sono verificate: il sistema memorizza che la playlist indicata è una serie tv
				\item \textbf{\# alt2}: Se almeno una condizione non è verificata: il sistema comunica all'utente che la playlist non può essere una serie tv
			\end{enumerate}
    \end{enumerate}\\\hline
\end{tabular}
\label{table_use_case:\lastUC}\newline
\end{table}

\begin{table}[bp]
    \centering
    \addtolength{\leftskip} {-2cm}
\begin{tabular}{ |p{2.6cm}|p{13cm}|  }
\hline
ID & UC\_\nextUC\\\hline
Categoria & Prodotto \\\hline
Nome & UC\_CreaAlbum\\\hline
Priorità & High \\\hline
Attori &  UtenteAutenticato \\\hline
Descrizione & Permette di trasformare una playlist in un album\\\hline
Pre-condizioni & L'utente ha il servizio per creare album, e ha indicato la playlist che intende trasformare in album\\\hline
Post-condizioni & Se non avvengono errori, la playlist indicata diventa un album\\\hline
Flusso &    \vspace{-5mm} \begin{enumerate}
		\item Il sistema verifica che:
			\begin{itemize}
			\item Tutti i prodotti della playlist siano prodotti musicali
			\item Tutti i prodotti della playlist appartengono all'utente stesso
			\item Per ogni prodotto della playlist, non esiste un'altra playlist dell'utente che contiene quel prodotto
			\end{itemize}
			 \begin{enumerate}[label*=\arabic*.]
				\item \textbf{\# alt1}: Se tutte le condizioni sono verificate: il sistema memorizza che la playlist indicata è un album
				\item \textbf{\# alt2}: Se almeno una condizione non è verificata: il sistema comunica all'utente che la playlist non può essere un album
			\end{enumerate}
    \end{enumerate}\\\hline
\end{tabular}
\label{table_use_case:\lastUC}\newline
\end{table}

\begin{table}[bp]
    \centering
    \addtolength{\leftskip} {-2cm}
\begin{tabular}{ |p{2.6cm}|p{13cm}|  }
\hline
ID & UC\_\nextUC \\\hline
Categoria & Risorse\\\hline
Nome & UC\_GestisciScadenzeAbbonamenti\\\hline
Priorità & High \\\hline
Attori &  Time \\\hline
Descrizione & Gestisce un abbonamento nel momento in cui la sottoscrizione scade: effettua un rinnovo, o lo rimuove dai piani di abbonamento dell'utente\\\hline
Pre-condizioni & Nessuna\\\hline
Post-condizioni & Nessun utente ha un abbonamento sottoscritto scaduto\\\hline
Flusso &  	\vspace{-5mm} \begin{enumerate}
			\item All'inizio di ogni giorno, il sistema individua tutti gli utenti che hanno un abbonamento sottoscritto
			\item Per ogni utente individuato, il sistema individua gli abbonamenti sottoscritti dall'utente (UC\_8)
			\item Per ogni abbonamento sottoscritto dall'utente, il sistema verifica se la sottoscrizione è scaduta:
			\begin{enumerate}[label*=\arabic*.]
				\item \textbf{\# alt1}: Se l'abbonamento non è scaduto, la sottoscrizione viene lasciata inalterata
				\item \textbf{\# alt2}: Se l'abbonamento è scaduto, allora:
				\begin{enumerate}[label*=\arabic*.]
					\item \textbf{\# alt1}: Se è attivo il rinnovo automatico e il piano di abbonamento è ancora sottoscrivibile, allora viene contattato il servizio di pagamento esterno per rinnovare l'abbonamento:
					\begin{enumerate}[label*=\arabic*.]
						\item \textbf{\# alt1}: Se il pagamento va a buon fine: la sottoscrizione dell'abbonamento viene rinnovata
						\item \textbf{\# alt2}: Seil pagamento non va a buon fine: il piano di abbonamento viene rimosso dalla lista degli abbonamenti sottoscritti dall'utente, e viene informato l'utente
					\end{enumerate}
					\item \textbf{\# alt2}: Altrimenti: il piano di abbonamento viene rimosso dalla lista degli abbonamenti sottoscritti dall'utente
				\end{enumerate}
			\end{enumerate}
		\end{enumerate}\\\hline
\end{tabular}
\label{table_use_case:\lastUC}\newline
\end{table}
%======================================

%================UC relativi a interfaccia grafica==========
\begin{table}[bp]
    \centering
    \addtolength{\leftskip} {-2cm}
\begin{tabular}{ |p{2.6cm}|p{13cm}|  }
\hline
ID & UC\_\nextUC\\\hline
Categoria & UI \\\hline
Nome & UC\_GestisciAbbonamenti\\\hline
Priorità & High \\\hline
Attori & ManagerAbbonamenti \\\hline
Descrizione & Fornisce l'interfaccia per la gestione degli abbonamenti\\\hline
Pre-condizioni & Nessuna\\\hline
Post-condizioni & Nessuna\\\hline
Flusso &    \vspace{-5mm} \begin{enumerate}
		\item L'utente accede alla pagina di gestione abbonamenti.
		\item Il sistema presenta la pagina di gestione abbonamenti.
		\item L'utente ricerca tramite filtri l'abbonamento o gli abbonamenti di interesse.
		\item Il sistema per ogni abbonamento mostra le azioni possibili su di esso,
		\item L'utente seleziona l'azione da intraprendere.
		\item Il sistema aggiorna il suo stato in seguito all'azione scelta:
			 \begin{enumerate}[label*=\arabic*.]
				\item \textbf{\# alt1}: Se l'azione influisce sui dati: il sistema si interfaccia con il Database, aggiorna i dati sull'abbonamento e comunica la riuscita dell'operazione mostrando i dati aggiornati.
				\item \textbf{\# alt2}: Se l'azione non influisce sui dati viene solo presentata una pagina di risposta inerente all'azione intrapresa (dettaglio abbonamento, statistiche ...)
			\end{enumerate}
    \end{enumerate}\\\hline
\end{tabular}
\label{table_use_case:\lastUC}\newline
\end{table}

%================================

\end{center}

\newpage
\noindent{\large \textbf{Revisioni 8}} \\ \\
\begin{tabular}{|c | c | c | c|} 
 	\hline
	 Numero & Data & Descrizione \\ [0.5ex] 
	\hline\hline
	1 & 20/02/2020 & Stesura iniziale con requisiti funzionali principali \\
	\hline
	2 & 26/02/2020 & Revisione dei principali use case\\
	\hline
\end{tabular}

