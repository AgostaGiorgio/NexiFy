\setlength{\arrayrulewidth}{.5mm}
\setlength{\tabcolsep}{5pt}
\renewcommand{\arraystretch}{2}
\renewcommand{\labelenumii}{\theenumii}
\renewcommand{\theenumii}{\theenumi.\arabic{enumii}.}

\subsection{Introduzione}
La seguente sezione mira all'identificazione dei casi d'uso presenti nel progetto. Di seguito vengono identificati oltre ai casi d'uso, anche gli attori coinvolti nel loro utilizzo.
Lo scopo principale del documento, è quello di avere un riferimento sui casi d'uso leggibile da chiunque, anche a personale non interno al progetto. Questo è utile al fine di permettere a tutti una comprensione e quindi una discussione sui casi d'uso.

\subsection{Legenda}
\large{\textbf{Attori}} \\
\begin{itemize}[]
	\item \textbf{ID}: rappresenta l'identificatore univoco di un attore. La sintassi è del tipo A\_X dove X è una stringa di interi separati da punto che rappresenta la gerarchia di relazioni padre-figlio.
	\item \textbf{Nome}: nome dell'attore, deve essere comunque univoco e deve dare una prima idea di cosa rappresenta quel preciso attore.
	\item \textbf{Genitore}: indica l'identificatore del primo antenato nella gerarchia degli attori.
	\item \textbf{Livello}: indica il grado di rilevanza dell'attore all'interno dell'intero sistema. Può assumere il valore di {Primario, Secondario, Di Supporto}.
	\item \textbf{Tipologia}: classifica l'attore in {Umano, Sistema}
	\item \textbf{Descrizione}: descrive brevemente cosa rappresenta all'interno del sistema l'attore.\\

\end{itemize}


\noindent \large{\textbf{Use Case}} \\
\begin{itemize}[]
	\item \textbf{ID}: identificativo univoco del caso d'uso, è della forma UC\_X dove X è una stringa di interi separati da punto che rappresenta la gerarchia di relazioni padre-figlio.
	\item \textbf{Categoria}: Macroarea di appartenza della funzionalità. Può assumere i valori {User, Subscriptions, Management}
	\item \textbf{Nome}: Nome identificativo del caso d'uso, serve anche per dare un'idea del tipo di attività.
	\item \textbf{Priorità}: Livello di priorità del caso d'uso. Può assumere i valori {High, Medium, Low}. 
	\item \textbf{Attori}: Lista di identificativi degli attori coinvolti nel caso d'uso.
	\item \textbf{Descrizione}: Breve descrizione della funzione svolta dal caso d'uso.
	\item \textbf{Flusso}: Passi logici eseguiti dal caso d'uso per ottenere il risultato desiderato.
	\item \textbf{Pre-condizioni}: Condizioni che devono essere verificate prima che il caso d'uso inizi, altrimenti non potrà essere svolto.
	\item \textbf{Post-condizioni}: Condizioni che devono essere verificate subito dopo l'esecuzione del caso d'uso.
	%\item \textbf{Flusso Alternativo}: 

\end{itemize}

% ============================== ATTORI ====================================

\subsection{Specifica Attori}

\begin{center}

\begin{tabular}{ |p{2cm}|p{10cm}|  }
\hline
ID & A\_1 \\\hline
Nome & Utente\\\hline
Genitore & - \\\hline
Livello &  Primario \\\hline
Tipologia & Umano \\\hline
Descrizione &  L' Utente è il ruolo assunto da tutti coloro che usufruiranno del servizio di streaming offerto dalla piattaforma. \\\hline
\end{tabular}
\label{table_attore:1}\newline

\begin{tabular}{ |p{2cm}|p{10cm}|  }
\hline
ID & A\_1.1 \\\hline
Nome & UtenteNonAutenticato\\\hline
Genitore & A\_1 \\\hline
Livello &  Primario \\\hline
Tipologia & Umano \\\hline
Descrizione &  Questo ruolo è assunto da tutti coloro che utilizzano la piattaforma senza aver effettuato l'accesso ad un profilo utente e che quindi avranno accesso solo a determinate funzionalità \\\hline
\end{tabular}
\label{table_attore:1.1}\newline

\begin{tabular}{ |p{2cm}|p{10cm}|  }
\hline
ID & A\_1.2 \\\hline
Nome & UtenteAutenticato\\\hline
Genitore & A\_1 \\\hline
Livello &  Primario \\\hline
Tipologia & Umano \\\hline
Descrizione &  Questo ruolo è assunto da tutti gli utenti che hanno sottoscritto un contratto con la piattaforma e si sono autenticati in fase di accesso ad essa. \\\hline
\end{tabular}
\label{table_attore:1.2}\newline

\begin{tabular}{ |p{2cm}|p{10cm}|  }
\hline
ID & A\_1.3 \\\hline
Nome & Staff\\\hline
Genitore & A\_1\\\hline
Livello &  Primario \\\hline
Tipologia & Umano \\\hline
Descrizione &  Rappresenta il ruolo di staff della piattaforma \\\hline
\end{tabular}
\label{table_attore:1.3}\newline

\begin{tabular}{ |p{2cm}|p{10cm}|  }
\hline
ID & A\_1.3.1 \\\hline
Nome & ManagerServizi\\\hline
Genitore & A\_1.3\\\hline
Livello &  Primario \\\hline
Tipologia & Umano \\\hline
Descrizione &  Rappresenta il ruolo di chi gestisce i servizi offerti dalla piattaforma \\\hline
\end{tabular}
\label{table_attore:1.3.1}\newline

\begin{tabular}{ |p{2cm}|p{10cm}|  }
\hline
ID & A\_2 \\\hline
Nome & Pagamento\\\hline
Genitore & - \\\hline
Livello &  Primario \\\hline
Tipologia & Sistema \\\hline
Descrizione &  Rappresenta il sistema di pagamento esterno al sistema della piattaforma \\\hline
\end{tabular}
\label{table_attore:2}\newline
\end{center}


% ================ USE CASE CHE VANNO BENE ===========
\subsection{Specifica Use Case}

\begin{center}

%======= UC relativi ai requisiti funzionali 1. ==========
\begin{table}[bp]
    \centering
    \addtolength{\leftskip} {-2cm}
\begin{tabular}{ |p{2.6cm}|p{13cm}|  }
\hline
ID & UC\_1 \\\hline
Categoria & Management\\\hline
Nome & UC\_CreaAbbonamento\\\hline
Priorità & High \\\hline
Attori &  Staff \\\hline
Descrizione & Crea un nuovo piano di abbonamento senza nessun servizio associato.\\\hline
Pre-condizioni &  Nessuna\\\hline
Post-condizioni &  Esiste un piano di abbonamento con il nome fornito dall'amministratore\\\hline
Flusso &  	\begin{enumerate}
			\item L'amministratore richiede la creazione di un nuovo piano di abbonamento;
			\item Il sistema richiede all'amministratore il nome, il prezzo e la durata del nuovo piano di abbonamento;
			\item L'amministratore fornisce i dati;
			\item Il sistema valida gli input, e:
				\begin{enumerate}[  ]
				\item \textbf{\# alt1}: Se non avvengono errori durante la validazione:
					\begin{enumerate}[label*=\arabic*.]
					\item Il sistema ricerca se esistono altri piani di abbonamento con il nome inserito:
						\begin{enumerate}[label*=\arabic*.]
						\item \textbf{\# alt1.1}: Se non esistono altri piani di abbonamento con il nome inserito: il piano di abbonamento viene creato, con i dati inseriti dall'amministratore, la lista dei servizi inizialmente vuota e inizialmente non sottoscrivibile
						\item \textbf{\# alt1.2}: Se esistono altri piani di abbonamento con il nome inserito: la creazione viene annullata e viene comunicato all'amministratore che esiste già un piano di abbonamento con il nome inserito	
						\end{enumerate}
					\end{enumerate}
				\item \textbf{\# alt2}: Se avvengono errori durante la validazione: il piano di abbonamento non viene creato e l'errore viene comunicato all'amministratore
				\end{enumerate}
			
			\end{enumerate}\\\hline
\end{tabular}
\label{table_use_case:1}\newline
\end{table}

\begin{table}[bp]
    \centering
    \addtolength{\leftskip} {-2cm}
\begin{tabular}{ |p{2.6cm}|p{13cm}|  }
\hline
ID & UC\_2 \\\hline
Categoria & Management\\\hline
Nome & UC\_RecuperaAbbonamentiEsistenti\\\hline
Priorità & High \\\hline
Attori &  Staff \\\hline
Descrizione & Recupera tutti i piani di abbonamento attualmente esistenti.\\\hline
Pre-condizioni &  Nessuna \\\hline
Post-condizioni &  Tutti i piani di abbonamento creati vengono restituiti\\\hline
Flusso &  	\begin{enumerate}
			\item Il sistema effettua una richiesta al database distribuito esterno per richiedere tutti i piani di abbonamento, e:
				\begin{enumerate}[  ]
				\item \textbf{\# alt1}: Se il database risponde con successo: viene restituita la lista dei piani di abbonamento trovati
				\item \textbf{\# alt2}: Se il database risponde con errori: viene mostrato l'errore all'amministratore, e la procedura viene abortita
				\end{enumerate}
		\end{enumerate}\\\hline
\end{tabular}
\label{table_use_case:2}\newline
\end{table}

\begin{table}[bp]
    \centering
    \addtolength{\leftskip} {-2cm}
\begin{tabular}{ |p{2.6cm}|p{13cm}|  }
\hline
ID & UC\_3 \\\hline
Categoria & Management\\\hline
Nome & UC\_RecuperaServizi\\\hline
Priorità & High \\\hline
Attori &  Staff \\\hline
Descrizione & Permette di recuperare la lista di tutti i servizi esistenti.\\\hline
Pre-condizioni &  Nessuna.\\\hline
Post-condizioni &  Tutti i servizi esistenti vengono restituiti\\\hline
Flusso &  	\begin{enumerate}
			\item Il sistema effettua una richiesta al database distribuito esterno per richiedere tutti i servizi, e:
				\begin{enumerate}[label*=\arabic*.]
				\item \textbf{\# alt1:} Se il database risponde con successo: il risultato della richiesta viene restituito
				\item \textbf{\# alt2:} Se il database risponde con errori: viene mostrato l'errore all'amministratore, e la procedura viene abortita
				\end{enumerate}
		\end{enumerate}\\\hline
\end{tabular}
\label{table_use_case:3}\newline
\end{table}

\begin{table}[bp]
    \centering
    \addtolength{\leftskip} {-2cm}
\begin{tabular}{ |p{2.6cm}|p{13cm}|  }
\hline
ID & UC\_4 \\\hline
Categoria & Management\\\hline
Nome & UC\_NascondiAbbonamento\\\hline
Priorità & High \\\hline
Attori &  Staff \\\hline
Descrizione & Rende non più sottoscrivibile un piano di abbonamento esistente.\\\hline
Pre-condizioni &  Nessuna \\\hline
Post-condizioni &  Il piano di abbonamento specificato non è più sottoscrivibile\\\hline
Flusso &  	\begin{enumerate}
		\item L'amministratore richiede di invalidare un piano di abbonamento;
		\item Il sistema fornisce una lista dei piani di abbonamento esistenti e sottoscrivibili
			\begin{enumerate}[  ]
			\item (UC\_2): il sistema recupera i piani di abbonamento esistenti
			\item vengono forniti nella lista solo i piani di abbonamento sottoscrivibili
			\end{enumerate}
		\item L'amministratore sceglie un piano di abbonamento tra quelli della lista
		\item Il piano di abbonamento scelto viene reso non sottoscrivibile
		\end{enumerate}\\\hline
\end{tabular}
\label{table_use_case:4}\newline
\end{table}

\begin{table}[bp]
    \centering
    \addtolength{\leftskip} {-2cm}
\begin{tabular}{ |p{2.6cm}|p{13cm}|  }
\hline
ID & UC\_5 \\\hline
Categoria & Management\\\hline
Nome & UC\_MostraAbbonamento\\\hline
Priorità & High \\\hline
Attori &  Staff \\\hline
Descrizione & Rende sottoscrivibile un piano di abbonamento esistente.\\\hline
Pre-condizioni &  Nessuna \\\hline
Post-condizioni &  Il piano di abbonamento specificato è sottoscrivibile\\\hline
Flusso &  	\begin{enumerate}
		\item L'amministratore richiede di rendere sottoscrivibile un piano di abbonamento;
		\item Il sistema fornisce una lista dei piani di abbonamento esistenti e non sottoscrivibili
			\begin{enumerate}[  ]
			\item (UC\_2): il sistema recupera i piani di abbonamento esistenti
			\item vengono forniti nella lista solo i piani di abbonamento non sottoscrivibili
			\end{enumerate}
		\item L'amministratore sceglie un piano di abbonamento tra quelli della lista
		\item Il piano di abbonamento scelto viene reso sottoscrivibile
		\end{enumerate}\\\hline
\end{tabular}
\label{table_use_case:5}\newline
\end{table}


\begin{table}[bp]
    \centering
    \addtolength{\leftskip} {-2cm}
\begin{tabular}{ |p{2.6cm}|p{13cm}|  }
\hline
ID & UC\_6 \\\hline
Categoria & Management\\\hline
Nome & UC\_AggiungiServizioAbbonamento\\\hline
Priorità & High \\\hline
Attori &  ManagerServizi \\\hline
Descrizione & Permette di aggiungere un servizio all'insieme dei servizi disponibili in un piano di abbonamento.\\\hline
Pre-condizioni & Nessuna \\\hline
Post-condizioni &  Il servizio scelto viene aggiunto al piano di abbonamento scelto \\\hline
Flusso &  	\begin{enumerate}
		\item L'amministratore richiede di estendere i servizi offerti da un piano di abbonamento;
		\item Il sistema fornisce una lista dei piani di abbonamento esistenti
		\begin{enumerate}[  ]
			\item (UC\_2): il sistema recupera i piani di abbonamento esistenti
		\end{enumerate}
		\item L'amministratore sceglie un piano di abbonamento
		\item Il sistema mostra la lista dei servizi che possono essere aggiunti		
			\begin{enumerate}[label*=\arabic*.]
			\item (UC\_3) Il sistema recupera tutti i servizi disponibili
			\item (UC\_8) Il sistema recupera tutti i servizi associati al piano di abbonamento indicato
			\item La lista dei servizi sarà la differenza tra il primo e il secondo insieme calcolati
			\end{enumerate}
		\item l'amministratore sceglie un servizio dalla lista
		\item Il sistema aggiunge il servizio scelto alla lista dei servizi per il piano di abbonamento indicato
		\end{enumerate}\\\hline
\end{tabular}
\label{table_use_case:6}\newline
\end{table}

\begin{table}[bp]
    \centering
    \addtolength{\leftskip} {-2cm}
\begin{tabular}{ |p{2.6cm}|p{13cm}|  }
\hline
ID & UC\_7 \\\hline
Categoria & Management\\\hline
Nome & UC\_RimuoviServizioAbbonamento\\\hline
Priorità & High \\\hline
Attori &  ManagerServizi \\\hline
Descrizione & Permette la rimozione di uno dei servizi forniti da un piano abbonamento.\\\hline
Pre-condizioni &  Nessuna \\\hline
Post-condizioni &  Il servizio scelto non fa parte della lista dei servizi del piano di abbonamento indicato\\\hline
Flusso &  	\begin{enumerate}
		\item L'amministratore richiede di rimuovere un servizio offerto da un piano di abbonamento;
		\item Il sistema fornisce una lista dei piani di abbonamento esistenti
		\begin{enumerate}[  ]
			\item (UC\_2): il sistema recupera i piani di abbonamento esistenti
		\end{enumerate}
		\item L'amministratore sceglie un piano di abbonamento
		\item Il sistema mostra la lista dei servizi che possono essere rimossi		
			\begin{enumerate}[label*=\arabic*.]
			\item (UC\_8) Il sistema recupera tutti i servizi associati al piano di abbonamento indicato
			\end{enumerate}
		\item l'amministratore sceglie un servizio dalla lista
		\item Il sistema rimuove il servizio scelto dalla lista dei servizi del piano di abbonamento indicato
		\end{enumerate}\\\hline
\end{tabular}
\label{table_use_case:7}\newline
\end{table}

\begin{table}[bp]
    \centering
    \addtolength{\leftskip} {-2cm}
\begin{tabular}{ |p{2.6cm}|p{13cm}|  }
\hline
ID & UC\_8 \\\hline
Categoria & Management\\\hline
Nome & UC\_RecuperaServiziAbbonamento\\\hline
Priorità & High \\\hline
Attori &  Staff \\\hline
Descrizione & Permette, dato un piano di abbonamento, di reperire tutti i servizi forniti da esso.\\\hline
Pre-condizioni &  Nessuna\\\hline
Post-condizioni &  I servizi forniti dal piano di abbonamento vengono restituiti \\\hline
Flusso &  	\begin{enumerate}
			\item Il sistema effettua una richiesta al database distribuito esterno per richiedere tutti i servizi nella lista del piano di abbonamento indicato, e:
				\begin{enumerate}[  ]
				\item \textbf{\# alt1}: Se il database risponde con successo: viene restituita la lista dei servizi trovata
				\item \textbf{\# alt2}: Se il database risponde con errori: viene mostrato l'errore all'amministratore, e la procedura viene abortita
				\end{enumerate}
		\end{enumerate}\\\hline
\end{tabular}
\label{table_use_case:8}\newline
\end{table}

\begin{table}[bp]
    \centering
    \addtolength{\leftskip} {-2cm}
\begin{tabular}{ |p{2.6cm}|p{13cm}|  }
\hline
ID & UC\_9 \\\hline
Categoria & Management\\\hline
Nome & UC\_RecuperaPianiAbbonamentoUtente\\\hline
Priorità & High \\\hline
Attori &  Staff \\\hline
Descrizione & Permette di recuperare la lista dei piani di abbonamento sottoscritti da un utente.\\\hline
Pre-condizioni &  Nessuna.\\\hline
Post-condizioni &  Viene restituita la lista dei piani di abbonamento sottoscritti dal'utente indicato\\\hline
Flusso &  	\begin{enumerate}	
		\item Il sistema effettua una richiesta al database distribuito esterno per richiedere tutti i piani di abbonamento sottoscritti dall'utente indicato, e:
			\begin{enumerate}[  ]
			\item \textbf{\# alt1}: Se il database risponde con successo: viene restituita la lista dei piani di abbonamento trovata
			\item \textbf{\# alt2}: Se il database risponde con errori: viene mostrato l'errore all'amministratore, e la procedura viene abortita
			\end{enumerate}
		\end{enumerate}\\\hline
\end{tabular}
\label{table_use_case:9}\newline
\end{table}
%==============================


%======= UC relativi a 2 e 3 ========
\begin{table}[bp]
    \centering
    \addtolength{\leftskip} {-2cm}
\begin{tabular}{ |p{2.6cm}|p{13cm}|  }
\hline
ID & UC\_10 \\\hline
Categoria & Management\\\hline
Nome & UC\_EffettuaPagamento\\\hline
Priorità & High \\\hline
Attori &  Staff \\\hline
Descrizione & Permette di effettuare un pagamento verso un utente (dal giorno dell'ultimo pagamento al momenti in cui viene invocato l'use-case).\\\hline
Pre-condizioni &  Nessuna\\\hline
Post-condizioni &  L'utente indicato viene pagato\\\hline
Flusso &  	
		\begin{enumerate}	
		\item L'amministratore richiede di effettuare un pagamento verso un utente
		\item Il sistema mostra la lista degli utenti che possono ricevere un pagamento
			\begin{enumerate}[  ]
			\item Il sistema recupera tutti gli utenti
			\item (UC\_9) per ogni utente il sistema recupera i suoi piani di abbonamento e da questi (UC\_8) recupera i servizi dell'utente
			\item Vengono restituiti gli utenti che hanno il servizio di pubblicare prodotti e il cui account è attivo
			\end{enumerate}
		\item Il sistema individua il sistema di pagamento esterno adatto ad effettuare il pagamento (sulla base dei metodi di pagamento forniti dall'utente)
		\item (UC\_11) Il sistema calcola l'importo da pagare all'utente
		\item Il sistema richiede al sistema di pagamento di effettuare un pagamento all'utente (con i dati inseriti dall'utente e l'importo calcolato dal sistema)
			\begin{enumerate}[  ]
			\item \textbf{\# alt1}: Se il pagamento va a buon fine: viene aggiornato l'ultimo giorno di pagamento per l'utente, viene inviata una mail all'utente comunicandogli che il pagamento è stato effettuato
			\item \textbf{\# alt2}: Se il pagamento non va a buon fine: viene comunicato all'amministratore che si è verificato un errore, e viene inviata una mail all'utente comunicandogli che un tentativo di pagamento non è andato a buon fine
			\end{enumerate}
		\end{enumerate}\\\hline
\end{tabular}
\label{table_use_case:10}\newline
\end{table}

\begin{table}[bp]
    \centering
    \addtolength{\leftskip} {-2cm}
\begin{tabular}{ |p{2.6cm}|p{13cm}|  }
\hline
ID & UC\_11 \\\hline
Categoria & Management\\\hline
Nome & UC\_CalcolaImportoDaPagare\\\hline
Priorità & High \\\hline
Attori &  Staff \\\hline
Descrizione & Calcola l'importa da pagare a un utente dall'ultima volta che ha ricevuto un pagamento fino al momento in cui viene invocato l'use-case\\\hline
Pre-condizioni &  L'utente indicato ha il servizio per pubblicare prodotti e l'account dell'utente è attivo\\\hline
Post-condizioni &  L'utente indicato viene pagato\\\hline
Flusso &  	
		\begin{enumerate}
		\item Il sistema ottiene l'ultimo istante in cui l'utente ha ricevuto un pagamento, e considera l'intervallo che va da quell'istante fino all'istante corrente	
		\item Il sistema effettua una richiesta al database distribuito per calcolare il numero di visualizzazioni complessive ricevute dai prodotti dell'utente nell'intervallo di tempo considerato
		\begin{enumerate}[  ]
			\item \textbf{\# alt1}: Se la richiesta al database ha esito positivo: il sistema restituisce f(numero visualizzazioni complessive), dove f è una funzione che indica come calcolare l'importo da pagare in funzione del numero di visualizzazioni
			\item \textbf{\# alt2}: Se la richiesta al database ha esito negativo: la procedura viene abortita
		\end{enumerate}
		\end{enumerate}\\\hline
\end{tabular}
\label{table_use_case:11}\newline
\end{table}

\begin{table}[bp]
    \centering
    \addtolength{\leftskip} {-2cm}
\begin{tabular}{ |p{2.6cm}|p{13cm}|  }
\hline
ID & UC\_12 \\\hline
Categoria & Management\\\hline
Nome & UC\_SospendiAccount\\\hline
Priorità & High \\\hline
Attori &  Staff \\\hline
Descrizione & Sospende l'account di un utente\\\hline
Pre-condizioni &  Nessuna\\\hline
Post-condizioni &  L'account dell'utente specificato è sospeso\\\hline
Flusso &  	
		\begin{enumerate}
		\item L'amministratore richiede di sospendere l'account di un utente
		\item Il sistema mostra la lista degli utenti che hanno un account attivo
			\begin{enumerate}[  ]
			\item Il sistema recupera tutti gli utenti
			\item vengono mostrati nella lista gli utenti con account attivo
			\end{enumerate}	
		\item L'amministratore sceglie un utente da sospendere
		\item L'account dell'utente viene sospeso
		\end{enumerate}\\\hline
\end{tabular}
\label{table_use_case:12}\newline
\end{table}
%=======================


%======= UC relativi a 4 ===========
\begin{table}[bp]
    \centering
    \addtolength{\leftskip} {-2cm}
\begin{tabular}{ |p{2.6cm}|p{13cm}|  }
\hline
ID & UC\_13 \\\hline
Categoria & Autenticazione \\\hline
Nome & UC\_EffettuaRegistrazione \\\hline
Priorità & High \\\hline
Attori &  UtenteNonAutenticato \\\hline
Descrizione & Registrazione persistente di un utente all'interno della piattaforma \\\hline
Flusso &  	\begin{enumerate}
			\item L'utente richiede di registrarsi alla piattaforma;
			\item Il sistema richiede all' utente non registrato informazioni quali: anagrafiche, email, email di recupero, password, conferma della password, tipo di abbonamento da sottoscrivere;
			\item L'utente non registrato riempie i campi richiesti e li invia al sistema;
			\item Il sistema verifica la correttezza dei dati e controlla se il piano di abbonamento scelto richiede il pagamento di una somma di denaro:
			\begin{enumerate}[  ]
				\item\textbf{\# alt1}: se lo richiede il sistema fornisce un lista dei metodi di pagamento disponibili
				\item L'utente seleziona il metodo di pagamento desiderato;
				\item Il sistema verifica che il metodo selezionato sia valido e reinderizza l'utente al sistema esterno di pagamento, e attende risposta dal sistema esterno:
				\begin{enumerate}[label*=\arabic*.]
					\item \textbf{\# alt1}: se il sistema risponde in maniera positiva: il pagamento è stato effettuato e l'utente viene registrato sulla piattaforma 
					\item \textbf{\# alt2}: se il sistema risponde in maniera negativa: il pagamento non è stato effettuato e si chiede all'utente di riprovare o rinunciare alla registrazione.
				\end{enumerate}
				\item\textbf{\# alt2}: se NON richiede nessun pagaento l'utente viene registrato sulla piattaforma
			\end{enumerate}
		\end{enumerate}\\\hline
Pre-condizioni &  Non esiste un utente registrato con la stessa email dell'utente che avvia la registrazione\\\hline
Post-condizioni &  L'utente che ha richiesto la registrazione è registrato persistentemente, ed il suo account, identificato dalla email formita nel modulo di registrazione, è abbinato al pacchetto di servizi scelto.\\\hline
\end{tabular}
\label{table_use_case:13}\newline
\end{table}

\begin{table}[bp]
    \centering
    \addtolength{\leftskip} {-2cm}
\begin{tabular}{ |p{2.6cm}|p{13cm}|  }
\hline
ID & UC\_14\\\hline
Categoria & Autenticazione \\\hline
Nome & UC\_ModificaProfilo \\\hline
Priorità & Medium \\\hline
Attori &  UtenteAutenticato \\\hline
Descrizione & Permette la modifica da parte di un utente di alcuni dati ineriti in fase di registrazione relativi all'account dell'utente stesso  \\\hline
Flusso &  	\begin{enumerate}
			\item L'utente richiede la modifica dei suoi dati;
			\item Il sistema fornisce i valori attualmente memorizzati sul sistema relativi ai dati modificabili;
			\item L'utente comunica al sistema i nuovi valori da memorizzare;
			\item Il sistema valida i dati proposti dall'utente:
			\begin{enumerate}[  ]
				\item\textbf{\# alt1}: Se i nuovi valori sono validi, questi vengono salvati sul sistema e quindi il profilo dell'utente viene aggiornato;
				\item\textbf{\# alt1}: Se i nuovi valori non sono validi, il salvataggio viene bloccato, l'utente viene avvisato e invitato a fornire dei valori validi.
			\end{enumerate}
		\end{enumerate}\\\hline
Pre-condizioni &  L'utente è già autenticato\\\hline
Post-condizioni &  Le informazioni relative all'utente sono aggiornate secondo i nuovi valori forniti dall'utente.\\\hline
\end{tabular}
\label{table_use_case:14}\newline
\end{table}

\begin{table}[bp]
    \centering
    \addtolength{\leftskip} {-2cm}
\begin{tabular}{ |p{2.6cm}|p{13cm}|  }
\hline
ID & UC\_15 \\\hline
Categoria & Autenticazione \\\hline
Nome & UC\_EffettuaLogin \\\hline
Priorità & High \\\hline
Attori &  UtenteAutenticato \\\hline
Descrizione & Permette ad un utente registrato di accedere al proprio profilo fornendo delle opportune credenziali \\\hline
Flusso &  	\begin{enumerate}
			\item L'utente richiede di accedere alla piattaforma;
			\item Il sistema richiede all' utente le credenziali di accesso quali: email e password;
			\item L'utente riempie i campi richiesti e li invia al sistema;
			\item Il sistema valida i dati e avvia la procedura di autenticazione interfacciandosi con la base di dati degli utenti:
			\begin{enumerate}[  ]
				\item \textbf{\# alt1}: se il sistema risponde in maniera positiva: l'utente è autenticato ed abilitato all'utilizzo dei servizi offerti dall'abbonamento in corso; 
				\item \textbf{\# alt2}: se il sistema risponde in maniera negativa: l'utente non viene autenticato ed è reindirizzato alla pagina di login.
			\end{enumerate}
		\end{enumerate}\\\hline
Pre-condizioni &  L'utente esiste e non è già autenticato\\\hline
Post-condizioni &  L'utente è autenticato ed abilitato ai servizi che gli spettano secondo il contratto sottoscritto.\\\hline
\end{tabular}
\label{table_use_case:15}\newline
\end{table}

\begin{table}[bp]
    \centering
    \addtolength{\leftskip} {-2cm}
\begin{tabular}{ |p{2.6cm}|p{13cm}|  }
\hline
ID & UC\_16\\\hline
Categoria & Autenticazione \\\hline
Nome & UC\_EffettuaLogout \\\hline
Priorità & High \\\hline
Attori &  UtenteAutenticato \\\hline
Descrizione & Permette ad un utente autenticato di uscrire dal proprio profilo \\\hline
Flusso &  	\begin{enumerate}
			\item L'utente richiede di uscire dalla piattaforma;
			\item Il sistema elimina la sessione dell'utente in questione.
		\end{enumerate}\\\hline
Pre-condizioni &  L'utente è già autenticato\\\hline
Post-condizioni &  L'utente non è più autenticato.\\\hline
\end{tabular}
\label{table_use_case:16}\newline
\end{table}
%==============================

%============ UC relativi a 6 =============
\begin{table}[bp]
    \centering
    \addtolength{\leftskip} {-2cm}
\begin{tabular}{ |p{2.6cm}|p{13cm}|  }
\hline
ID & UC\_17 \\\hline
Categoria & Risorse\\\hline
Nome & UC\_SottoscriviAbbonamento\\\hline
Priorità & High \\\hline
Attori &  UtenteAutenticato \\\hline
Descrizione & Permette a un utente di sottoscrivere un nuovo abbonamento con la piattaforma.\\\hline
Flusso &  	\begin{enumerate}
			\item L'utente richiede di sottoscrivere un piano di abbonamento;
			\item Il sistema fornisce una lista dei piani di abbonamento sottoscrivibili e non ancora sottoscritti dall'utente
				\begin{enumerate}[  ]
				\item (UC\_2): il sistema recupera i piani di abbonamento esistenti
				\item (UC\_9): il sistema recupera i piani di abbonamento sottoscritti da un utente
				\item vengono forniti nella lista solo i piani di abbonamento sottoscrivibili che non sono sottoscritti dall'utente
				\end{enumerate}
			\item L'utente sceglie il piano di abbonamento
			\item Il sistema reinderizza l'utente al sistema esterno di pagamento (sulla base del meotodo di pagamento scelto in fase di registrazione), e attende risposta dal sistema esterno
			\begin{enumerate}[  ]
				\item \textbf{\# alt1}: Se il pagamento va a buon fine: il sistema aggiunge il piano di abbonamento alla lista degli abbonamenti attivi dell'utente, salvando il timestamp in cui il piano è stato attivato
				\item \textbf{\# alt2}: Se il pagamento non va a buon fine: il sistema comunica l'errore all'utente e la procedura termina
			\end{enumerate}
		\end{enumerate}\\\hline
Pre-condizioni & L'utente non ha sottoscritto tutti i i piani di abbonamento\\\hline
Post-condizioni &  L'utente ha accesso a tutti i servizi offerti dal nuovo abbonamento sottoscritto.\\\hline
\end{tabular}
\label{table_use_case:17}\newline
\end{table}

\begin{table}[bp]
    \centering
    \addtolength{\leftskip} {-2cm}
\begin{tabular}{ |p{2.6cm}|p{13cm}|  }
\hline
ID & UC\_18 \\\hline
Categoria & Risorse\\\hline
Nome & UC\_DisdiciAbbonamento\\\hline
Priorità & High \\\hline
Attori &  UtenteAutenticato, Amministratore \\\hline
Descrizione & Permette a un utente di disdire un abbonamento sottoscritto, evitando che questo si rinnovi automaticamente nella data prevista.\\\hline
Flusso &  	\begin{enumerate}
			\item L'utente richiede di disdire un piano di abbonamento che aveva sottoscritto
			\item (UC\_9) Il sistema fornisce una lista dei piani di abbonamento sottoscritti dall'utente
			\item L'utente sceglie dalla lista il piano di abbonamento da disdire
			\item Il rinnovo automatico per il piano di abbonamento scelto viene annullato, e l'utente potrà usare i servizi del piano di abbonamento fino alla scadenza
		\end{enumerate}\\\hline
Pre-condizioni & L'utente ha almeno un piano di abbonamento sottoscritto.\\\hline
Post-condizioni &  Il rinnovo automatico per il piano di abbonamento indicato è disattivato.\\\hline
\end{tabular}
\label{table_use_case:18}\newline
\end{table}

\begin{table}[bp]
    \centering
    \addtolength{\leftskip} {-2cm}
\begin{tabular}{ |p{2.6cm}|p{13cm}|  }
\hline
ID & UC\_19 \\\hline
Categoria & Risorse\\\hline
Nome & UC\_CambiaAbbonamento\\\hline
Priorità & High \\\hline
Attori &  UtenteAutenticato \\\hline
Descrizione & Permette a un utente di sottoscrivere un nuovo abbonamento con la piattaforma.\\\hline
Flusso &  	\begin{enumerate}
			\item L'utente richiede di cambiare uno dei suoi abbonamenti attivi (selezionato da una lista), con un'altro abbonamento attivabile (selezionandolo da una lista);
			\item il sistema controlla le politiche di cambio;
			\begin{enumerate}[  ]
				\item \textbf{\# alt1}: se queste politiche permettono il cambio di abbonamento dal vecchio al nuovo per questo utente.
				\begin{enumerate}[label=\arabic*]
					\item \textbf{\# alt1}: se il prezzo degli abbonamento è uguale, il cambio verra effettuato immediatamente
					\item \textbf{\# alt2}: altrimenti verra reindirizzato l'utente sul sistema di pagamento esterno (A\_2).
				\end{enumerate}
				\item \textbf{\# alt2}: altrimenti, viene mostrato all'utente un messaggio di errore e annullata l'operazione.
			\end{enumerate}
			\end{enumerate}\\\hline
Pre-condizioni & L'utente è registrato sulla piattaforma, ha effettuato con successo il Login.\\\hline
Post-condizioni &  l'utente ha sottoscritto il nuovo abbonamento e non possiede piu il precedente.\\\hline
\end{tabular}
\label{table_use_case:19}\newline
\end{table}
%========================



%======= UC relativi a 7 ===========
\begin{table}[bp]
    \centering
    \addtolength{\leftskip} {-2cm}
\begin{tabular}{ |p{2.6cm}|p{13cm}|  }
\hline
ID & UC\_20\\\hline
Categoria & Prodotto \\\hline
Nome & UC\_CreaProdotto \\\hline
Priorità & High \\\hline
Attori &  UtenteAutenticato \\\hline
Descrizione & Permette ad un utente autenticato di uscrire dal proprio profilo \\\hline
Pre-condizioni &  L'utente autenticato ha il servizio per pubblicare prodotti\\\hline
Post-condizioni &  Viene creato un nuovo prodotto con proprietario l'utente autenticato\\\hline
Flusso &  	\begin{enumerate}
			\item L'utente richiede di creare un nuovo prodotto
			\item (UC\_21) Il sistema permette all'utente di inserire informazioni di base sul prodotto
			\item Il sistema chiede all'utente di scegliere tra prodotto video e prodotto musicale
			\item L'utente sceglie e:
			\begin{enumerate}[  ]
				\item \textbf{\# alt1}: Se l'utente sceglie risorsa video: il sistema continua la procedura permettendo all'utente di caricare un prodotto video (UC\_22)
				\item \textbf{\# alt2}: Se l'utente sceglie risorsa musicale: il sistema continua la procedura permettendo all'utente di caricare un prodotto musiclae (UC\_23)
			\end{enumerate}
			\item il proprietario del prodotto è l'utente che ha effettuato la richiesta di creazione
			\item la visibilità del prodotto è inizialmente impostata come "privato"
		\end{enumerate}\\\hline
\end{tabular}
\label{table_use_case:20}\newline
\end{table}

\begin{table}[bp]
    \centering
    \addtolength{\leftskip} {-2cm}
\begin{tabular}{ |p{2.6cm}|p{13cm}|  }
\hline
ID & UC\_21\\\hline
Categoria & Prodotto \\\hline
Nome & UC\_CompilaInformazioniDiBase \\\hline
Priorità & High \\\hline
Attori &  UtenteAutenticato \\\hline
Descrizione & Permette ad un utente autenticato di uscrire dal proprio profilo \\\hline
Pre-condizioni &  Vengono dati un utente e un nuovo prodotto ancora in creazione\\\hline
Post-condizioni &  I campi base del nuovo prodotto in creazione sono compilati\\\hline
Flusso &  	\begin{enumerate}
			\item Il sistema richiede all'utente di inserire nome del prodotto, descrizione del prodotto, genere del prodotto (tra una lista di generi disponibili)
			\item L'utente inserisce i dati per il prodotto
			\item Il sistema valida i dati e verifica che non esista un altro prodotto con lo stesso nome tra quelli creati dall'utente:
			\begin{enumerate}[  ]
				\item \textbf{\# alt1}: Se non si verificano errori: i dati vengono correttamente memorizzati, e:
				\begin{enumerate}[label*=\arabic*.]
					\item Il sistema mostra all'utente un questionario a domande chiuse, atto a calcolare l'età minima per visionare il prodotto
					\item L'utente compila il questionario
					\item Sulla base delle risposte dell'utente, il sistema calcola l'età minima per visionare il prodotto
				\end{enumerate}
				\item \textbf{\# alt2}: Se si verificano errori, viene informato l'utente e viene riportato alla schermata per ricompilare i campi
			\end{enumerate}
		\end{enumerate}\\\hline
\end{tabular}
\label{table_use_case:21}\newline
\end{table}

\begin{table}[bp]
    \centering
    \addtolength{\leftskip} {-2cm}
\begin{tabular}{ |p{2.6cm}|p{13cm}|  }
\hline
ID & UC\_22\\\hline
Categoria & Prodotto \\\hline
Nome & UC\_CreaVideo \\\hline
Priorità & High \\\hline
Attori &  UtenteAutenticato \\\hline
Descrizione & Viene completata la creazione di un prodotto, che è stato scelto come prodotto video\\\hline
Pre-condizioni &   Vengono dati un utente e un nuovo prodotto (video) ancora in creazione\\\hline
Post-condizioni &  Vengono completate le caratteristiche del prodotto video con proprietario l'utente autenticato\\\hline
Flusso &  	\begin{enumerate}
			\item Il sistema richiede di caricare un file video (obbligatorio), almeno un file audio e per ogni file audio deve essere associata una lingua (scelta da una lista di lingue conosciute), un insieme di file dei sottotitoli (opzionali, ma per ogni file dei sottotitoli deve essere associata la lingua)
			\item L'utente carica i file sulla piattaforma
			\item Il sistema carica i file sulla CDN esterna e li associa al prodotto creato dall'utente
		\end{enumerate}\\\hline
\end{tabular}
\label{table_use_case:22}\newline
\end{table}

\begin{table}[bp]
    \centering
    \addtolength{\leftskip} {-2cm}
\begin{tabular}{ |p{2.6cm}|p{13cm}|  }
\hline
ID & UC\_23\\\hline
Categoria & Prodotto \\\hline
Nome & UC\_CreaCanzone \\\hline
Priorità & High \\\hline
Attori &  UtenteAutenticato \\\hline
Descrizione & Viene completata la creazione di un prodotto, che è stato scelto come prodotto musicale\\\hline
Pre-condizioni &   Vengono dati un utente e un nuovo prodotto (musicale) ancora in creazione\\\hline
Post-condizioni &  Vengono completate le caratteristiche del prodotto musicale con proprietario l'utente autenticato\\\hline
Flusso &  	\begin{enumerate}
			\item Il sistema richiede un file audio obbligatorio, un file dei lyrics (opzionale), e un file del video musicale (opzionale)
			\item L'utente carica i file sulla piattaforma
			\item Il sistema carica i file sulla CDN esterna e li associa al prodotto creato dall'utente
		\end{enumerate}\\\hline
\end{tabular}
\label{table_use_case:23}\newline
\end{table}

\begin{table}[bp]
    \centering
    \addtolength{\leftskip} {-2cm}
\begin{tabular}{ |p{2.6cm}|p{13cm}|  }
\hline
ID & UC\_24\\\hline
Categoria & Prodotto \\\hline
Nome & UC\_CambiaStatoPubblicazione \\\hline
Priorità & High \\\hline
Attori &  UtenteAutenticato \\\hline
Descrizione & Permette a un utente autenticato di cambiare lo stato di pubblicazione di uno dei suoi prodotti\\\hline
Pre-condizioni &  L'utente deve avere il servizio per la pubblicazione di prodotti\\\hline
Post-condizioni &  Lo stato di pubblicazione del prodotto viene cambiato con quello scelto dall'utente\\\hline
Flusso &  	\begin{enumerate}
			\item L'utente autenticato richiede di cambiare lo stato di pubblicazione di un prodotto
			\item Il sistema mostra all'utente la lista dei suoi prodotti
			\item L'utente seleziona uno dei suoi prodotti
			\item Il sistema richiede il nuovo stato di pubblicazione per il prodotto
			\item L'utente sceglie il nuovo stato di pubblicazione tra "privato", "pubblico" e "amici", e:
				\begin{enumerate}[  ]
					\item \textbf{\# alt1:} L'utente sceglie privato: il sistema cambia lo stato di pubblicazione in "privato" e il prodotto diventa accessibile solamente all'utente stesso
					\item \textbf{\# alt2:} L'utente sceglie pubblico: il sistema cambia lo stato di pubblicazione in "pubblico" e il prodotto diventa accessibile a tutti gli utenti che hanno il servizio di visionare quel tipo di prodotto
					\item \textbf{\# alt3:} L'utente sceglie amici: il sistema cambia lo stato di pubblicazione in "amici", e:
					\begin{enumerate}[label*=\arabic*.]
						\item Il sistema richiede una lista di email, cioè gli utenti che possono accedere al prodotto
						\item L'utente fornisce la lista di email
						\item Il sistema valida le email, e per ogni email inserita:
							\begin{enumerate}[label*=\arabic*.]
								\item \textbf{\# alt1:} Se l'email è valida e corrisponde a un utente esistente, allora viene aggiunta alla lista degli utenti che possono accedere al prodotto
								\item \textbf{\# alt2:} Se l'email non è valida o non corrisponde a un utente esistente, viene comunicato all'utente e non viene inserita nella lista
							\end{enumerate}
					\end{enumerate}
				\end{enumerate}
		\end{enumerate}\\\hline
\end{tabular}
\label{table_use_case:24}\newline
\end{table}
%====================


%============ UC relativi a 8 ============= 
\begin{table}[bp]
    \centering
    \addtolength{\leftskip} {-2cm}
\begin{tabular}{ |p{2.6cm}|p{13cm}|  }
\hline
ID & UC\_25 \\\hline
Categoria & Risorse\\\hline
Nome & UC\_RiproduciVideo\\\hline
Priorità & High \\\hline
Attori &  UtenteAutenticato \\\hline
Descrizione & Permette di riprodurre una risorsa video.\\\hline
Flusso &  	\begin{enumerate}
			\item L'utente selezione a seguito di una ricerca la risorsa da riprodurre (la lista di risorse riproducibili varia a seconda dell'abbonamento/i del richiedente)
			\item il sistema fornisce al client le informazioni sul nodo della cdn dal quale reperire la risorsa 
			\item il client contatta il nodo e stabiliscono una connessione
			\item inizia lo stram del pacchetto mkv 
			\end{enumerate}
			\\\hline
Pre-condizioni & L'utente è registrato sulla piattaforma, ha effettuato con successo il Login.\\\hline
Post-condizioni & NESSUNA\\\hline
\end{tabular}
\label{table_use_case:25}\newline
\end{table}

\begin{table}[bp]
    \centering
    \addtolength{\leftskip} {-2cm}
\begin{tabular}{ |p{2.6cm}|p{13cm}|  }
\hline
ID & UC\_26 \\\hline
Categoria & Risorse\\\hline
Nome & UC\_RiproduciMusica\\\hline
Priorità & High \\\hline
Attori &  UtenteAutenticato \\\hline
Descrizione & riproduce una risorsa musicale, con eventuale traccia video adeguata (lyrics o video musicale).\\\hline
Flusso &  	\begin{enumerate}
			\item L'utente selezione a seguito di una ricerca la risorsa da riprodurre (la lista di risorse riproducibili varia a seconda dell'abbonamento/i del richiedente)
			\item il sistema fornisce al client le informazioni sul nodo della cdn dal quale reperire la risorsa 
			\item il client contatta il nodo e stabiliscono una connessione
			\item \begin{enumerate}[label= ]
				\item \textbf{\# alt1}: Se il file ha associato un file di lyrics, verra inviata al client anche quest'ultima nella prima fase di comunicazione tra client e server
				\item \textbf{\# alt2}: Altrimenti non sarà possibile visualizzare il testo del brano.
			\end{enumerate}
			\item \begin{enumerate}[label= ]
				\item \textbf{\# alt1}: se il file ha associato un video musicale, il file mkv conterrà quella traccia video
				\item \textbf{\# alt2}: altrimenti verrà inviato un mkv con solo la traccia audio
			\end{enumerate}
			\item inizia lo stram del pacchetto mkv
			\end{enumerate}
			\\\hline
Pre-condizioni & L'utente è registrato sulla piattaforma, ha effettuato con successo il Login.\\\hline
Post-condizioni & NESSUNA\\\hline
\end{tabular}
\label{table_use_case:26}\newline
\end{table}

\begin{table}[bp]
    \centering
    \addtolength{\leftskip} {-2cm}
\begin{tabular}{ |p{2.6cm}|p{13cm}|  }
\hline
ID & UC\_27 \\\hline
Categoria & Risorse\\\hline
Nome & UC\_PausaPlayer\\\hline
Priorità & High \\\hline
Attori &  UtenteAutenticato \\\hline
Descrizione & mette in pausa la visualizzazione della risorsa.\\\hline
Flusso &  	\begin{enumerate}
			\item L'utente richiede la pausa di riproduzione al player
			\item il player ferma la riproduzione della risorsa ma continua a ricevere pacchetti dalla CDN 
			\end{enumerate}
			\\\hline
Pre-condizioni & L'utente è registrato sulla piattaforma, ha effettuato con successo il Login, sta riproducento una risorsa.\\\hline
Post-condizioni & NESSUNA\\\hline
\end{tabular}
\label{table_use_case:27}\newline
\end{table}

\begin{table}[bp]
    \centering
    \addtolength{\leftskip} {-2cm}
\begin{tabular}{ |p{2.6cm}|p{13cm}|  }
\hline
ID & UC\_28 \\\hline
Categoria & Risorse\\\hline
Nome & UC\_SpostaPuntoRiproduzionePlayer\\\hline
Priorità & High \\\hline
Attori &  UtenteAutenticato \\\hline
Descrizione & cambia il punto di riproduzione della risorsa (pu\'o essere spostata avanti o indietro).\\\hline
Flusso &  	\begin{enumerate}
			\item L'utente seleziona lo spostamento di riproduzione ad un certo punto compreso tra 0 e lunghezza della risorsa
			\item il player interrompe la riproduzione come da UC\_19
			\item il client comunica al nodo della CDN che sta fornendo la risorsa il nuovo istante di tempo dal quale proseguire
			\item il server risponde con il prossimo pacchetto da riprodurre. che coincide con il pacchetto dell'istante richiesto
		\end{enumerate}
			\\\hline
Pre-condizioni & L'utente è registrato sulla piattaforma, ha effettuato con successo il Login, sta riproducento una risorsa.\\\hline
Post-condizioni & la riproduzione continua dall'istante selezionato\\\hline
\end{tabular}
\label{table_use_case:28}\newline
\end{table}
%====================


%============ UC relativi a 13 ============= 
\begin{table}[bp]
    \centering
    \addtolength{\leftskip} {-2cm}
\begin{tabular}{ |p{2.6cm}|p{13cm}|  }
\hline
ID & UC\_29 \\\hline
Categoria & Risorse\\\hline
Nome & UC\_SegnalaProdotto\\\hline
Priorità & High \\\hline
Attori &  UtenteAutenticato, Staff \\\hline
Descrizione & Permette di segnalare un prodotto (con relativa motivazione della segnalazione).\\\hline
Flusso &  	\begin{enumerate}
			\item L'utente seleziona il prodotto da segnalare:
			\item \begin{enumerate}[label= ]
				\item \textbf{\# alt1}: Se il prodotto da segnalare è un profilo utente, sarà selezionato da un pulsante nella pagina di dettaglio.
				\item \textbf{\# alt2}: Se il prodotto da segnalare è una risorsa, sarà selezionato da un pulsante nella pagina di visualizzazione della risorsa
				\end{enumerate}
			\item All'utente viene richiesta una motivazione della segnalazione.
			\item il sistema riceve l'identificativo del prodotto, la sua tipologia (profilo, risorsa ...), la motivazione e provvede ad aprire la segnalazione
			\begin{enumerate}[label=\arabic*. ]
				\item Viene memorizzato in modo persistente l'utente che ha segnalatore, il prodotto segnalato, e la motivazione
				\item Viene memorizzato in modo persistente l'utente segnalatore, il prodotto segnalato, la motivazione e l'istante in cui è stata sottomessa.
			\end{enumerate}.
			\item L'utente viene segnalato dell'esito della sengalazione
			\begin{enumerate}[label= ]
				\item \textbf{\# alt1}: Se il la segnalazione è andata a buon fine, l'utente riceverà un responso positivo
				\item \textbf{\# alt2}: Se il la segnalazione è andata a buon fine, l'utente riceverà un responso negativo
			\end{enumerate}
			\end{enumerate}
			\\\hline
Pre-condizioni & L'utente è registrato sulla piattaforma, ha effettuato con successo il Login.\\\hline
Post-condizioni & Il prodotto selezionato ha una segnalazione aperta\\\hline
\end{tabular}
\label{table_use_case:29}\newline
\end{table}

\begin{table}[bp]
    \centering
    \addtolength{\leftskip} {-2cm}
\begin{tabular}{ |p{2.6cm}|p{13cm}|  }
\hline
ID & UC\_30 \\\hline
Categoria & Risorse\\\hline
Nome & UC\_OttieniSegnalazioni\\\hline
Priorità & High \\\hline
Attori &  Staff \\\hline
Descrizione & permette agli amministratori di ottenere le segnalazioni per un certo utente.\\\hline
Flusso &  	\begin{enumerate}
			\item L'operatore dello staff richiede al sistema la lista delle segnalazioni.
			\item Il sistema ricerca le segnalazioni fornisce la lista di segnalazioni all'operatore richiedente.
			\end{enumerate}
			\\\hline
Pre-condizioni & NESSUNA\\\hline
Post-condizioni & NESSUNA\\\hline
\end{tabular}
\label{table_use_case:30}\newline
\end{table}

\begin{table}[bp]
    \centering
    \addtolength{\leftskip} {-2cm}
\begin{tabular}{ |p{2.6cm}|p{13cm}|  }
\hline
ID & UC\_31 \\\hline
Categoria & Risorse\\\hline
Nome & UC\_ChiudiSegnalazione\\\hline
Priorità & High \\\hline
Attori &  Staff \\\hline
Descrizione & permette agli amministratori di ottenere le segnalazioni per un certo utente.\\\hline
Flusso &  	\begin{enumerate}
			\item L'operatore dello staff richiede al sistema la lista delle segnalazioni secondo UC\_30.
%			POTREBBE ESSERE UN ALTRO USE CASE
%			\item L'operatore potra richiedere di visionare un elemento della lista:
%				\begin{enumerate}
%				\item \textbf{\# alt1}: Se è stato segnalato un profilo utente si potr\'a visionare il dettaglio di esso.
%				\item \textbf{\# alt2}: Se è stata segnalata una risorsa sar\'a visionabile la risorsa in questione.
%				\end{enumerate}
			\item L'operatore seleziona il prodotto da chiudere fornendo un esito.
			\item il sistema riceve la segnalazione da chiudere, l'esito e chiude la segnalazione: 
			memorizza in maniera persistente l'esito della segnalazione. 
			\begin{enumerate}
			\item memorizza persistentemente l'esito della segnalazione associato al prodotto segnalato
			\end{enumerate}
			\item \begin{enumerate}[label= ]
				\item \textbf{\# alt1}: Se il la chiudura della segnalazione è andata a buon fine, l'utente riceverà un responso positivo
				\item \textbf{\# alt2}: Se il la chiudura della segnalazione è andata a buon fine, l'utente riceverà un responso negativo
			\end{enumerate}
			\end{enumerate}
			\\\hline
Pre-condizioni & NESSUNA\\\hline
Post-condizioni & La segnalazione selezione è chiusa ed ha un esito\\\hline
\end{tabular}
\label{table_use_case:31}\newline
\end{table}
%===========================

%================UC relativi a 5==========
\begin{table}[bp]
    \centering
    \addtolength{\leftskip} {-2cm}
\begin{tabular}{ |p{2.6cm}|p{13cm}|  }
\hline
ID & UC\_32 \\\hline
Categoria & Ricerca \\\hline
Nome & UC\_RicercaContenuto\\\hline
Priorità & High \\\hline
Attori &  Utente \\\hline
Descrizione & Cerca contenuti (sia prodotti che playlist) coerenti con una stringa digitata dall'utente\\\hline
Pre-condizioni &  Nessuna\\\hline
Post-condizioni &  Nessuna\\\hline
Flusso &  	\begin{enumerate}
			\item L'utente richiede di ricercare una risorsa;
			\item Il sistema richiede all' utente la stringa di ricerca;
			\item L'utente fornisce al sistema la stringa;
			\item Il sistema valida l'input e:
			\begin{enumerate}[label*=\arabic*.]
			\item \textbf{\# alt1:} Se l'input supera la fase di validazione: il sistema effettua una richiesta al database distribuito, richiedendo tutti i prodotti pubblici che contengono la stringa inserita dall'utente nel proprio nome, e tutte le serie tv e album pubblici che contengono la stringa inserita dall'utente nel proprio nome
				\begin{enumerate}[label*=\arabic*.]
					\item \textbf{\# alt1:} Se il database risponde con successo, il sistema mostra i risultati ottenuti all'utente
					\item \textbf{\# alt2:} Se si verificano errori nella richiesta, la procedura viene annullata e si comunicano gli errori all'utente
				\end{enumerate}
			\item \textbf{\# alt2:} Altrimenti: la procedura viene annullata e viene comunicato all'utente che l'input non è valido
			\end{enumerate}
			
		\end{enumerate}\\\hline
\end{tabular}
\label{table_use_case:32}\newline
\end{table}

\begin{table}[bp]
    \centering
    \addtolength{\leftskip} {-2cm}
\begin{tabular}{ |p{2.6cm}|p{13cm}|  }
\hline
ID & UC\_33 \\\hline
Categoria & Ricerca \\\hline
Nome & UC\_RicercaPopolari\\\hline
Priorità & Medium \\\hline
Attori &  Utente \\\hline
Descrizione & Cerca contenuti (sia prodotti che playlist) ritenuti popolari\\\hline
Pre-condizioni & Nessuna\\\hline
Post-condizioni &  Vengono mostrati video, canzoni, serie tv, album ritenuti popolari (se ne esistono)\\\hline
Flusso &  	\begin{enumerate}
		\item Il sistema fa una richiesta al database per ottenere i primi 10 prodotti video (non appartententi a serie tv) per numero di visualizzazioni ricevute negli ultimi 7 giorni:
				\begin{enumerate}[label*=\arabic*.]
					\item \textbf{\# alt1:} Se il database risponde con successo, il sistema mostra i video popolari all'utente
					\item \textbf{\# alt2:} Se si verificano errori nella richiesta, si comunica che non sono disponibili video popolari al momento
				\end{enumerate}
		\item Il sistema fa una richiesta al database per ottenere i primi 10 prodotti musicali per numero di visualizzazioni ricevute negli ultimi 7 giorni:
				\begin{enumerate}[label*=\arabic*.]
					\item \textbf{\# alt1:} Se il database risponde con successo, il sistema mostra le canzoni popolari all'utente
					\item \textbf{\# alt2:} Se si verificano errori nella richiesta, si comunica che non sono disponibili musicali popolari al momento
				\end{enumerate}
		\item Il sistema fa una richiesta al database per ottenere le prime 10 serie tv per numero di visualizzazioni (le visualizzazioni di una serie tv sono la somma delle visualizzazioni dei singoli episodi) ricevute negli ultimi 7 giorni:
				\begin{enumerate}[label*=\arabic*.]
					\item \textbf{\# alt1:} Se il database risponde con successo, il sistema mostra le serie tv popolari all'utente
					\item \textbf{\# alt2:} Se si verificano errori nella richiesta, si comunica che non sono disponibili serie tv popolari al momento
				\end{enumerate}
		\item Il sistema fa una richiesta al database per ottenere i primi 10 album per numero di visualizzazioni (le visualizzazioni di un album sono la somma delle visualizzazioni delle singole canzoni) ricevute negli ultimi 7 giorni:
				\begin{enumerate}[label*=\arabic*.]
					\item \textbf{\# alt1:} Se il database risponde con successo, il sistema mostra gli album popolari all'utente
					\item \textbf{\# alt2:} Se si verificano errori nella richiesta, si comunica che non sono disponibili album popolari al momento
				\end{enumerate}
		\end{enumerate}\\\hline
\end{tabular}
\label{table_use_case:33}\newline
\end{table}

\begin{table}[bp]
    \centering
    \addtolength{\leftskip} {-2cm}
    \vspace{-4.35cm}
\begin{tabular}{ |p{2.6cm}|p{13cm}|  }
\hline
ID & UC\_34 \\\hline
Categoria & Ricerca \\\hline
Nome & UC\_SuggerisciContenuti\\\hline
Priorità & Medium \\\hline
Attori &  UtenteAutenticato \\\hline
Descrizione & Suggerisce contenuti (sia prodotti che playlist) all'utente sulla base delle ultime visualizzazioni\\\hline
Pre-condizioni &  Nessuna\\\hline
Post-condizioni &  Nessuna\\\hline
Flusso &  	\vspace{-5mm} \begin{enumerate}
		\item Il sistema richiede al database il genere più visualizzato dall'utente considerando gli ultimi 20 prodotti video riprodotti (o tutti i prodotti video riprodotti se l'utente ha visualizzato meno di 20 prodotti video),:
		\begin{enumerate}[label*=\arabic*.]
			\item \textbf{\# alt1:} Se il database risponde con successo, e il genere più visualizzato è @genere:
			\begin{enumerate}[label*=\arabic*.]
			\item Il sistema fa una richiesta al database per ottenere 5 prodotti video (non appartententi a una serie tv) con genere @genere
				\begin{enumerate}[label*=\arabic*.]
					\item \textbf{\# alt1:} Se il database risponde con successo, il sistema mostra i prodotti video all'utente
					\item \textbf{\# alt2:} Se si verificano errori nella richiesta, non vengono suggeriti prodotti video
				\end{enumerate}
			\item Il sistema fa una richiesta al database per ottenere 5 serie tv con genere @genere
				\begin{enumerate}[label*=\arabic*.]
					\item \textbf{\# alt1:} Se il database risponde con successo, il sistema mostra le serie tv all'utente
					\item \textbf{\# alt2:} Se si verificano errori nella richiesta, non vengono suggerite serie tv
				\end{enumerate}
			\end{enumerate}
			\item \textbf{\# alt2:} Se si verificano errori nella richiesta (per esempio se l'utente non ha prodotti video visualizzati), non vengono suggeriti prodotti video all'utente
		\end{enumerate}
		\item Il sistema richiede al database il genere più visualizzato dall'utente considerando gli ultimi 20 prodotti musicali riprodotti (o tutti i prodotti musicali riprodotti se l'utente ha visualizzato meno di 20 prodotti musicali),:
		\begin{enumerate}[label*=\arabic*.]
			\item \textbf{\# alt1:} Se il database risponde con successo, e il genere più visualizzato è @genere:
			\begin{enumerate}[label*=\arabic*.]
			\item Il sistema fa una richiesta al database per ottenere 5 prodotti musicali con genere @genere
				\begin{enumerate}[label*=\arabic*.]
					\item \textbf{\# alt1:} Se il database risponde con successo, il sistema mostra i prodotti musicali all'utente
					\item \textbf{\# alt2:} Se si verificano errori nella richiesta, non vengono suggeriti prodotti musicali
				\end{enumerate}
			\item Il sistema fa una richiesta al database per ottenere 5 album con genere @genere
				\begin{enumerate}[label*=\arabic*.]
					\item \textbf{\# alt1:} Se il database risponde con successo, il sistema mostra gli album all'utente
					\item \textbf{\# alt2:} Se si verificano errori nella richiesta, non vengono suggeriti album
				\end{enumerate}
			\end{enumerate}
			\item \textbf{\# alt2:} Se si verificano errori nella richiesta (per esempio se l'utente non ha prodotti musicali visualizzati), non vengono suggeriti prodotti musicali all'utente
		\end{enumerate}
		\end{enumerate}\\\hline
\end{tabular}
\label{table_use_case:34}\newline
\end{table}
%=============================

%================UC relativi a 16==========
\begin{table}[bp]
    \centering
    \addtolength{\leftskip} {-2cm}
\begin{tabular}{ |p{2.6cm}|p{13cm}|  }
\hline
ID & UC\_35 \\\hline
Categoria & Risorse \\\hline
Nome & UC\_VisualizzaPubblicità\\\hline
Priorità & High \\\hline
Attori &  Staff \\\hline
Descrizione & Riproduce uno spot pubblicitario\\\hline
Pre-condizioni & L'utente è registrato e autenticato correttamente e inoltre il suo piano di abbonamenti non lo esenta dalla riproduzione periodica di spot pubblicitari\\\hline
Post-condizioni & Nessuna\\\hline
%immagino di avere una variabile che mi permette di sapere se mandare 
%la pubblicità oppure no. Se la mando ovviamente la variabile va resettata altrimenti decremento e riproduco 
%ciò che mi è stato chiesto
Flusso &  	\begin{enumerate}
		\item Il sistema controlla che il piano di abbonamento associato all'utente non lo esenti dalle pubblicità:
		 \begin{enumerate}[label*=\arabic*.]
			\item \textbf{\# alt1}: Se l'utente \'e esentato dalla riproduzione di spot pubblicitari il sistema non riproduce nessuno spot
			\item \textbf{\# alt2}: Se l'utente non \'e esentato dalle pubblicità allora il sistema controlla quante riproduzioni mancano all'avvio dello spot pubblicitario:
			 \begin{enumerate}[label*=\arabic*.]
				\item \textbf{\# alt2.1}: Se mancano 0 riproduzioni allora il sistema resetta questo valore a quello di default e riproduce uno spot pubblicitario qualsiasi
				\item \textbf{\# alt2.2}: Se mancano 1 o più riproduzioni allora il sistema decrementa questo valore di 1 e non riproduce nessuno spot.
			\end{enumerate}
					\end{enumerate}

		\end{enumerate}\\\hline
\end{tabular}
\label{table_use_case:35}\newline
\end{table}
%======================================

% ============================== USE CASE DA RIVEDERE ====================================

\textbf{USE CASE DA RIVEDERE} \newline

\begin{table}[bp]
    \centering
    \addtolength{\leftskip} {-2cm}
\begin{tabular}{ |p{2.6cm}|p{13cm}|  }
\hline
ID & UC\_36 \\\hline
Categoria & Prodotto \\\hline
Nome & UC\_CreaPlaylist\\\hline
Priorità & High \\\hline
Attori &  UtenteAutenticato \\\hline
Descrizione & Permette di creare una nuova playlist (vuota)\\\hline
Pre-condizioni & L'utente è registrato e autenticato correttamente e inoltre il suo piano di abbonamento permette la creazione di playlist\\\hline
Post-condizioni & Una nuova playlist vuote è asscociata all'utente che l'ha creata\\\hline
Flusso &  	\begin{enumerate}
\item L'utente chiede al sistema la creazione di una nuova playlist;
		\item Il sistema verifica sulla base di dati che il piano di abbonamento sottoscritto dall'utente permetta la creazione di nuove playlist:
		 \begin{enumerate}[label*=\arabic*.]
			\item \textbf{\# alt1}: Se il piano associato all'utente non permette la creazione di playlist l'operazione fallisce e l'utente viene avvisato;
			\item \textbf{\# alt2}: Se il piano associato all'utente permette la creazione di playlist, il sistema richiede all'utente il nome della nuova playlist;
			\item L'utente fornisce l'informazione richiesta;
			\item Il sistema, dopo aver validato l'informazione fornitagli dall'utente, controlle se è già esistente una playlist con quel nome tra quelle create dallo stesso utente:
			\begin{enumerate}[label*=\arabic*.]
				\item \textbf{\# alt2.1}: Se non ci sono altre playlist, il sistema procede con la creazione di una nuova con il nome fornitogli dall'utente
				\item \textbf{\# alt2.2}: Se esiste almeno 1 playlist associata a quell'utente, l'operazione fallisce e il sistema chiede all'utente un nuovo nome diverso d quello appena fornito.
			\end{enumerate}
					\end{enumerate}

		\end{enumerate}\\\hline
\end{tabular}
\label{table_use_case:36}\newline
\end{table}

\begin{table}[bp]
    \centering
    \addtolength{\leftskip} {-2cm}
\begin{tabular}{ |p{2.6cm}|p{13cm}|  }
\hline
ID & UC\_37 \\\hline
Categoria & Prodotto \\\hline
Nome & UC\_AggiungiProdottoPlaylist\\\hline
Priorità & High \\\hline
Attori &  UtenteAutenticato \\\hline
Descrizione & Permette di agguingere una risorsa all'interno di una playlist esistente\\\hline
Pre-condizioni & L'utente è registrato e autenticato correttamente e la playlist è associata a lui\\\hline
Post-condizioni & Una nuova risorsa farà parte della playlist scelta\\\hline
Flusso &  	\begin{enumerate}
\item L'utente chiede al sistema di aggiungere una risorsa ad una playlist;
		\item Il sistema verifica sulla base di dati che il piano di abbonamento sottoscritto dall'utente permetta la modifica di playlist:
		 \begin{enumerate}[label*=\arabic*.]
			\item \textbf{\# alt1}: Se il piano associato all'utente non permette la modifica di playlist l'operazione fallisce e l'utente viene avvisato;
			\item \textbf{\# alt2}: Se il piano associato all'utente permette la modifica di playlist, il sistema  fornisce la lista delle playlist associate ad esso;
			\begin{enumerate}[label*=\arabic*.]
				\item L'utente sceglie la playlist e la comunica al sistema;	
				\item Il sistema chiede all'utente di fornire l'informazione riguardo la risorsa da aggiungere;
				\item L'utente fornisce la risorsa da aggiungere;
				\item Il sistema procede con l'aggiunta della risorsa all'interno della playist scelta e avvisa l'utente della riuscita dell'operazione.
			\end{enumerate}
					\end{enumerate}

		\end{enumerate}\\\hline
\end{tabular}
\label{table_use_case:37}\newline
\end{table}

\begin{table}[bp]
    \centering
    \addtolength{\leftskip} {-2cm}
\begin{tabular}{ |p{2.6cm}|p{13cm}|  }
\hline
ID & UC\_38\\\hline
Categoria & Prodotto \\\hline
Nome & UC\_RimuoviProdottoPlaylist\\\hline
Priorità & High \\\hline
Attori &  UtenteAutenticato \\\hline
Descrizione & Permette di rimuovere una risorsa all'interno di una playlist esistente\\\hline
Pre-condizioni & L'utente è registrato e autenticato correttamente e la playlist è associata a lui\\\hline
Post-condizioni & La risorsa non farà più parte della playlist\\\hline
Flusso &  	\begin{enumerate}
\item L'utente chiede al sistema di rimuovere una risorsa da una playlist;
		\item Il sistema verifica sulla base di dati che il piano di abbonamento sottoscritto dall'utente permetta la modifica di playlist:
		 \begin{enumerate}[label*=\arabic*.]
			\item \textbf{\# alt1}: Se il piano associato all'utente non permette la modifica di playlist l'operazione fallisce e l'utente viene avvisato;
			\item \textbf{\# alt2}: Se il piano associato all'utente permette la modifica di playlist, il sistema  fornisce la lista delle playlist associate ad esso;
			\begin{enumerate}[label*=\arabic*.]
				\item L'utente sceglie la playlist da modificare dalla lista;
				\item Il sistema fornisce la lista di risorse associate alla playlist scelta;
				\item L'utente sceglie la risorsa e la comunica al sistema;
				\item Il sistema procede con la rimozione della risorsa scelta dalla playist e avvisa l'utente della riuscita dell'operazione.
			\end{enumerate}
		\end{enumerate}
		\end{enumerate}\\\hline
\end{tabular}
\label{table_use_case:38}\newline

\begin{tabular}{ |p{2cm}|p{13cm}|  }
\hline
ID & UC\_39\\\hline
Categoria & Prodotto \\\hline
Nome & UC\_CambiaVisibilit\'aPlaylist\\\hline
Priorità & High \\\hline
Attori &  UtenteAutenticato \\\hline
Descrizione & Permette di rendere pubblica, privata una playlist\\\hline
Pre-condizioni & L'utente è registrato e autenticato correttamente e la playlist è associata a lui\\\hline
Post-condizioni & La playlist avr\'a visibilit\'a aggiornata a quella fornita dall'utente\\\hline
Flusso &  	\begin{enumerate}
		\item L'utente chiede al sistema di cambiare la visibilit\'a di una playlist;
		\item Il sistema verifica sulla base di dati che il piano di abbonamento sottoscritto dall'utente permetta la modifica di playlist:
		 \begin{enumerate}[label*=\arabic*.]
			\item \textbf{\# alt1}: Se il piano associato all'utente non permette la modifica di playlist l'operazione fallisce e l'utente viene avvisato;
			\item \textbf{\# alt2}: Se il piano associato all'utente permette la modifica di playlist, il sistema  fornisce la lista delle playlist associate ad esso;
			\begin{enumerate}[label*=\arabic*.]
				\item L'utente sceglie la playlist da modificare dalla lista;
				\item Il sistema fornisce la lista di opzioni per visibilit\'a supportate dalla piattaforma;
				\item L'utente sceglie la visibilità e la comunica al sistema;
				\item Il sistema procede con l'aggiornamento dell'informaizone secondo le scelte fatte dall'utente.
			\end{enumerate}
					\end{enumerate}

		\end{enumerate}\\\hline
\end{tabular}
\label{table_use_case:39}\newline

\begin{tabular}{ |p{2cm}|p{13cm}|  }
\hline
ID & UC\_40\\\hline
Categoria & Prodotto \\\hline
Nome & UC\_RiproduciPlaylist\\\hline
Priorità & High \\\hline
Attori &  UtenteAutenticato \\\hline
Descrizione & Permette di riprodurre in ordine le risorse raccolte all'interno di una playlist\\\hline
Pre-condizioni & L'utente è registrato e autenticato correttamente\\\hline
Post-condizioni & Nessuna\\\hline
Flusso &  	\begin{enumerate}
		\item L'utente chiede al sistema di riprodurre una playlist;
		\item Il sistema chiede all'utente l'informazione riguardo la playlist da riprodurre;
		\item L'utente fornisce al sistema quale playlist riprodurre;
		\item Il sistema per ogni risorsa associata alla playlist scelta dall'utente esegue UC\_41
		\end{enumerate}\\\hline
\end{tabular}
\label{table_use_case:40}\newline
%=====================================

% ============================== USE CASE DA RIVEDERE ====================================

\textbf{USE CASE DA RIVEDERE} \newline


\end{table}

\begin{table}[bp]
    \centering
    \addtolength{\leftskip} {-2cm}
\begin{tabular}{ |p{2.6cm}|p{13cm}|  }
\hline
ID & UC\_14 \\\hline
Categoria & Risorse\\\hline
Nome & UC\_RimuoviRisorsa\\\hline
Priorità & High \\\hline
Attori &  Partner \\\hline
Descrizione & Permette a un utente di rimuoere dalla piattaforma una risorsa.\\\hline
Flusso &  	\begin{enumerate}
			\item L'utente richiede di rimuoere dalla piattaforma una risorsa;
			\item Il sistema verifica sulla base di dati che l'abbonamento sottoscritto dall'utente permetta l'operazione:
			\begin{enumerate}
				\item Se l'abbonamento include il servizio:
				\begin{enumerate}
					\item Il sistema richiede il Titolo della risorsa da caricare e informazioni varie come Descrizione Breve,  Descrizione Completa, Tipo, Categoria, Pubblico Consigliato e Collezione;
					\item L'utente fornisce i dati richiesti;
					\item Il sistema verifica sulla base di dati che l'utente abbia pubblicato una risorsa gli stessi dati:
					\begin{enumerate}
						\item Se il sistema risponde in maniera positiva, allora la risorsa viene rimossa;
						\item Se il sistema risponde in maniera negativa, allora  la risorsa non viene rimossa e l'utente viene avvisato del fallimento dell'operazione.
					\end{enumerate}
				\end{enumerate}
				\item Se l'abbonamento non include il servizio, allora la risorsa non viene rimossa e l'utente viene avvisato del fallimento dell'operazione.
			\end{enumerate}
		\end{enumerate}\\\hline
Pre-condizioni & L'utente è registrato sulla piattaforma, ha effettuato il Login ed ha pubblicato almeno una risorsa.\\\hline
Post-condizioni & La risorsa non è più accessibile da nessun servizio e risulta inoltre rimossa da ogni playlist .\\\hline
\end{tabular}
\label{table_use_case:14}\newline
\end{table}

\begin{table}[bp]
    \centering
    \addtolength{\leftskip} {-2cm}
\begin{tabular}{ |p{2.6cm}|p{13cm}|  }
\hline
ID & UC\_15 \\\hline
Categoria & Risorse\\\hline
Nome & UC\_ModificaRisorsa\\\hline
Priorità & High \\\hline
Attori &  UtenteAutenticato \\\hline
Descrizione & Permette a un utente di modificare una propria risorsa.\\\hline
Flusso &  	\begin{enumerate}
			\item L'utente richiede di modificare una risorsa;
			\item Il sistema verifica sulla base di dati che l'abbonamento sottoscritto dall'utente permetta l'operazione:
			\begin{enumerate}
				\item Se l'abbonamento include il servizio:
				\begin{enumerate}
					\item Il sistema richiede il Titolo della risorsa da caricare e informazioni varie come Descrizione Breve,  Descrizione Completa, Tipo, Categoria, Pubblico Consigliato e Collezione;
					\item L'utente fornisce i dati richiesti;
					\item Il sistema verifica sulla base di dati che l'utente abbia pubblicato una risorsa gli stessi dati:
					\begin{enumerate}
						\item Se il sistema risponde in maniera positiva:
							\subitem A.1. Il sistema chiede i nuovi valori dei dati;
							\subitem A.2. L'utente fornisce i dati richiesti;
							\subitem A.3. Il sistema aggiorna i dati della risorsa.
						\item Se il sistema risponde in maniera negativa, allora  la risorsa non viene modificata e l'utente viene avvisato del fallimento dell'operazione.
					\end{enumerate}
				\end{enumerate}
				\item Se l'abbonamento non include il servizio, allora la risorsa non viene modificata e l'utente viene avvisato del fallimento dell'operazione.
			\end{enumerate}
		\end{enumerate}\\\hline
Pre-condizioni & L'utente è registrato sulla piattaforma, ha effettuato il Login, possiede un abbonamento che permetta la pubblicazione di risorse ed ha pubblicato almeno una risorsa.\\\hline
Post-condizioni & La risorsa è aggiornata con i nuovi dati.\\\hline
\end{tabular}
\label{table_use_case:15}\newline
\end{table}

%DA PENSARE BENE SE E COME METTERE I SEGUENTI USE-CASE: %
%NUOVO SERVIZIO - ELIMINA SERVIZIO - DOWNLOAD RISORSA%

\end{center}