\setlength{\arrayrulewidth}{.5mm}
\setlength{\tabcolsep}{5pt}
\renewcommand{\arraystretch}{2}
\renewcommand{\labelenumii}{\theenumii}
\renewcommand{\theenumii}{\theenumi.\arabic{enumii}.}

\subsection{Introduzione}
La seguente sezione mira all'identificazione dei casi d'uso presenti nel progetto. Di seguito vengono identificati oltre ai casi d'uso, anche gli attori coinvolti nel loro utilizzo.
Lo scopo principale del documento, è quello di avere un riferimento sui casi d'uso leggibile da chiunque, anche a personale non interno al progetto. Questo è utile al fine di permettere a tutti una comprensione e quindi una discussione sui casi d'uso.

\subsection{Legenda}
\large{\textbf{Attori}} \\
\begin{itemize}[]
	\item \textbf{ID}: rappresenta l'identificatore univoco di un attore. La sintassi è del tipo A\_X dove X è stringa di interi separati da punto che rappresenta la gerarchia di relazioni padre-figlio.
	\item \textbf{Nome}: nome dell'attore, deve essere comunque univoco e deve dare una prima idea di cosa rappresenta quel preciso attore.
	\item \textbf{Genitore}: indica l'identificatore del primo antenato nella gerarchia degli attori.
	\item \textbf{Livello}: indica il grado di rilevanza dell'attore all'interno dell'intero sistema. Può assumere il valore di {Primario, Secondario, Di Supporto}.
	\item \textbf{Tipologia}: classifica l'attore in {Umano, Sistema}
	\item \textbf{Descrizione}: descrive brevemente cosa rappresenta all'interno del sistema l'attore.\\

\end{itemize}


\noindent \large{\textbf{Use Case}} \\
\begin{itemize}[]
	\item \textbf{ID}: identificativo univoco del caso d'uso, è della forma UC\_X dove X è stringa di interi separati da punto che rappresenta la gerarchia di relazioni padre-figlio.
	\item \textbf{Categoria}: Macroarea di appartenza della funzionalità. Può assumere i valori {User, Subscriptions, Management}
	\item \textbf{Nome}: Nome identificativo del caso d'uso, serve anche per dare un'idea del tipo di attività.
	\item \textbf{Priorità}: Livello di priorità del caso d'uso. Può assumere i valori {High, Medium, Low}. 
	\item \textbf{Attori}: Lista di identificativi degli attori coinvolti nel caso d'uso.
	\item \textbf{Descrizione}: Breve descrizione della funzione svolta dal caso d'uso.
	\item \textbf{Flusso}: Passi logici eseguiti dal caso d'uso per ottenere il risultato desiderato.
	\item \textbf{Pre-condizioni}: Condizioni che devono essere verificate prima che il caso d'uso inizi, altrimenti non potrà essere svolto.
	\item \textbf{Post-condizioni}: Condizioni che devono essere verificate subito dopo l'esecuzione del caso d'uso.
	%\item \textbf{Flusso Alternativo}: 

\end{itemize}

% ============================== ATTORI ====================================

\subsection{Specifica Attori}

\begin{center}

\begin{tabular}{ |p{2cm}|p{10cm}|  }
\hline
ID & A\_1 \\\hline
Nome & Utente\\\hline
Genitore & - \\\hline
Livello &  Primario \\\hline
Tipologia & Umano \\\hline
Descrizione &  L' Utente è il ruolo assunto dagli utentei che usufruiranno del servizio di streaming offerto dalla piattaforma. \\\hline
\end{tabular}
\label{table_attore:1}\newline
%\label{table_attore:A\_1}\newline

\begin{tabular}{ |p{2cm}|p{10cm}|  }
\hline
ID & A\_1.1 \\\hline
Nome & UtenteNonRegistrato\\\hline
Genitore & A\_1 \\\hline
Livello &  Primario \\\hline
Tipologia & Umano \\\hline
Descrizione &  questo ruolo è assunto da tutti gli utenti che utilizzano la piattaforma senza aver effettuato l'accesso ad un profilo utente e dunque avrà accesso solo a determinate funzionalità \\\hline
\end{tabular}
\label{table_attore:1.1}\newline
%\label{table_attore:A\_1.1}\newline

\begin{tabular}{ |p{2cm}|p{10cm}|  }
\hline
ID & A\_1.2 \\\hline
Nome & UtenteRegistrato\\\hline
Genitore & A\_1 \\\hline
Livello &  Primario \\\hline
Tipologia & Umano \\\hline
Descrizione &  questo ruolo è assunto da tutti gli utenti che hanno sottoscritto un contratto con la piattaforma e si sono autenticati in fase di accesso ad essa. \\\hline
\end{tabular}
\label{table_attore:1.2}\newline
%\label{table_attore:A\_1.2}\newline

\begin{tabular}{ |p{2cm}|p{10cm}|  }
\hline
ID & A\_1.3 \\\hline
Nome & Amministratore\\\hline
Genitore & A\_1\\\hline
Livello &  Primario \\\hline
Tipologia & Umano \\\hline
Descrizione &  Rappresenta il ruolo di un amministratore della piattaforma \\\hline
\end{tabular}
\label{table_attore:1.3}\newline
%\label{table_attore:A\_1.3}\newline

\begin{tabular}{ |p{2cm}|p{10cm}|  }
\hline
ID & A\_2 \\\hline
Nome & Pagamento\\\hline
Genitore & - \\\hline
Livello &  Primario \\\hline
Tipologia & Sistema \\\hline
Descrizione &  rappresenta il sistema di pagamento esterno al sistema della piattaforma \\\hline
\end{tabular}
\label{table_attore:2}\newline
%\label{table_attore:A\_2}\newline

% ============================== USE CASE ====================================

\subsection{Specifica Use Case}

\begin{tabular}{ |p{2cm}|p{13cm}|  }
\hline
ID & UC\_1 \\\hline
Categoria & User \\\hline
Nome & UC\_EffettuaRegistrazione \\\hline
Priorità & High \\\hline
Attori &  ---- \\\hline
Descrizione & registrazione persistente di un itente all'interno della piattaforma \\\hline
Flusso &  	\begin{enumerate}
			\item l'utente richiede di registrarsi alla piattaforma,
			\item Il sistema richiede all' utente non registrato le credenziali di accesso quali: email, password, conferma della password, tipo di abbonamento da sottoscrivere.
			\item l'utente non registrato riempie i campi richiesti e li invia al sistema.
			\item il sistema verifica la correttezza dei dati e richiede il metodo di pagamento desiderato tra quelli disponibili.
			\item l'utente seleziona il metodo di pagamento desiderato.
			\item il sistema verifica che il metodo selezionato sia valido e reinderizza l'utente al sistema esterno di pagamento, e attende risposta dal sistema esterno:
			\begin{enumerate}[  ]
				\item \textbf{\# alt1}: se il sistema risponde in maniera positiva: il pagamento è stato effettuato e l'utente viene registrato sulla piattaforma 
				\item \textbf{\# alt2}: se il sistema risponde in maniera negativa: il pagamento non è stato effettuato e si chiede all'utente di riprovare o rinunciare alla registrazione.
			\end{enumerate}
		\end{enumerate}\\\hline
Pre-condizioni &  Non esiste un utente registrato con la stessa email dell'utente che avvia la registrazione\\\hline
Post-condizioni &  L'utente che ha richiesto la registrazione è registrato persistentemente, ed il suo account è identificato dalla email formita nel modulo di registrazione.\\\hline
\end{tabular}
\label{table_use_case:1}\newline
%\label{table_attore:A\_1}\newline

\begin{tabular}{ |p{2cm}|p{13cm}|  }
\hline
ID & UC\_2 \\\hline
Categoria & User \\\hline
Nome & UC\_EffettuaLogin \\\hline
Priorità & High \\\hline
Attori &  ---- \\\hline
Descrizione & Accesso al sistema identificandosi tramite credenziali \\\hline
Flusso &  	\begin{enumerate}
			\item l'utente richiede di accedere alla piattaforma,
			\item Il sistema richiede all' utente le credenziali di accesso quali: email e password.
			\item l'utente riempie i campi richiesti e li invia al sistema.
			\item il sistema valida i dati e avvia la procedura di autenticazione interfacciandosi con la base di dati degli utenti.
			\begin{enumerate}[  ]
				\item \textbf{\# alt1}: se il sistema risponde in maniera positiva: l'utente è autenticato ed abilitato all'utilizzo della piattaforma 
				\item \textbf{\# alt2}: se il sistema risponde in maniera negativa: l'utente non viene autenticato ed è reindirizzato alla pagina di login.
			\end{enumerate}
		\end{enumerate}\\\hline
Pre-condizioni &  L'utente esiste ed non è già autenticato\\\hline
Post-condizioni &  L'utente è autenticato ed abilitato alle funzionalità che gli spettano secondo il contratto sottoscritto.\\\hline
\end{tabular}
\label{table_use_case:2}\newline
%\label{table_attore:A\_1}\newline

\begin{tabular}{ |p{2cm}|p{13cm}|  }
\hline
ID & UC\_3 \\\hline
Categoria & User \\\hline
Nome & UC\_RicercaRisorsa\\\hline
Priorità & High \\\hline
Attori &  ---- \\\hline
Descrizione & fornisce una lista filtrata di risorse disponibili. Le risorse vengono filtrate in base alla coerenza con la stringa di ricerca fornita.\\\hline
Flusso &  	\begin{enumerate}
			\item l'utente richiede di ricercare una risorsa,
			\item Il sistema richiede all' utente la stringa di ricerca.
			\item l'utente fornisce al sistema la stringa.
			\item il sistema valida l'input e avvia la procedura di ricerca delle risorse sui server e la restituisce all'utente %%%.
			% da scrivere passaggi cdn 
			
		\end{enumerate}\\\hline
Pre-condizioni &  Nessuna\\\hline
Post-condizioni &  Nessuna\\\hline
\end{tabular}
\label{table_use_case:3}\newline
%\label{table_attore:A\_1}\newline

\begin{tabular}{ |p{2cm}|p{13cm}|  }
\hline
ID & UC\_4 \\\hline
Categoria & Management\\\hline
Nome & UC\_CreaAbbonamento\\\hline
Priorità & High \\\hline
Attori &  ---- \\\hline
Descrizione & crea un nuovo ruolo di abbonamento, inizializzandolo con nessuna funzionalità abbinata.\\\hline
Flusso &  	\begin{enumerate}
			\item l'amministratore richiede la creazione di un nuovo abbonamento,
			\item Il sistema richiede all' amministratore il nomedel ruolo.
			\item l'amministratore fornisce al sistema la stringa.
			\item il sistema valida l'input e verifica che non esista già un ruolo con lo stesso nome e:
				\begin{enumerate}[  ]
					\item \textbf{\# alt1}: se il sistema risponde in maniera positiva: il nuovo ruolo di abbonamento è stato creato e disponibile all'utilizzo.
					\item \textbf{\# alt2}: se il sistema risponde in maniera negativa: il nuovo ruolo di abbonamento non è stato creato e dunque non utilizzabile sulla piattaforma.
				\end{enumerate}
			
		\end{enumerate}\\\hline
Pre-condizioni &  Non già un abbonamento con lo stesso nome di quello da creare\\\hline
Post-condizioni &  Esiste un nuovo abbonamento con il nome richiesto\\\hline
\end{tabular}
\label{table_use_case:4}\newline
%\label{table_attore:A\_1}\newline

\begin{tabular}{ |p{2cm}|p{13cm}|  }
\hline
ID & UC\_5 \\\hline
Categoria & Management\\\hline
Nome & UC\_EliminaAbbonamento\\\hline
Priorità & High \\\hline
Attori &  ---- \\\hline
Descrizione & rende non piu sottoscrivibile un ruolo di abbonamento esistente.\\\hline
Flusso &  	\begin{enumerate}
			\item l'amministratore richiede di invalidare un abbonamento.
			\item Il sistema richiede all' amministratore il nome dell' abbonamento.
			\item l'amministratore fornisce al sistema la stringa.
			\item il sistema valida l'input, verifica che esista un ruolo con quel nome e che non ci sia un'utente attivo con quell'abbonamento in corso di validità:
				\begin{enumerate}[  ]
					\item \textbf{\# alt1}: queste condizioni sono verificate: il ruolo di abbonamento è non sarà piu sottoscrivibile da quel momento in pioi.
					\item \textbf{\# alt2}: queste condizioni non sono verificate: l'abbonamento rimane valido e l'amministratore viene informato del fallimento dell'invalidazione.
				\end{enumerate}
		\end{enumerate}\\\hline
Pre-condizioni &  Il ruolo da eliminare è esistente\\\hline
Post-condizioni &  Il ruolo con nome specificato non è piu sottoscrivibile\\\hline
\end{tabular}
\label{table_use_case:5}\newline
%\label{table_attore:A\_1}\newline

\begin{tabular}{ |p{2cm}|p{13cm}|  }
\hline
ID & UC\_6 \\\hline
Categoria & Management\\\hline
Nome & UC\_AggiungiFunzioneAbbonamento\\\hline
Priorità & High \\\hline
Attori &  ---- \\\hline
Descrizione & permette di estendere di un elemento l'insieme delle funzioni collegate ad un abbonamento con una funzionalità esistente.\\\hline
Flusso &  	\begin{enumerate}
			\item l'amministratore richiede di estendere le funzionalità di  un abbonamento.
			\item Il sistema richiede all' amministratore il nome dell' abbonamento ed un elemento della lista delle funzionalità aggiungibili..
			\begin{enumerate}
				\item (UC\_6.1) Il sistema ricerca all'interno della base di dati tutti i nomi di funzionalità disponibili.
			\end{enumerate}
			\item l'amministratore fornisce al sistema la stringa del nome e della funzionalità da aggiungere.
			\item il sistema valida l'input e memorizza l'associazione tra l'abbonamento e la funzionalitò scelta.
		\end{enumerate}\\\hline
Pre-condizioni &  l'abbonamento scelto e la funzionalità esistono.\newline 
			L'abbonamento non è invalidato.\\\hline
Post-condizioni &  L'abbonamento e la funzionalità sono associati.\\\hline
\end{tabular}
\label{table_use_case:6}\newline
%\label{table_attore:A\_1}\newline

\begin{tabular}{ |p{2cm}|p{13cm}|  }
\hline
ID & UC\_6.1 \\\hline
Categoria & Management\\\hline
Nome & UC\_RecuperaFunzioniAbbonamento\\\hline
Priorità & High \\\hline
Attori &  ---- \\\hline
Descrizione & permette dato un nome di abbonamento valido, di reperire tutte le funzionalitò ad esso associate.\\\hline
Flusso &  	\begin{enumerate}
			\item l'amministratore richiede di recuperare le funzionalità di un abbonamento.
			\item Il sistema richiede all' amministratore il nome dell' abbonamento.
			\item l'amministratore fornisce al sistema la stringa del nome.
			\item il sistema valida l'input e ricerca tutte le funzionalità associate all'abbonamento sulla base di dati.
		\end{enumerate}\\\hline
Pre-condizioni &  l'abbonamento scelto esiste.\\\hline
Post-condizioni &  Nessuna.\\\hline
\end{tabular}
\label{table_use_case:6.1}\newline
%\label{table_attore:A\_1}\newline

\begin{tabular}{ |p{2cm}|p{13cm}|  }
\hline
ID & UC\_7 \\\hline
Categoria & Management\\\hline
Nome & UC\_RimuoviFunzioneAbbonamento\\\hline
Priorità & High \\\hline
Attori &  ---- \\\hline
Descrizione & permette di ridurre di un elemento l'insieme delle funzioni collegate ad un abbonamento, di una funzionalità già associata ad esso.\\\hline
Flusso &  	\begin{enumerate}
			\item l'amministratore richiede di ridurre le funzionalità di  un abbonamento.
			\item Il sistema richiede all' amministratore il nome dell' abbonamento ed un elemento della lista delle funzionalità associate all'abbonamento.
			\begin{enumerate}
				\item (UC\_7.1) Il sistema ricerca all'interno della base di dati tutti i nomi di funzionalità associati all'abbonamento specificato.
			\end{enumerate}
			\item l'amministratore fornisce al sistema la stringa del nome e della funzionalità da rimuovere.
			\item il sistema valida l'input e rimuove l'associazione tra l'abbonamento e la funzionalitò scelta dalla base di dati.
		\end{enumerate}\\\hline
Pre-condizioni &  l'abbonamento scelto e la funzionalità esistono e sono associati.\newline 
			L'abbonamento non è invalidato.\\\hline
Post-condizioni &  L'abbonamento e la funzionalità non sono associati.\\\hline
\end{tabular}
\label{table_use_case:7}\newline
%\label{table_attore:A\_1}\newline

\begin{tabular}{ |p{2cm}|p{13cm}|  }
\hline
ID & UC\_7.1 \\\hline
Categoria & Management\\\hline
Nome & UC\_RecuperaFunzioniAbbonamenti\\\hline
Priorità & High \\\hline
Attori &  ---- \\\hline
Descrizione & permette di recuperare la lista di tutte le funzionalità associabili ad un abbonamento.\\\hline
Flusso &  	\begin{enumerate}
			\item l'amministratore richiede di recuperare le funzionalità degli abbonamenti.
			\item il sistema ricerca tutte le funzionalità sulla base di dati.
		\end{enumerate}\\\hline
Pre-condizioni &  Nessuna.\\\hline
Post-condizioni &  Nessuna.\\\hline
\end{tabular}
\label{table_use_case:7.1}\newline
%\label{table_attore:A\_1}\newline

\end{center}