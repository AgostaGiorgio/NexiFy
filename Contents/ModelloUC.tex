\setlength{\arrayrulewidth}{.5mm}
\setlength{\tabcolsep}{5pt}
\renewcommand{\arraystretch}{2}
\renewcommand{\labelenumii}{\theenumii}
\renewcommand{\theenumii}{\theenumi.\arabic{enumii}.}

\subsection{Introduzione}
La seguente sezione mira all'identificazione dei casi d'uso presenti nel progetto. Di seguito vengono identificati oltre ai casi d'uso, anche gli attori coinvolti nel loro utilizzo.
Lo scopo principale del documento, è quello di avere un riferimento sui casi d'uso leggibile da chiunque, anche a personale non interno al progetto. Questo è utile al fine di permettere a tutti una comprensione e quindi una discussione sui casi d'uso.

\subsection{Legenda}
\large{\textbf{Attori}} \\
\begin{itemize}[]
	\item \textbf{ID}: rappresenta l'identificatore univoco di un attore. La sintassi è del tipo A\_X dove X è una stringa di interi separati da punto che rappresenta la gerarchia di relazioni padre-figlio.
	\item \textbf{Nome}: nome dell'attore, deve essere comunque univoco e deve dare una prima idea di cosa rappresenta quel preciso attore.
	\item \textbf{Genitore}: indica l'identificatore del primo antenato nella gerarchia degli attori.
	\item \textbf{Livello}: indica il grado di rilevanza dell'attore all'interno dell'intero sistema. Può assumere il valore di {Primario, Secondario, Di Supporto}.
	\item \textbf{Tipologia}: classifica l'attore in {Umano, Sistema}
	\item \textbf{Descrizione}: descrive brevemente cosa rappresenta all'interno del sistema l'attore.\\

\end{itemize}


\noindent \large{\textbf{Use Case}} \\
\begin{itemize}[]
	\item \textbf{ID}: identificativo univoco del caso d'uso, è della forma UC\_X dove X è una stringa di interi separati da punto che rappresenta la gerarchia di relazioni padre-figlio.
	\item \textbf{Categoria}: Macroarea di appartenza della funzionalità. Può assumere i valori {User, Subscriptions, Management}
	\item \textbf{Nome}: Nome identificativo del caso d'uso, serve anche per dare un'idea del tipo di attività.
	\item \textbf{Priorità}: Livello di priorità del caso d'uso. Può assumere i valori {High, Medium, Low}. 
	\item \textbf{Attori}: Lista di identificativi degli attori coinvolti nel caso d'uso.
	\item \textbf{Descrizione}: Breve descrizione della funzione svolta dal caso d'uso.
	\item \textbf{Flusso}: Passi logici eseguiti dal caso d'uso per ottenere il risultato desiderato.
	\item \textbf{Pre-condizioni}: Condizioni che devono essere verificate prima che il caso d'uso inizi, altrimenti non potrà essere svolto.
	\item \textbf{Post-condizioni}: Condizioni che devono essere verificate subito dopo l'esecuzione del caso d'uso.
	%\item \textbf{Flusso Alternativo}: 

\end{itemize}

% ============================== ATTORI ====================================

\subsection{Specifica Attori}

\begin{center}

\begin{tabular}{ |p{2cm}|p{10cm}|  }
\hline
ID & A\_1 \\\hline
Nome & Utente\\\hline
Genitore & - \\\hline
Livello &  Primario \\\hline
Tipologia & Umano \\\hline
Descrizione &  L' Utente è il ruolo assunto da tutti coloro che usufruiranno del servizio di streaming offerto dalla piattaforma. \\\hline
\end{tabular}
\label{table_attore:1}\newline

\begin{tabular}{ |p{2cm}|p{10cm}|  }
\hline
ID & A\_1.1 \\\hline
Nome & UtenteNonRegistrato\\\hline
Genitore & A\_1 \\\hline
Livello &  Primario \\\hline
Tipologia & Umano \\\hline
Descrizione &  Questo ruolo è assunto da tutti coloro che utilizzano la piattaforma senza aver effettuato l'accesso ad un profilo utente e che quindi avranno accesso solo a determinate funzionalità \\\hline
\end{tabular}
\label{table_attore:1.1}\newline

\begin{tabular}{ |p{2cm}|p{10cm}|  }
\hline
ID & A\_1.2 \\\hline
Nome & UtenteRegistrato\\\hline
Genitore & A\_1 \\\hline
Livello &  Primario \\\hline
Tipologia & Umano \\\hline
Descrizione &  Questo ruolo è assunto da tutti gli utenti che hanno sottoscritto un contratto con la piattaforma e si sono autenticati in fase di accesso ad essa. \\\hline
\end{tabular}
\label{table_attore:1.2}\newline

\begin{tabular}{ |p{2cm}|p{10cm}|  }
\hline
ID & A\_1.3 \\\hline
Nome & Amministratore\\\hline
Genitore & A\_1\\\hline
Livello &  Primario \\\hline
Tipologia & Umano \\\hline
Descrizione &  Rappresenta il ruolo di un amministratore della piattaforma \\\hline
\end{tabular}
\label{table_attore:1.3}\newline

\begin{tabular}{ |p{2cm}|p{10cm}|  }
\hline
ID & A\_2 \\\hline
Nome & Pagamento\\\hline
Genitore & - \\\hline
Livello &  Primario \\\hline
Tipologia & Sistema \\\hline
Descrizione &  Rappresenta il sistema di pagamento esterno al sistema della piattaforma \\\hline
\end{tabular}
\label{table_attore:2}\newline

% ============================== USE CASE ====================================

\subsection{Specifica Use Case}

\begin{tabular}{ |p{2cm}|p{13cm}|  }
\hline
ID & UC\_1 \\\hline
Categoria & User \\\hline
Nome & UC\_EffettuaRegistrazione \\\hline
Priorità & High \\\hline
Attori &  ---- \\\hline
Descrizione & Registrazione persistente di un itente all'interno della piattaforma \\\hline
Flusso &  	\begin{enumerate}
			\item L'utente richiede di registrarsi alla piattaforma;
			\item Il sistema richiede all' utente non registrato le credenziali di accesso quali: email, password, conferma della password, tipo di abbonamento da sottoscrivere;
			\item L'utente non registrato riempie i campi richiesti e li invia al sistema;
			\item Il sistema verifica la correttezza dei dati e richiede il metodo di pagamento desiderato tra quelli disponibili;
			\item L'utente seleziona il metodo di pagamento desiderato;
			\item Il sistema verifica che il metodo selezionato sia valido e reinderizza l'utente al sistema esterno di pagamento, e attende risposta dal sistema esterno:
			\begin{enumerate}[  ]
				\item \textbf{\# alt1}: se il sistema risponde in maniera positiva: il pagamento è stato effettuato e l'utente viene registrato sulla piattaforma 
				\item \textbf{\# alt2}: se il sistema risponde in maniera negativa: il pagamento non è stato effettuato e si chiede all'utente di riprovare o rinunciare alla registrazione.
			\end{enumerate}
		\end{enumerate}\\\hline
Pre-condizioni &  Non esiste un utente registrato con la stessa email dell'utente che avvia la registrazione\\\hline
Post-condizioni &  L'utente che ha richiesto la registrazione è registrato persistentemente, ed il suo account, identificato dalla email formita nel modulo di registrazione, è abbinato al pacchetto di servizi scelto.\\\hline
\end{tabular}
\label{table_use_case:1}\newline

\begin{tabular}{ |p{2cm}|p{13cm}|  }
\hline
ID & UC\_2 \\\hline
Categoria & User \\\hline
Nome & UC\_EffettuaLogin \\\hline
Priorità & High \\\hline
Attori &  ---- \\\hline
Descrizione & Accesso al sistema identificandosi tramite credenziali \\\hline
Flusso &  	\begin{enumerate}
			\item L'utente richiede di accedere alla piattaforma;
			\item Il sistema richiede all' utente le credenziali di accesso quali: email e password;
			\item L'utente riempie i campi richiesti e li invia al sistema;
			\item Il sistema valida i dati e avvia la procedura di autenticazione interfacciandosi con la base di dati degli utenti:
			\begin{enumerate}[  ]
				\item \textbf{\# alt1}: se il sistema risponde in maniera positiva: l'utente è autenticato ed abilitato all'utilizzo dei servizi offerti dall'abbonamento in corso; 
				\item \textbf{\# alt2}: se il sistema risponde in maniera negativa: l'utente non viene autenticato ed è reindirizzato alla pagina di login.
			\end{enumerate}
		\end{enumerate}\\\hline
Pre-condizioni &  L'utente esiste e non è già autenticato\\\hline
Post-condizioni &  L'utente è autenticato ed abilitato ai servizi che gli spettano secondo il contratto sottoscritto.\\\hline
\end{tabular}
\label{table_use_case:2}\newline

\begin{tabular}{ |p{2cm}|p{13cm}|  }
\hline
ID & UC\_3 \\\hline
Categoria & User \\\hline
Nome & UC\_RicercaRisorsa\\\hline
Priorità & High \\\hline
Attori &  ---- \\\hline
Descrizione & Fornisce una lista filtrata di risorse disponibili. Le risorse vengono filtrate in base alla coerenza con la stringa di ricerca fornita.\\\hline
Flusso &  	\begin{enumerate}
			\item L'utente richiede di ricercare una risorsa;
			\item Il sistema richiede all' utente la stringa di ricerca;
			\item L'utente fornisce al sistema la stringa;
			\item Il sistema valida l'input e avvia la procedura di ricerca delle risorse sui server e la restituisce all'utente. %%%.
			% da scrivere passaggi cdn 
			
		\end{enumerate}\\\hline
Pre-condizioni &  Nessuna\\\hline
Post-condizioni &  Nessuna\\\hline
\end{tabular}
\label{table_use_case:3}\newline

\begin{tabular}{ |p{2cm}|p{13cm}|  }
\hline
ID & UC\_4 \\\hline
Categoria & Management\\\hline
Nome & UC\_CreaAbbonamento\\\hline
Priorità & High \\\hline
Attori &  ---- \\\hline
Descrizione & Crea un nuovo abbonamento senza nessun servizio usufruibile da questo.\\\hline
Flusso &  	\begin{enumerate}
			\item L'amministratore richiede la creazione di un nuovo abbonamento;
			\item Il sistema richiede all' amministratore il nome del nuovo abbonamento;
			\item L'amministratore fornisce al sistema la stringa;
			\item Il sistema valida l'input e verifica che non esista già un abbonamento con lo stesso nome e:
				\begin{enumerate}[  ]
					\item \textbf{\# alt1}: se il sistema risponde in maniera positiva: il nuovo abbonamento è stato creato e disponibile all'utilizzo.
					\item \textbf{\# alt2}: se il sistema risponde in maniera negativa: il nuovo abbonamento non è stato creato e dunque non utilizzabile sulla piattaforma.
				\end{enumerate}
			
		\end{enumerate}\\\hline
Pre-condizioni &  Non esiste già un abbonamento con lo stesso nome di quello da creare\\\hline
Post-condizioni &  Esiste un nuovo abbonamento con il nome fornito\\\hline
\end{tabular}
\label{table_use_case:4}\newline

\begin{tabular}{ |p{2cm}|p{13cm}|  }
\hline
ID & UC\_5 \\\hline
Categoria & Management\\\hline
Nome & UC\_EliminaAbbonamento\\\hline
Priorità & High \\\hline
Attori &  ---- \\\hline
Descrizione & Rende non più sottoscrivibile un abbonamento esistente.\\\hline
Flusso &  	\begin{enumerate}
			\item L'amministratore richiede di invalidare un abbonamento;
			\item Il sistema richiede all' amministratore il nome dell' abbonamento;
			\item L'amministratore fornisce al sistema la stringa;
			\item Il sistema valida l'input, verifica che esista un abbonamento con quel nome e che non ci sia un'utente attivo con quell'abbonamento in corso di validità:
				\begin{enumerate}[  ]
					\item \textbf{\# alt1}: se queste condizioni sono verificate l' abbonamento non sarà piu sottoscrivibile da quel momento in poi.
					\item \textbf{\# alt2}: se queste condizioni non sono verificate l'abbonamento rimane valido e l'amministratore viene informato del fallimento dell'invalidazione.
%DA RIVEDERE-- MAGARI E' MEGLIO NON RENDERLO PIU SOTTOSCRIVIBILE E A QUELLI CHE GIA LO HANNO GLI SI ANNULLA IL RINNOVO AUTOMATICO%
				\end{enumerate}
		\end{enumerate}\\\hline
Pre-condizioni &  L'abbonamento da eliminare è presente sul sistema\\\hline
Post-condizioni &   L'abbonamento con nome specificato non è piu sottoscrivibile\\\hline
\end{tabular}
\label{table_use_case:5}\newline

\begin{tabular}{ |p{2cm}|p{13cm}|  }
\hline
ID & UC\_6 \\\hline
Categoria & Management\\\hline
Nome & UC\_AggiungiServizioAbbonamento\\\hline
Priorità & High \\\hline
Attori &  ---- \\\hline
Descrizione & Permette di estendere di un elemento l'insieme dei servizi collegati ad un abbonamento.\\\hline
Flusso &  	\begin{enumerate}
			\item L'amministratore richiede di estendere i servizi offerti da un abbonamento;
			\item Il sistema richiede all' amministratore il nome dell' abbonamento ed un elemento della lista dei servizi aggiungibili..
			\begin{enumerate}
				\item (UC\_7.1) Il sistema ricerca all'interno della base di dati tutti i nomi dei servizi disponibili.
			\end{enumerate}
			\item L'amministratore fornisce al sistema il nome dell'abbonamento e del servizio da aggiungere;
			\item Il sistema valida l'input e memorizza l'associazione tra l'abbonamento e il servizio scelto.
		\end{enumerate}\\\hline
Pre-condizioni & L'abbonamento scelto e il servizio esistono;\newline 
			L'abbonamento non è invalidato.\\\hline
Post-condizioni &  L'abbonamento e il servizio sono associati.\\\hline
\end{tabular}
\label{table_use_case:6}\newline

\begin{tabular}{ |p{2cm}|p{13cm}|  }
\hline
ID & UC\_6.1 \\\hline
Categoria & Management\\\hline
Nome & UC\_RecuperaServizi\\\hline
Priorità & High \\\hline
Attori &  ---- \\\hline
Descrizione & Permette di recuperare la lista di tutti i servizi associabili ad un qualsiasi abbonamento.\\\hline
Flusso &  	\begin{enumerate}
			\item L'amministratore richiede di recuperare i servizi disponibili.
			\item Il sistema ricerca tutti i servizi sulla base di dati.
		\end{enumerate}\\\hline
Pre-condizioni &  Nessuna.\\\hline
Post-condizioni &  Nessuna.\\\hline
\end{tabular}
\label{table_use_case:6.1}\newline

\begin{tabular}{ |p{2cm}|p{13cm}|  }
\hline
ID & UC\_7 \\\hline
Categoria & Management\\\hline
Nome & UC\_RimuoviServizioAbbonamento\\\hline
Priorità & High \\\hline
Attori &  ---- \\\hline
Descrizione & Permette la rimozione di uno dei servizi collegati ad un abbonamento.\\\hline
Flusso &  	\begin{enumerate}
			\item L'amministratore richiede di ridurre i servizi offerti da  un abbonamento.
			\item Il sistema richiede all' amministratore il nome dell' abbonamento e un elemento della lista dei servizi associati all'abbonamento.
			\begin{enumerate}
				\item (UC\_6.1) Il sistema ricerca all'interno della base di dati tutti i nomi di servizi associati all'abbonamento specificato.
			\end{enumerate}
			\item L'amministratore fornisce al sistema il nome dell'abbonamento e del servizio da rimuovere.
			\item Il sistema valida l'input e rimuove l'associazione tra l'abbonamento e il servizio scelto.
		\end{enumerate}\\\hline
Pre-condizioni &  L'abbonamento scelto e il servizio esistono e sono associati.\newline 
			L'abbonamento è validato.\\\hline
Post-condizioni &  L'abbonamento e il servizio non sono associati.\\\hline
\end{tabular}
\label{table_use_case:7}\newline

\begin{tabular}{ |p{2cm}|p{13cm}|  }
\hline
ID & UC\_7.1 \\\hline
Categoria & Management\\\hline
Nome & UC\_RecuperaServiziAbbonamento\\\hline
Priorità & High \\\hline
Attori &  ---- \\\hline
Descrizione & Permette, dato un nome di  un abbonamento, di reperire tutte i servizi usufruibili da esso.\\\hline
Flusso &  	\begin{enumerate}
			\item L'amministratore richiede di recuperare i servizi di un abbonamento;
			\item Il sistema richiede all' amministratore il nome dell' abbonamento;
			\item L'amministratore fornisce al sistema la stringa;
			\item Il sistema valida l'input e ricerca tutte i servizi associati all'abbonamento sulla base di dati;
		\end{enumerate}\\\hline
Pre-condizioni &  L'abbonamento scelto esiste.\\\hline
Post-condizioni &  Nessuna.\\\hline
\end{tabular}
\label{table_use_case:7.1}\newline

\begin{tabular}{ |p{2cm}|p{13cm}|  }
\hline
ID & UC\_8 \\\hline
Categoria & User\\\hline
Nome & UC\_SottoscriviAbbonamento\\\hline
Priorità & High \\\hline
Attori &  ---- \\\hline
Descrizione & Permette a un utente di sottoscrivere un nuovo abbonamento con la piattaforma.\\\hline
Flusso &  	\begin{enumerate}
			\item L'utente richiede di sottoscrivere un piano di abbonamento;
			\item Il sistema richiede il nome dell'abbonamento da sottoscrivere;
			\item L'utente fornisce la stringa;
			\item Il sistema valida l'input, verifica che esista un abbonamento sottoscrivibile con quel nome :
			\begin{enumerate}[  ]
				\item \textbf{\# alt1}: se queste condizioni sono verificate l'abbonamento verrà associato all'account dell'utente.
				\item \textbf{\# alt2}: se queste condizioni non sono verificate non viene portata nessuna variazione nello stato dell'account dell'utente e questo viene avvisato del fallimento dell'operazione.
			\end{enumerate}
		\end{enumerate}\\\hline
Pre-condizioni & L'utente è registrato sulla piattaforma e ha effettuato con successo il Login.\\\hline
Post-condizioni &  L'utente ha accesso a tutti i servizi offerti dal nuovo abbonamento sottoscritto.\\\hline
\end{tabular}
\label{table_use_case:8}\newline
%DA DECIDERE BENE COME COMPORTARSI SE L'UTENTE HA GIA' IN CORSO UN ABBONAMENTO: DOPO LA SCADENZA DEL VECCHIO PARTE IL NUOVO ?%

\begin{tabular}{ |p{2cm}|p{13cm}|  }
\hline
ID & UC\_9 \\\hline
Categoria & User\\\hline
Nome & UC\_DisdiciAbbonamento\\\hline
Priorità & High \\\hline
Attori &  ---- \\\hline
Descrizione & Permette a un utente di disdire l'abbonamento sottoscritto, evitando che questo si rinnovi automaticamente nella data prevista.\\\hline
Flusso &  	\begin{enumerate}
			\item L'utente richiede di disdire il piano di abbonamento a cui è associato;
			\item Il sistema verifica sulla base di dati lo stato del rinnovo automatico dell'abbonamento per l'utente in questione:
			\begin{enumerate}[  ]
				\item \textbf{\# alt1}: se il rinnovo automatico è attivo, lo disasttiva e avvisa l'utente che la modifica è stata apportata correttamente;
				\item \textbf{\# alt2}: se il rinnovo automatico non è attivo avvisa l'utente che l'abbonamento era già stato disdetto;
			\end{enumerate}
		\end{enumerate}\\\hline
Pre-condizioni & L'utente è registrato sulla piattaforma, ha effettuato con successo il Login.\\\hline
Post-condizioni &  L'utente ha accesso a tutti i servizi offerti dal piano di abbonamento basilare ( Free ).\\\hline
\end{tabular}
\label{table_use_case:9}\newline

\begin{tabular}{ |p{2cm}|p{13cm}|  }
\hline
ID & UC\_10 \\\hline
Categoria & User\\\hline
Nome & UC\_CreaPlaylist\\\hline
Priorità & High \\\hline
Attori &  ---- \\\hline
Descrizione & Permette a un utente di creare una propria playlist, senza nessuna risorsa al suo interno.\\\hline
Flusso &  	\begin{enumerate}
			\item L'utente richiede di creare una nuova playlist;
			\item Il sistema verifica sulla base di dati che l'abbonamento sottoscritto dall'utente permetta la creazione di playlist:
			\begin{enumerate}
				\item Se l'abbonamento include il servizio:
				\begin{enumerate}
					\item Il sistema richiede il nome della playlist da creare;
					\item L'utente fornisce la stringa;
					\item Il sistema verifica sulla base di dati che l'utente non abbia già una playlist con lo stesso nome:
					\begin{enumerate}
						\item Se il sistema risponde in maniera positiva, allora la playlist viene creata;
						\item Se il sistema risponde in maniera negativa, allora la playlist non viene creata e l'utente viene avvisato del fallimento dell'operazione.
					\end{enumerate}
				\end{enumerate}
				\item Se l'abbonamento non include il servizio, allora la playlist non viene creata e l'utente viene avvisato del fallimento dell'operazione.
			\end{enumerate}
		\end{enumerate}\\\hline
Pre-condizioni & L'utente è registrato sulla piattaforma, ha effettuato con successo il Login,  l'abbonamento sottoscritto permette la creazione di playlist e non esiste già una playlist associato allo stesso utente con lo stesso nome.\\\hline
Post-condizioni & Esiste una playlist con il nome specificato associata all'utente.\\\hline
\end{tabular}
\label{table_use_case:10}\newline

\begin{tabular}{ |p{2cm}|p{13cm}|  }
\hline
ID & UC\_11 \\\hline
Categoria & User\\\hline
Nome & UC\_AggiungiRisorsaAPlaylist\\\hline
Priorità & High \\\hline
Attori &  ---- \\\hline
Descrizione & Permette a un utente di aggiungere una risorsa ad una sua playlist.\\\hline
Flusso &  	\begin{enumerate}
			\item L'utente richiede di aggiungere una risorsa ad una sua playlist;
			\item Il sistema richiede il nome della risorsa e il nome della playlist;
			\item L'utente invia le informazioni richieste;
			\item Il sistema verifica sulla base di dati che tra le playlist associate all'utente sia presente quella con il nome comunicato:
			\begin{enumerate}[  ]
				\item \textbf{\# alt1}: Se il sistema risponde in maniera positiva, allora la risorsa viene agigunta alla playlist;
				\item \textbf{\# alt2}: Se il sistema risponde in maniera negativa, allora la risorsa non viene  agigunta alla playlist e l'utente viene avvisato del fallimento dell'operazione.
			\end{enumerate}
		\end{enumerate}\\\hline
Pre-condizioni & L'utente è registrato sulla piattaforma, ha effettuato con successo il Login e ha almeno 1 playlist a se associata.\\\hline
Post-condizioni & All'interno della playlist è presente un riferimento alla risorsa scelta.\\\hline
\end{tabular}
\label{table_use_case:11}\newline

\begin{tabular}{ |p{2cm}|p{13cm}|  }
\hline
ID & UC\_12 \\\hline
Categoria & User\\\hline
Nome & UC\_RimuoveiRisorsaDaPlaylist\\\hline
Priorità & High \\\hline
Attori &  ---- \\\hline
Descrizione & Permette a un utente di rimuovere una risorsa da una sua playlist..\\\hline
Flusso &  	\begin{enumerate}
			\item L'utente richiede di rimuovere una risorsa da una sua playlist;
			\item Il sistema richiede il nome della risorsa e il nome della playlist;
			\item L'utente invia le informazioni richieste;
			\item Il sistema verifica sulla base di dati che tra le playlist associate all'utente sia presente quella con il nome comunicato:
			\begin{enumerate}[  ]
				\item \textbf{\# alt1}: Se il sistema risponde in maniera positiva, allora la risorsa viene agigunta alla playlist;
				\item \textbf{\# alt2}: Se il sistema risponde in maniera negativa, allora la risorsa non viene  agigunta alla playlist e l'utente viene avvisato del fallimento dell'operazione.
			\end{enumerate}
		\end{enumerate}\\\hline
Pre-condizioni & L'utente è registrato sulla piattaforma, ha effettuato con successo il Login e ha almeno 1 playlist a se associata.\\\hline
Post-condizioni & All'interno della playlist è presente un riferimento alla risorsa scelta.\\\hline
\end{tabular}
\label{table_use_case:12}\newline

\begin{tabular}{ |p{2cm}|p{13cm}|  }
\hline
ID & UC\_13 \\\hline
Categoria & User\\\hline
Nome & UC\_PubblicaRisorsa\\\hline
Priorità & High \\\hline
Attori &  ---- \\\hline
Descrizione & Permette a un utente di pubblicare sulla piattaforma una nuova risorsa.\\\hline
Flusso &  	\begin{enumerate}
			\item L'utente richiede di caricare una nuova risorsa sulla piattaforma;
			\item Il sistema verifica sulla base di dati che l'abbonamento sottoscritto dall'utente permetta la pubblicazione di nuove risorse:
			\begin{enumerate}
				\item Se l'abbonamento include il servizio:
				\begin{enumerate}
					\item Il sistema richiede il Titolo della risorsa da caricare e informazioni varie come Descrizione Breve,  Descrizione Completa, Tipo, Categoria, Pubblico Consigliato e Collezione (quest'ultima potrebbe non esserci);
					\item L'utente fornisce i dati richiesti;
					\item Il sistema verifica sulla base di dati che l'utente non abbia già pubblicato una risorsa gli stessi dati:
					\begin{enumerate}
						\item Se il sistema risponde in maniera positiva, allora la risorsa viene pubblicata;
						\item Se il sistema risponde in maniera negativa, allora  la risorsa non viene pubblicata e l'utente viene avvisato del fallimento dell'operazione.
					\end{enumerate}
				\end{enumerate}
				\item Se l'abbonamento non include il servizio, allora la risorsa non viene pubblicata e l'utente viene avvisato del fallimento dell'operazione.
			\end{enumerate}
		\end{enumerate}\\\hline
Pre-condizioni & L'utente è registrato sulla piattaforma, ha effettuato il Login.\\\hline
Post-condizioni & La piattaforma fornisce una nuova Risorsa con i dati forniti dall'utente (ovviamente la risorsa è associata all'utente) .\\\hline
\end{tabular}
\label{table_use_case:13} \newline

\begin{tabular}{ |p{2cm}|p{13cm}|  }
\hline
ID & UC\_14 \\\hline
Categoria & User\\\hline
Nome & UC\_RimuoviRisorsa\\\hline
Priorità & High \\\hline
Attori &  ---- \\\hline
Descrizione & Permette a un utente di rimuoere dalla piattaforma una risorsa.\\\hline
Flusso &  	\begin{enumerate}
			\item L'utente richiede di rimuoere dalla piattaforma una risorsa;
			\item Il sistema verifica sulla base di dati che l'abbonamento sottoscritto dall'utente permetta l'operazione:
			\begin{enumerate}
				\item Se l'abbonamento include il servizio:
				\begin{enumerate}
					\item Il sistema richiede il Titolo della risorsa da caricare e informazioni varie come Descrizione Breve,  Descrizione Completa, Tipo, Categoria, Pubblico Consigliato e Collezione;
					\item L'utente fornisce i dati richiesti;
					\item Il sistema verifica sulla base di dati che l'utente abbia pubblicato una risorsa gli stessi dati:
					\begin{enumerate}
						\item Se il sistema risponde in maniera positiva, allora la risorsa viene rimossa;
						\item Se il sistema risponde in maniera negativa, allora  la risorsa non viene rimossa e l'utente viene avvisato del fallimento dell'operazione.
					\end{enumerate}
				\end{enumerate}
				\item Se l'abbonamento non include il servizio, allora la risorsa non viene rimossa e l'utente viene avvisato del fallimento dell'operazione.
			\end{enumerate}
		\end{enumerate}\\\hline
Pre-condizioni & L'utente è registrato sulla piattaforma, ha effettuato il Login.\\\hline
Post-condizioni & La risorsa non è più accessibile da nessun servizio e risulta inoltre rimossa da ogni playlist .\\\hline
\end{tabular}
\label{table_use_case:14}\newline

\begin{tabular}{ |p{2cm}|p{13cm}|  }
\hline
ID & UC\_15 \\\hline
Categoria & User\\\hline
Nome & UC\_ModificaRisorsa\\\hline
Priorità & High \\\hline
Attori &  ---- \\\hline
Descrizione & Permette a un utente di modificare una propria risorsa.\\\hline
Flusso &  	\begin{enumerate}
			\item L'utente richiede di modificare una risorsa;
			\item Il sistema verifica sulla base di dati che l'abbonamento sottoscritto dall'utente permetta l'operazione:
			\begin{enumerate}
				\item Se l'abbonamento include il servizio:
				\begin{enumerate}
					\item Il sistema richiede il Titolo della risorsa da caricare e informazioni varie come Descrizione Breve,  Descrizione Completa, Tipo, Categoria, Pubblico Consigliato e Collezione;
					\item L'utente fornisce i dati richiesti;
					\item Il sistema verifica sulla base di dati che l'utente abbia pubblicato una risorsa gli stessi dati:
					\begin{enumerate}
						\item Se il sistema risponde in maniera positiva:
							\subitem A.1. Il sistema chiede i nuovi valori dei dati;
							\subitem A.2. L'utente fornisce i dati richiesti;
							\subitem A.3. Il sistema aggiorna i dati della risorsa.
						\item Se il sistema risponde in maniera negativa, allora  la risorsa non viene modificata e l'utente viene avvisato del fallimento dell'operazione.
					\end{enumerate}
				\end{enumerate}
				\item Se l'abbonamento non include il servizio, allora la risorsa non viene modificata e l'utente viene avvisato del fallimento dell'operazione.
			\end{enumerate}
		\end{enumerate}\\\hline
Pre-condizioni & L'utente è registrato sulla piattaforma, ha effettuato il Login.\\\hline
Post-condizioni & La risorsa è aggiornata con i nuovi dati.\\\hline
\end{tabular}
\label{table_use_case:15}\newline

%DA PENSARE BENE SE E COME METTERE I SEGUENTI USE-CASE: %
%NUOVO SERVIZIO - ELIMINA SERVIZIO - DOWNLOAD RISORSA%

\end{center}