
\subsection{Requisiti sui dati}
\begin{enumerate}

\item \hypertarget{AnReqUt}{Degli utenti registrati si vuole mantenere:}
	\begin{enumerate}[label*=\arabic*.]
	\item Nome
	\item Cognome
	\item Data di nascita
	\item Username
	\item Password (hash)
	\item Email di recupero
	\item data di registrazione
	\item Metodi di pagamento per pagare gli abbonamenti
	\item Eventuali metodi di pagamento per ricevere pagamenti
	\item Stato dell'account (può essere attivo, sospeso, ecc.)
	\item La lista degli abbonamenti sottoscritti. Di ogni abbonamento sottoscritto interessa:
		\begin{enumerate}[label*=\arabic*.]
		\item Data di sottoscrizione
		\item \hyperlink{AnReqPianiAbb}{Piano di abbonamento a cui fa riferimento}
		\end{enumerate}
	\end{enumerate}

\item \hypertarget{AnReqPianiAbb}{Dei piani di abbonamento si vuole mantenere:}
	\begin{enumerate}[label*=\arabic*.]
	\item Nome
	\item Prezzo
	\item Durata
	\item \hyperlink{AnReqServizi}{Lista dei servizi offerti}
	\end{enumerate}

\item \hypertarget{AnReqServizi}{Dei servizi si vuole mantenere:}
	\begin{enumerate}[label*=\arabic*.]
	\item Codice identificativo
	\item Descrizione
	\end{enumerate}

\item  \hypertarget{AnReqProdMult}{Dei prodotti multimediali interessa:}
	\begin{enumerate}[label*=\arabic*.]
	\item Titolo
	\item Descrizione
	\item Età minima per visionarlo
	\item Genere
	\item Visibilità (può essere privato, visibile solamente ad alcuni utenti o pubblico)
	\item \hyperlink{AnReqUt}{Proprietario del prodotto (utente che lo carica)}
	\item Si dividono in:
		\begin{enumerate}[label*=\arabic*.]
		\item \hypertarget{AnReqVideo}{Prodotti video, di cui interessa:}
			\begin{enumerate}[label*=\arabic*.]	
			\item File del video vero e proprio
			\item File dell'audio (eventualmente più di uno per supportare più lingue)
			\item File dei sottotitoli (opzionali)
			\end{enumerate}
		\item \hypertarget{AnReqMusicali}{Prodotti musicali, di cui interessa:}
			\begin{enumerate}[label*=\arabic*.]
			\item File della musica
			\item File delle lyrics (opzionale)
			\item File del video musicale (opzionale)
			\end{enumerate}
		\end{enumerate}
	\end{enumerate}

\item \hypertarget{AnReqPlaylist}{Delle playlist interessa}:
	\begin{enumerate}[label*=\arabic*.]
	\item Nome
	\item Descrizione
	\item Visibilità (può essere privata, visibile solamente ad alcuni utenti o pubblica)
	\item \hyperlink{AnReqUt}{Proprietario della playlist (utente che la crea)}
	\item \hyperlink{AnReqProdMult}{Lista dei prodotti multimediali che contiene}
	\end{enumerate}

\item \hypertarget{AnReqSerieTv}{Delle serie tv interessa:}
	\begin{enumerate}[label*=\arabic*.]
	\item \hyperlink{AnReqPlaylist}{Sono delle playlist tali che:}
		\begin{enumerate}[label*=\arabic*.]
		\item Tutti i prodotti sono video
		\item Il proprietario della playlist e il proprietario di tutti i prodotti che contiene, sono lo stesso utente
		\item Ogni prodotto appartiene a una sola serie tv
		\end{enumerate}
	\item Genere
	\end{enumerate}

\item \hypertarget{AnReqAlbum}{Degli album interessa:}
	\begin{enumerate}[label*=\arabic*.]
	\item \hyperlink{AnReqPlaylist}{Sono delle playlist tali che:}
		\begin{enumerate}[label*=\arabic*.]
		\item Tutti i prodotti sono musicali
		\item Il proprietario della playlist e il proprietario di tutti i prodotti che contiene, sono lo stesso utente
		\end{enumerate}
	\item Genere
	\end{enumerate}

\item \hypertarget{AnReqVisual}{Delle visualizzazioni interessa:}
	\begin{enumerate}[label*=\arabic*.]
	\item \hyperlink{AnReqProdMult}{Il prodotto multimediale visualizzato}
	\item \hyperlink{AnReqUt}{L'utente che visualizza il prodotto}
	\item L'istante in cui avviene la visualizzazione
	\item Durante una visualizzazione è possibile:
		\begin{enumerate}[label*=\arabic*.]
		\item Dare un voto da 1 a 5 al prodotto visualizzato
			\begin{enumerate}[label*=\arabic*.]
			\item Se il prodotto fa parte di una serie tv o di un album, è possibile valutare la serie tv o l'album oltre che il singolo prodotto
			\end{enumerate}
		\item Commentare il prodotto visualizzato
			\begin{enumerate}[label*=\arabic*.]
			\item Se il prodotto fa parte di una serie tv o di un album, è possibile commentare la serie tv o l'album oltre che il singolo prodotto
			\end{enumerate}
		\end{enumerate}
	\end{enumerate}

\item \hypertarget{AnReqSegnalazioni}{Delle segnalazioni interessa:}
	\begin{enumerate}[label*=\arabic*.]
	\item \hyperlink{AnReqUt}{L'utente} che effettua la segnalazione
	\item \hyperlink{AnReqProdMult}{Il prodotto} segnalato
	\item Motivo della segnalazione
	\end{enumerate} 

\item \hypertarget{AnReqCodaRip}{Della coda di riproduzione interessa:}
	\begin{enumerate}[label*=\arabic*.]
	\item è locale ad ogni utente
	\item Lista ordinata dei \hyperlink{AnReqUt}{prodotti} da riprodurre
	\end{enumerate} 


\end{enumerate}


\subsection{Requisiti Funzionali}
\begin{enumerate}
	
	% requisiti relativi agli amministratori

	\item REQ\_F\_GestisciPianiAbbonamento
		\begin{enumerate}[label*=\arabic*.]      
		\item REQ\_F\_CreaAbbonamento
			\begin{itemize}	
			\item Priorità: Primaria
			\item Descrizione: permette agli amministratori di creare un nuovo piano di abbonamento (inizialmente con nessun servizio associato)
			\end{itemize}
		\item REQ\_F\_EliminaAbbonamento
			\begin{itemize}	
			\item Priorità: Primaria
			\item Descrizione: permette agli amministratori di eliminare un piano di abbonamento
			\end{itemize}
			
		\item REQ\_F\_AggiungiServizioAbbonamento
			\begin{itemize}	
			\item Priorità: Primaria
			\item Descrizione: permette agli amministratori di aggiungere un certo servizio ad un certo piano di abbonamento
			\end{itemize}
	
		\item REQ\_F\_RimuoviServizioAbbonamento
			\begin{itemize}	
			\item Priorità: Primaria
			\item Descrizione: permette agli amministratori di rimuovere un certo servizio ad un certo piano di abbonamento
			\end{itemize}

		\item REQ\_F\_RecuperaServiziAbbonamento
			\begin{itemize}	
			\item Priorità: Primaria
			\item Descrizione: permette agli amministratori di recuperare tutti i servizi associati ad un certo abbonamento
			\end{itemize}
	
		\item REQ\_F\_RecuperaPianiAbbonamentoUtente
			\begin{itemize}	
			\item Priorità: Primaria
			\item Descrizione: permette di recuperare tutti i piani di abbonamento sottoscritti da un certo utente
			\end{itemize}

		\item REQ\_F\_RecuperaServizi
			\begin{itemize}	
			\item Priorità: Primaria
			\item Descrizione: permette agli amministratori di recuperare tutti i servizi esistenti
			\end{itemize}
		\end{enumerate}

	\item REQ\_F\_EffettuaPagamento
		\begin{itemize}	
		\item Priorità: Primaria
		\item Descrizione: permette agli amministratori di effettuare pagamenti verso utenti che hanno il servizio di pubblicare contenuti
		\end{itemize}
	\item REQ\_F\_SospendiAccount
		\begin{itemize}	
		\item Priorità: Primaria
		\item Descrizione: permette agli amministratori di sospendere un account di un utente, per un certo periodo o permanentemente  (per svariate motivazioni)	
		\end{itemize}

	%requisiti relativi agli utenti
	
	%registrazione
	\item REQ\_F\_GestisciAccount
		\begin{enumerate}[label*=\arabic*.]
		\item REQ\_F\_EffettuaRegistrazione	
			\begin{itemize}	
			\item Priorità: Primaria
			\item Descrizione: permette la registrazione di un nuovo utente (richiedendo vari dati quali nome, cognome, ecc.)
			\end{itemize}
		\item REQ\_F\_ModificaProfilo
			\begin{itemize}	
			\item Priorità: Secondaria
			\item Descrizione: permette la modifica del profilo di un utente (può cambiare alcuni dati inseriti in fase di registrazione)
			\end{itemize}

		\item REQ\_F\_EffettuaLogin
			\begin{itemize}	
			\item Priorità: Primaria
			\item Descrizione: permette ad un utente registrato di accedere al proprio account
			\end{itemize}

		\item REQ\_F\_EffettuaLogout
			\begin{itemize}	
			\item Priorità: Primaria
			\item Descrizione: permette ad un utente di uscire dal proprio account
			\end{itemize}
		\end{enumerate}

	%ricerche
	\item REQ\_F\_GestisciRicerche
		\begin{enumerate}[label*=\arabic*.]
		\item REQ\_F\_RicercaContenuto
			\begin{itemize}	
			\item Priorità: Primaria
			\item Descrizione: permette la ricerca di contenuti (sia \hyperlink{AnReqProdMult}{prodotti} che \hyperlink{AnReqPlaylist}{playlist}) coerenti con una stringa digitata dall'utente
			\end{itemize}
		\item REQ\_F\_RicercaPopolari
			\begin{itemize}
			\item Priorità: Secondaria
			\item Descrizione: permette la ricerca di contenuti (sia \hyperlink{AnReqProdMult}{prodotti} che \hyperlink{AnReqPlaylist}{playlist}) ritenuti popolari in base a vari criteri interni alla piattaforma 
			\end{itemize}	
		
		\item REQ\_F\_SuggerisciContenuti
			\begin{itemize}	
			\item Priorità: Secondaria
			\item Descrizione: suggerisce dei contenuti (sia \hyperlink{AnReqProdMult}{prodotti} che \hyperlink{AnReqPlaylist}{playlist})  ad un utente, in base agli ultimi contenuti \hyperlink{AnReqVisual}{visualizzati}
			\end{itemize}
		\end{enumerate}		

	%abbonamenti
	\item REQ\_F\_GestisciSottoscrizioniAbbonamenti
		\begin{enumerate}[label*=\arabic*.]			
		\item REQ\_F\_SottoscriviAbbonamento
			\begin{itemize}
			\item Priorità: Primaria	
			\item Descrizione: permette ad un utente registrato di sottoscrivere un nuovo abbonamento (relativo a un piano di abbonamento esistente)
			\end{itemize}

		\item REQ\_F\_DisdiciAbbonamento
			\begin{itemize}	
			\item Priorità: Primaria
			\item Descrizione: permette ad un utente registrato di disdire un abbonamento sottoscritto (evitando quindi il prossimo rinnovo, l'abbonamento rimane comunque valido fino alla scadenza)
			\end{itemize}
				
		\item REQ\_F\_CambiaAbbonamento
			\begin{itemize}	
			\item Priorità: Secondaria
			\item Descrizione: permette ad un utente registrato di sostituire un abbonamento in corso di validità con un altro abbonamento, pagando eventualmente un sovrapprezzo
			\end{itemize}
		\end{enumerate}
		%cambiare abbonamento?? bisogna chiarire...		
	
	%pubblicazione contenuti: l'utente può caricare video (che poi si suddividono in film, serie tv, ecc), canzoni. 
	%per video: serve il video (muto), e le varie tracce audio (diverse lingue) [internamente il sistema genererà un mkv con video e le tracce audio]
	%per musica: serve la traccia audio, un eventuale file dei lyrics (con timestamp per poterlo sincronizzare), un eventuale video musicale (che viene unito all'audio, come per i video)
	
	\item REQ\_F\_CreaProdotto
		\begin{itemize}
		\item Priorità: Primaria
		\item Descrizione: permette agli utenti (che hanno l'opportuno servizio) di creare un nuovo \hyperlink{AnReqProdMult}{prodotto} da pubblicare sulla piattaforma
		\end{itemize}
		\begin{enumerate}[label*=\arabic*.]
		\item REQ\_F\_CompilaInformazioniDiBase
			\begin{itemize}
			\item Priorità: Primaria
			\item Descrizione: permette di riempire campi di base del \hyperlink{AnReqProdMult}{prodotto} (come titolo, descrizione, ecc)
			\end{itemize}
		\item REQ\_F\_CreaVideo
			\begin{itemize}
			\item Priorità: Primaria
			\item Descrizione: permette agli utenti (che dispongono dell'opportuno servizio) di creare un nuovo \hyperlink{AnReqVideo}{video} da pubblicare sulla piattaforma
			\end{itemize}
			\begin{enumerate}[label*=\arabic*.]
			\item REQ\_F\_AggiungiSottotitoli
				\begin{itemize}
				\item Priorità: Secondaria
				\item Descrizione: permette di aggiungere un file dei sottotitoli al video
				\end{itemize}
			\item REQ\_F\_CaricaFileMultimedialiVideo
				\begin{itemize}
				\item Priorità: Primaria
				\item Descrizione: permette di caricare le risorse multimediali per un video (è obbligatorio un file video, possono poi essere aggiunte varie tracce audio in lingue diverse)
				\end{itemize}
			\end{enumerate}
	
		\item REQ\_F\_CreaCanzone
			\begin{itemize}
			\item Priorità: Primaria
			\item Descrizione: permette agli utenti (che dispongono dell'opportuno servizio) di creare un nuovo \hyperlink{AnReqMusicali}{prodotto musicale} da pubblicare sulla piattaforma
			\end{itemize}
			\begin{enumerate}[label*=\arabic*.]
			\item REQ\_F\_AggiungiLyrics
				\begin{itemize}
				\item Priorità: Secondaria
				\item Descrizione: permette di aggiungere un file delle lyrics
				\end{itemize}
			\item REQ\_F\_CaricaFileMultimedialiCanzone
				\begin{itemize}
				\item Priorità: Primaria
				\item Descrizione: permette di caricare le risorse multimediali per una canzone (è obbligatorio un file audio, può essere caricato anche un video multimediale)
				\end{itemize}
			\end{enumerate}	
		
		\item REQ\_F\_CambiaStatoPubblicazione
			\begin{itemize}
			\item Priorità: Primaria
			\item Descrizione: permette di modificare lo stato di pubblicazione del prodotto (può essere reso pubbico, oppure lo si può rimettere privato/in bozza)
			\end{itemize}		
		\end{enumerate}
	
%riproduzione di risorse: chiamiamo riproduciRisorsa, poi questa penserà a usare RiproduciVideo o RiproduciMusica a seconda della risorsa, in RiproduciMusica è necessario trovare la traccia video adeguata, oppure facciamo che la musica in realtà deve essere caricata unitamente al video?
	\item REQ\_F\_RiproduciProdotto
		\begin{itemize}
		\item Priorità: Primaria	
		\item Descrizione: riproduce una \hyperlink{AnReqProdMult}{prodotto multimediale} (sia esso video, audio)
		\end{itemize}
    		\begin{enumerate}[label*=\arabic*.]      				
		\item REQ\_F\_RiproduciVideo
			\begin{itemize}
			\item Priorità: Primaria
			\item Descrizione: riproduce una \hyperlink{AnReqVideo}{risorsa video}
			\end{itemize}
		\item REQ\_F\_RiproduciMusica
			\begin{itemize}
			\item Priorità: Primaria
			\item Descrizione: riproduce una \hyperlink{AnReqMusicali}{risorsa musicale}, con eventuale traccia video adeguata (lyrics o video musicale)
			\end{itemize}
			
		\item REQ\_F\_PausaPlayer
			\begin{itemize}
			\item Priorità: Primaria
			\item Descrizione: mette in pausa la visualizzazione della risorsa
			\end{itemize}
			
		\item REQ\_F\_SpostaPuntoRiproduzione
			\begin{itemize}
			\item Priorità: Primaria
			\item Descrizione: cambia il punto di riproduzione della risorsa (può essere spostata avanti o indietro)
			\end{itemize}
		\item REQ\_F\_RiproduciAudioInBackground
			\begin{itemize}
			\item Priorità: Secondaria
			\item Descrizione: Riproduce l'audio di un prodotto, questa funzionalità viene attivata quando l'app/sito viene messo in background (la traccia video viene inibita)
			\end{itemize}		
		\end{enumerate}
		
	%playlist... qualsiasi risorsa, o solo audio? Io direi qualsiasi risorsa e poi se la vede l'utente
	\item REQ\_F\_GestisciPlaylist
		\begin{enumerate}[label*=\arabic*.]
		\item REQ\_F\_CreaPlaylist
			\begin{itemize}	
			\item Priorità: Primaria
			\item Descrizione: permette ad utenti registrati di creare una \hyperlink{AnReqPlaylist}{playlist} (inizialmente vuota)
			\end{itemize}
		\item REQ\_F\_AggiungiProdottoAPlaylist
			\begin{itemize}	
			\item Priorità: Primaria
			\item Descrizione: permette ad utenti registrati di aggiungere un \hyperlink{AnReqProdMult}{prodotto} a una loro playlist precedentemente creata
			\end{itemize}

		\item REQ\_F\_RimuoviProdottoDaPlaylist
			\begin{itemize}	
			\item Priorità: Primaria
			\item Descrizione: permette ad utenti registrati di rimuovere un prodotto da una loro playlist
			\end{itemize}

		\item REQ\_F\_RiproduciPlaylist
			\begin{itemize}	
			\item Priorità: Primaria
			\item Descrizione: riproduce i prodotti della playlist, uno dopo l'altro
			\end{itemize}
		
		\item REQ\_F\_CambiaVisibilitàPlaylist
			\begin{itemize}	
			\item Priorità: Primaria
			\item Descrizione: cambia la visibilità della playlist (può essere pubblica, visibile solo all'utente che l'ha creata, o visibile per alcuni account)
			\end{itemize}
		\end{enumerate}		
	
	%serie tv e album gestite come playlist, che però sono pubbliche
	\item REQ\_F\_CreaSerieTv
		\begin{itemize}	
		\item Priorità: Primaria
		\item Descrizione: una \hyperlink{AnReqSerieTv}{serie tv} è una playlist di video con dei vincoli: tutti gli episodi devono essere caricati dall'account che crea la serie tv, e un episodio non può appartenere a più serie tv.
		\end{itemize}

	\item REQ\_F\_CreaAlbum
		\begin{itemize}	
		\item Priorità: Primaria
		\item Descrizione: un \hyperlink{AnReqAlbum}{album} è una playlist di canzoni con dei vincoli: tutte le canzoni devono essere caricate dall'account che crea l'album.
		\end{itemize}

	%commenti/votazioni
	\item REQ\_F\_GestisciValutazioni
		\begin{enumerate}[label*=\arabic*.]
		\item REQ\_F\_VotaContenuto
			\begin{itemize}	
			\item Priorità: Secondaria
			\item Descrizione: permette di votare un contenuto (singoli \hyperlink{AnReqProdMult}{prodotti} o anche \hyperlink{AnReqPlaylist}{playlist} pubbliche, e quindi album e serie tv), agli utenti che l'hanno visionata
			\end{itemize}
		
		\item REQ\_F\_CommentaContenuto
			\begin{itemize}	
			\item Priorità: Secondaria
			\item Descrizione: permette di commentare un contenuto (singoli \hyperlink{AnReqProdMult}{prodotti} o anche \hyperlink{AnReqPlaylist}{playlist} pubbliche, e quindi album e serie tv), agli utenti che l'hanno visionata
			\end{itemize}
		\item REQ\_F\_RimuoviCommento
			\begin{itemize}	
			\item Priorità: Secondaria
			\item Descrizione: permette ad un utente di rimuovere un proprio commento su un contenuto
			\end{itemize}
		\end{enumerate}
	
	%segnalazioni
	\item REQ\_F\_GestisciSegnalazioni
		\begin{enumerate}[label*=\arabic*.]
		\item REQ\_F\_SegnalaProdotto
			\begin{itemize}
			\item Priorità: Primaria		
			\item Descrizione: permette di \hyperlink{AnReqSegnalazioni}{segnalare} un \hyperlink{AnReqProdMult}{prodotto} (con relativa motivazione della segnalazione)
			\end{itemize}
		\item REQ\_F\_OttieniSegnalazioni
			\begin{itemize}	
			\item Priorità: Primaria
			\item Descrizione: permette agli amministratori di ottenere le segnalazioni per un certo utente
			\end{itemize}
		\end{enumerate}

	\item REQ\_F\_OttieniCronologia
		\begin{itemize}	
		\item Priorità: Secondaria
		\item Descrizione: ottiene gli ultimi contenuti \hyperlink{AnReqVisual}{visualizzati} da un utente (si può specificare quanti contenuti si vogliono, o di quanti giorni si vogliono)
		\end{itemize}

	\item REQ\_F\_DownloadProdotto
		\begin{itemize}	
		\item Priorità: Secondaria
		\item Descrizione: effettua il download di un certo \hyperlink{AnReqProdMult}{prodotto}, in modo da poterlo riprodurre anche in assenza di connessione internet
		\end{itemize}
	
	\item REQ\_F\_VisualizzaPubblicità
		\begin{itemize}
		\item Priorità: Primaria
		\item Descrizione: riproduce uno spot pubblicitario (avviene solo per account che non hanno alcuni servizi, e in determinate situazioni)
		\end{itemize}
	
	%requisiti svolti internamente dal sistema
		
	\item REQ\_F\_CalcolaQualitàContenuto
		\begin{itemize}	
		\item Priorità: Secondaria
		\item Descrizione: calcola il voto di un certo contenuto, in base ai voti ricevuti
		\end{itemize}
	
	\item REQ\_F\_GestisciCodaDiRiproduzione
		\begin{enumerate}[label*=\arabic*.]
		\item REQ\_F\_AggiungiProdottoAllaCoda
			\begin{itemize}	
			\item Priorità: Secondaria
			\item Descrizione: aggiunge un prodotto in fondo alla coda di riproduzione
			\end{itemize}
		\item REQ\_F\_RimuoviProdottoDallaCoda
			\begin{itemize}	
			\item Priorità: Secondaria
			\item Descrizione: rimuove il prodotto in testa alla coda di riproduzione
			\end{itemize}
		\end{enumerate}


\end{enumerate}

\subsection{Requisiti Non Funzionali}

\begin{enumerate}	
	\item REQ\_NF\_TempoDiRisposta
		\begin{itemize}
		\item Priorità: Primaria
		\item Descrizione: la piattaforma deve effettuare le richieste degli utenti in tempi ragionevoli (pochi secondi al massimo), nel caso in cui questo non fosse momentaneamente possibile, è necessario comunicarlo all'utente precisando l'errore
		\end{itemize}

	\item REQ\_NF\_Privacy
		\begin{itemize}
		\item Priorità: Primaria
		\item Descrizione: la piattaforma dovrà garantire la riservatezza di dati sensibili degli utenti
		\end{itemize}

	\item REQ\_NF\_Compatibilità
		\begin{itemize}
		\item Priorità: Secondaria
		\item Descrizione: la piattaforma dovrà essere compatibile su dispositivi Android e iOS (per quanto riguarda l'app), e sui maggiori browser: Firefox, Chrome, Safari (per quanto riguarda la web app)
		\end{itemize}
	
	\item REQ\_NF\_Backup
		\begin{itemize}
		\item Priorità: Primaria
		\item Descrizione: la piattaforma deve effettuare backup dei contenuti caricati dagli utenti (insieme ai loro dati), in modo da garantire la disponibilità e la persistenza dei dati
		\end{itemize}

	\item REQ\_NF\_ContenutiNonAppropriati
		\begin{itemize}
		\item Priorità: Primaria
		\item Descrizione: le segnalazioni inviate dagli utenti devono essere analizzate tempestivamente, in modo da rimuovere eventuali contenuti non appropriati
		\end{itemize}

\end{enumerate}

\noindent{\LARGE \textbf{Revisioni 6}} \\ \\
\begin{tabular}{|c | c | c | c|} 
 	\hline
	 Numero & Data & Descrizione \\ [0.5ex] 
	\hline\hline
	1 & 06/02/2020 & Stesura iniziale \\ 
	\hline
\end{tabular}
