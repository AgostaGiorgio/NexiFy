
\subsection{Introduzione}
In questa sezione verrà fatta una descrizione del sistema che scende in maggiori dettagli rispetto a quanto
si è fatto in precedenza nel documento.

\subsection{Interazione con gli utenti}
La piattaforma è indirizzata principalmente alla fruzione di contenuti, quali brani musicali, film e serie tv.\\
Al momento della consegna iniziale, i contenuti potranno essere visionati attraverso web app e app per mobile.
La web app è pensata per l'uso in casa, in quanto da la possibilità di visualizzare film e serie tv su schermi
di diverse dimensioni ed è accessibile anche da smart tv. L'app per mobile, invece, è pensata per una
interazione in mobilità, offrendo un'interfaccia più versatile e la possibilità di riprodurre brani musicali
quando l'app si trova in modalità background nel sistema operativo mobile.\\
Per quanto rigurda il player, esso viene descritto in dettaglio nella "Visione del Sistema".\\
Sulla piattaforma saranno inizialmente presenti dei listing dei contenuti che sono stati caricati, organizzati per 
categoria e tipo di risorsa multimediale (es. brano musicale o video). I contenuti potranno essere trovati
con una ricerca per titolo, autore, ecc.. Inoltre, agli utenti saranno suggerite delle nuove uscite,
oppure altri elementi che potrebbero interessargli.\\

\subsection{Accesso alla piattaforma}
Gli utenti che vorranno usufruire dei servizi offerti, dovranno sottoscrivere un abbonamento. Esisteranno diversi
tipi di abbonamento che saranno svincolati dagli insiemi di servizi a cui danno accesso. Quindi, una volta definiti
i servizi, gli amministratori della piattaforma potranno modellare degli abbonamenti che includono alcuni o
tutti questi servizi. Di conseguenza, i tipi di abbonamento offerti alla consegna potranno non essere più
disponibili dopo del tempo e potranno esserne costruiti di nuovi.
Le prime offerte sono descritte nella "Configurazione del Sistema alla Consegna".

\subsection{Pubblicazione dei contenuti}
NexiFy da la possibilità a chiunque, in seguito alla sottoscrizione di un contratto da Partner,
di caricare contenuti sulla piattaforma. In questo modo viene permesso ai creatori amatoriali o emergenti,
di pubblicare i loro lavori, senza doversi necessariamente affidare a case produttrici od aver a disposizione
un grande budget.\\
Al momento della pubblicazione, verrà richiesto al Partner di inserire delle informazioni riguardanti
il contenuto proposto. Tra queste vi sono:
\begin{itemize}
    \item Titolo 
    \item Descrizione breve
    \item Descrizione completa
    \item Tipo (es. film o brano musicale)
    \item Categoria
    \item Collezione, se applicabile
    \item Pubblico consigliato
\end{itemize}
In alcuni casi, i singoli elementi caricati verranno organizzati in collezioni. Ad esempio, gli espisodi delle
serie TV appartengono alle serie stesse, oppure, i brani musicali appartengono ad album.\\
Alcuni contenuti potrebbero non essere indicati per un pubblico giovanile, quindi il Partner dovrà rispondere ad
un questionario che lo aiuterà a classificare la sua pubblicazione e a generare delle opportune etichette
relative alle fasce d'età consentite. Nel caso in cui un Partner dichiarasse il falso nei questionari, ci 
saranno ripercussioni più o meno gravi sul suo account che possono arrivare fino alla sospensione definitiva
di quest'ultimo.
